\section*{RSA}

\subsection*{Fermats Little Theorem}
Exploring RSA cryptosystem`s foundation, focus on Fermat`s little theorem: for prime P and any number Z that is 1 \textless{}= Z \textless{}= P-1 and GCD(Z,P) = 1, then Z\textasciicircum{}(P-1) mod P = 1.
This theorem`s proof hinges on the properties of prime numbers, notably that elements relatively prime to P have multiplicative inverses modulo P\@.
This concept underpins not only RSA but also algorithms for primality testing.
The theorem`s proof, described as both beautiful and reasonably challenging, leverages the relationship between primes and their inverses.
The theorem is distinct from the more complex Fermat`s Last Theorem, famously proved by Andrew Wiles centuries after Fermat`s initial claim.

\subsection*{Fermats Thm.  Proof}
Fermat`s Little Theorem states that for a prime p and any integer z between 1 and p-1, z\textasciicircum{}(p-1) mod p equals 1.
To prove this, we define set S containing integers 1 through p-1.
S` is another set created by multiplying each element in S by z and taking mod p.
Through an example with p=7 and z=4, we see S` is a permutation of S--different order but contains the same elements.
Establishing that S and S` are permutations of one another is a step towards proving the theorem.

\subsection*{Proof  Key Lemma}
S and S` are sets with elements related to a prime number P\@.
S consists of numbers 1 to P-1.
S` has elements of the form kZ mod P for k=1 to P-1.
To prove S and S` are identical, it is shown that S` elements are distinct and non-zero.
For distinctness, assuming two elements are equal leads to a contradiction using the fact that every number 1 to P-1 has an inverse modulo P\@.
Multiplying both sides of the congruence by the inverse of Z shows that equal elements must have the same index, contradicting the presumption of distinct indices, proving all elements in S` are distinct.
For non-zero proof, if any S` element is zero, multiplying by Z`s inverse suggests index I would be zero modulo P, outside the range of 1 to P-1, another contradiction.
Thus, S` elements are non-zero and occupy all values 1 to P-1, implying S equals S` with order as the only differing factor.

\subsection*{Proof  Finishing Up}
Goal to demonstrate Z\textasciicircum{}(P-1)  1 mod P\@.
Examined sets S (1 to P-1) and S` (S elements x Z mod P).
Proved S and S` identical, different orders.
Multiplying S and S` elements separately yields identical products mod P: S gives 1x2x\ldotsx(P-1), S` yields Zx1, Zx2,\ldots, Zx(P-1).
Sides equal mod P, left is $(P-1)!$ and right is Z\textasciicircum{}(P-1) x (P-1)!.
As P is prime, each term 1 to P-1 has an inverse mod P\@.
Multiplying by inverses cancels $(P-1)!$, results in Z\textasciicircum{}(P-1)  1 mod P, confirming the theorem.

\subsection*{Eulers Theorem}
Euler`s theorem extends Fermat`s little theorem from prime numbers to any positive integer N by asserting Z\textasciicircum{}ph(N)  1 mod N, given Z and N are relatively prime (GCD(Z,N)=1)\. ph(N), Euler`s totient function, signifies the count of numbers \textless{}= N that are coprime to N\@.
For a prime P, ph(P) equals P-1, mirroring the case in Fermat`s theorem: raising Z to P-1 yields a remainder of 1 when divided by P\@.
Euler`s theorem thus generalizes Fermat`s by applying a similar concept to a broader set of numbers, N, not limited to primes.

\subsection*{Eulers Totient Quiz Question}
Euler`s theorem examined for N = P * Q, with P and Q as prime numbers.
Focus on determining phi of N (Euler`s totient function value) when N is the product of these primes.
Objective: calculate quantity of numbers between 1 and P * Q that are relatively prime to PQ\@.
Euler`s theorem here is an extension of Fermat`s little theorem when involving two primes.

\subsection*{Eulers Totient Quiz Solution}
Phi\_of\_n, when n=p*q (p and q are prime), is derived by counting the numbers between 1 and pq that are co-prime to pq\. p multiples from p to pq, and q multiples from q to pq, have a gcd of p or q with pq.
Once removed, p*q original count minus the q multiples of p and p multiples of q, with pq double-counted and thus corrected by adding 1, results in phi\_of\_n=(p-1)*(q-1).

\subsection*{Eulers Thm\. for N=pq}
Euler`s Theorem states for N = PQ (P, Q = prime), if Z and N are coprime (GCD(Z, N) = 1), then Z\textasciicircum{}(phi(N)) = 1 mod N, where phi(N) = (P-1)(Q-1).
This is foundational for RSA encryption.
RSA uses prime pair P, Q to create N and leverages (P-1)(Q-1) to generate encryption and decryption keys, enabling secure message exchange.
The theorem underpins the fundamental mechanism of RSA`s cryptographic process.

\subsection*{RSA Alg.
Idea}
RSA relies on the difficulty of factoring the product of two large prime numbers.
The foundation of RSA encryption/decryption begins with Fermat`s Theorem and the concept that a number raised to the product of a specific pair of exponents related to modulo prime numbers will return the original number.
In RSA, a message, Z, encrypted with an exponent, B, and then decrypted with its inverse exponent, C - mutually inverse with respect to mod (P-1), where P is a prime number - yields the original message Z under mod P due to Fermat`s Theorem.
To secure the process against deducing C given B and P, Euler`s Theorem is utilized with two primes, P and Q, and their product N\@.
Encryption and decryption exponents, D and E, are inverses mod (P-1)(Q-1).
Z raised to the power DE under mod N returns Z due to Euler`s Theorem, assuming Z and N are relatively prime.
The public knows E and N but not D - the decryption key - which is privately derived from P, Q and E using the extended Euclidean algorithm.
The RSA algorithm`s security is underpinned by the impracticality of deducing D without knowledge of P and Q, even while knowing their product, N\@.

\subsection*{Crypto Setting}
Alice needs to securely send a message (M) to Bob in the presence of an eavesdropper, Eve.
To accomplish this, Alice encrypts M using an encryption scheme to produce E(M), which is sent to Bob through a communication line visible to Eve.
Bob decrypts E(M) using his private decryption key, which Eve cannot access.
The system used is a public key cryptosystem, RSA\@.
In RSA, Bob generates a public key (N and E) by multiplying two prime numbers (P and Q) and selecting E that is coprime to (P-1)(Q-1).
He broadcasts N and E publicly.
Anyone, including Alice, can send messages to Bob by encrypting them with his public key.
Bob is the only one who knows P and Q, allowing him to compute his private key (D), the modular multiplicative inverse of E modulo (P-1)(Q-1).
D enables Bob to decrypt messages by raising them to the power of D and taking the result modulo N\@.
This process does not require prior private communication between Alice and Bob, as public key encryption allows for secure message transmission without prior key exchange.

\subsection*{RSA Protocol  Keys}
Bob, the receiver in RSA protocol, creates his public and private keys starting with two randomly chosen N-bit prime numbers P and Q\@.
He verifies their primality with an efficient testing algorithm.
Next, Bob selects an encryption exponent E, relatively prime to the product of (P-1)(Q-1).
He initially tries E=3 and uses Euclid`s GCD algorithm to verify relativeness.
If unsuccessful, he continues with the next odd numbers until he finds a suitable E or resets with new primes.
Multiplying P and Q, Bob gets N and can then publish his public key (N, E).
For his private key, he determines D, the modular multiplicative inverse of E relative to (P-1)(Q-1), using the Extended Euclid algorithm.
D remains secret, while N and E are shared publicly, completing his key setup.

\subsection*{RSA Protocol  Encrypting}
Alice desires to send an encrypted message M to Bob.
She retrieves Bob`s public key (N, E) and encrypts M by raising it to power E and then taking modulus N, resulting in the encrypted message Y\@.
For this operation, Alice employs a fast modular exponentiation algorithm for efficiency.
Upon receiving Y, Bob decrypts it using his private key D by raising Y to power D and taking modulus N to recover the original message M\@.
The decryption works because D is chosen as the multiplicative inverse of E mod (P-1)(Q-1), where N equals PQ and P, Q are primes.
Euler`s theorem ensures this process succeeds when M and N are relatively prime (or share a prime factor).
Bob`s decryption process is more computationally intensive due to his large D value, thus necessitating the use of the fast modular exponentiation algorithm.
The RSA algorithm relies on the generation of random prime numbers, P and Q, which underpin the security of the encryption; methods of prime verification are explored subsequently in the text.

\subsection*{RSA  Potential Pitfalls}
RSA encryption depends on the relative primality of the message, m, and the modulus n (product of two large primes P and Q).
If m shares a common factor with n, the cryptosystem can be broken as the greatest common divisor (GCD) provides the shared prime factor.
Encrypted messages (m raised to power e mod n) can expose common factors, enabling adversaries to factorize n using Euclid`s algorithm and ultimately decrypt messages.
To ensure security, m must be less than n but not too small to avoid direct decryption via cube roots in cases where e=3.
Converting text to binary and segmenting into n-bit strings assures m \textless{} n, with leading bit considerations ensuring m`s maximum value is properly lesser than n.
Avoiding small messages can be achieved by padding the message with a sufficiently large random number R\@.
However, when a message is sent multiple times with the same exponent but different moduli, encrypted messages can be used in conjunction to decrypt the original message using the Chinese remainder theorem.
Lastly, the RSA protocol necessitates effective primality testing to select appropriate P and Q, ensuring the continued security of the encryption mechanism.

\subsection*{Recap of RSA}
RSA encryption is fundamentally based on number theory, relying on properties of prime numbers and modular arithmetic.
It utilizes Fermat`s little theorem, which states that if you have a prime number p and another number z, with z and p being relatively prime (having a greatest common divisor of one), then z raised to the power of (p-1) and taken modulo p will equal one.
However, RSA actually employs a generalization of this theorem, known as Euler`s theorem.
Specifically, RSA applies Euler`s theorem to the product of two primes, p and q.
In this scenario, if you take a number z that is relatively prime to N (where N is the multiplication of p and q) and raise z to the power of (p-1)(q-1), modulo N will also yield one.
This power of (p-1)(q-1) is derived from the count of numbers between 1 and N that are relatively prime to N, where N is the product of two distinct prime numbers, p and q.
Understanding this mathematical background is key to grasping RSA algorithm`s encryption and decryption processes.

\subsection*{Recap of RSA \#2}
Euler`s theorem underpins the RSA algorithm.
Start by selecting prime numbers p and q, multiply them to get N\@.
Determine e, coprime to (p-1)(q-1), ensuring gcd(e, (p-1)(q-1)) is one.
An inverse of e exists due to coprimality, defined as d, the private key--calculated using the extended Euclid algorithm.
Publicize the public key N and e for encryption; keep d secret for decryption.
Messages are encrypted as Y by raising them to the power of e and taking mod N, then sent.
Decryption involves raising Y to the power of d, mod N, to retrieve the message m.
Verification of e`s coprimality and inversion uses the Euclid algorithm.
For ease and efficiency in calculations, e is chosen small or fast modular exponentiation via repeated squaring is used.
Security hinges on the difficulty of deriving (p-1)(q-1) from N without factoring N into p and q, assuming factoring is computationally hard.
The effectiveness of RSA depends on secure prime number selection for p and q.

\subsection*{Random Primes}
To securely select primes P and Q for cryptographic purposes, randomly choosing them ensures independence and security against those with prior access to prime tables.
Random prime selection involves generating a random N-bit number R and testing for primality.
R is determined by randomly assigning 0 or 1 to each bit, ensuring each bit and trial is independent.
If R is prime, it`s used; otherwise, the process repeats.
Primes being numerically dense means the probability R is prime is roughly 1/N, so for large bit sizes like 1000 or 2000, a prime is expected after \textasciitilde{}1000 or 2000 trials--a tolerable repetition rate.
The remaining challenge concerns the primality test for R\@.

\subsection*{Primality  Fermats Test}
Deriving a Primality Testing Algorithm based on Fermat`s Little Theorem which states that if r is prime, any z (1 \textless{}= z \textless{} r) raised to r-1 modulo r equals one.
To test if r is prime, find z where z\textasciicircum{}(r-1) modulo r  1; such a z indicates r is composite and serves as a Fermat witness.
All composite numbers have at least one Fermat witness; many have a multitude, excluding Pseudo Primes.
With numerous witnesses, finding them is straightforward, forming the basis of the Primality Testing Algorithm.
The approach includes a strategy for handling Pseudo Primes, thus proving every composite has a Fermat witness.

\subsection*{Trivial Witness}
R being investigated for primality needs Fermat witnesses, Z, proving R composite if Z\textasciicircum{}R-1  1 mod R\@.
All composite Rs have \textgreater{}=1 Fermat witness, shown by choosing a divisor of R as Z\@.
Since Z and R share a GCD \textgreater{} 1, Z has no inverse mod R (inverse exists iff GCD(Z,R)=1).
Assuming Z\textasciicircum{}R-1 = 1 mod R wrongly suggests Z has an inverse (Z\textasciicircum{}R-2), contradicting absence of an inverse, thus proving Z\textasciicircum{}R-1 must not equal 1 mod R and R is composite with a Fermat witness.

\subsection*{Non-Trivial Witness}
A trivial Fermat witness is a number z that shares a non-trivial divisor with a composite number r, immediately indicating r is not prime without needing Fermat`s test.
All composite numbers have trivial Fermat witnesses since they have non-trivial divisors.
The presence of a non-trivial Fermat witness, which is relatively prime to r, however, is significant.
While some composite numbers, called pseudo primes, lack non-trivial Fermat witnesses, most have at least one; when one exists, many exist, making them easy to detect.
This abundance of non-trivial Fermat witnesses is crucial for developing effective primality testing algorithms.

\subsection*{No Non-Trivial Witnesses}
Fermat witnesses are used to determine if a number r is composite; z is a non-trivial Fermat witness if z\textasciicircum{}(r-1)  1 mod r and z and r are relatively prime.
Composite numbers typically have a trivial Fermat witness, but some, known as Carmichael Numbers, do not have non-trivial Fermat witnesses and thus seem prime under Fermat`s test.
Since Carmichael Numbers (like 561, 1105) are rare, they`re usually disregarded, allowing the use of Fermat`s test as a simple method to check for primeness, as long as a non-trivial Fermat witness can be identified for composite numbers.

\subsection*{Primality  Many Witnesses}
Assuming every composite number has at least one non-trivial Fermat witness, it follows that at least half of the numbers 1 through r-1 are witnesses, suggesting that r is composite if for any number z from this set, z\textasciicircum{}(r-1) is not congruent to 1 mod r.
This fact can be exploited to devise a primality test, which ignores Carmichael numbers, based on Fermat`s little theorem.
If z\textasciicircum{}(r-1) is congruent to 1 mod r for any chosen z from the set, then r may be prime; if not, r is definitely composite.

\subsection*{Simple Primality Alg.}
Using a simple algorithm based on Fermat`s Little Theorem to test the primality of an n-bit number r, where a randomly selected Z from the range 1 to r-1 is used for Fermat`s Test.
If Z\textasciicircum{}r-1 \% r equals 1, r might be prime; if not, r is definitively composite as Z acts as a witness.
This method does not address Carmichael numbers but is still efficient for primality testing, with the theorem stating at least half the numbers in the set will witness r`s compositeness if r is indeed composite.

\subsection*{Primality Alg.
Analysis}
Primality testing algorithm always correctly identifies a prime number r, as all z raised to r\_minus\_one modulo r equals one, according to Fermat`s theorem.
When r is composite and not a Carmichael number, the algorithm can still be correct if it finds a Fermat witness (where z raised to r\_minus\_one modulo r doesn`t equal one).
However, the algorithm may falsely identify r as prime if it finds a z that`s not a Fermat witness.
Since at least half of z`s are Fermat witnesses, the probability of a false positive--mistaking r for prime--is at most 1/2.
Current aim is to enhance the algorithm to reduce this false positive rate.

\subsection*{Boosting Success}
Our original primality testing algorithm had a max false positive rate of  1/2 .
To reduce this rate, we now select K random numbers from 1 to R-1 and run a Fermat test on each.
If any chosen number is a Fermat witness (its R-1 power is not congruent to 1 mod R), R is composite.
If all numbers pass, R is likely prime.
The probability of a composite R being incorrectly identified as prime (false positive) is at most  1/2 \textasciicircum{}K\@.
By choosing K=100 from a large set, the false positive probability is significantly reduced to  1/2 \textasciicircum{}100, making the chance of an incorrect prime identification negligibly small and acceptable for cryptographic applications, assuming R is not a Carmichael number.
Dealing with Carmichael numbers requires a slightly different approach, which is explained in the textbook.

\subsection*{Addendum  Pseudoprimes}
Primality testing improved to handle Carmichael numbers using an enhanced algo that goes beyond Fermat`s test, which can fail for these tricky composites.
Example given with 1729, which is a Carmichael number; a standard primality test might falsely categorize it as prime.
Typically, we`d pick a random Z, raise Z to X-1, then mod X, expecting a result other than 1 to confirm compositeness.
For 1729, this does not work--raising 5 to 1728 mod 1729 indeed gives 1, leading to an incorrect assumption of primality.
The novel approach involves factoring out powers of 2 from X-1 until reaching an odd exponent, then using fast modular exponentiation for computing.
When the calculated power of 5 reaches 1 mod X during repeated squaring, rather than stopping, we backtrack to the first occurrence of a value different from 1, which is a `non-trivial` square root of 1 mod X\@.
Non-trivial here means other than +-1, as primes only have these as square roots of 1 mod X\@.
Finding such a non-trivial square root hence confirms the number in question is composite.
While the underlying mathematics for why this works with at least 75\% of possible Z choices is complex, the operational part of the algo remains straightforward.
Essentially, it`s the initial process with an additional check for a non-trivial square root of 1 in cases where the standard Fermat`s test doesn`t detect compositeness in numbers like Carmichael numbers.

