\section*{Max Flow Generalization}

\subsection*{Max-Flow with Demands}
We`re exploring a generalized max-flow problem with added demand constraints for each edge.
In this revised flow network, both capacities and demands define each edge, with the new aim to identify a feasible flow that meets or exceeds these demands while complying with the standard max-flow problem`s constraints.
The central question we`re tackling is whether it`s possible to satisfy all demands with a feasible flow within the network.
Once established, we can then look to optimize the feasible flow to maximize its size.

\subsection*{Feasible Flow Example Question}
Examining a directed graph with start and end vertex for feasible flow, marking edges with demand (green numbers) and capacity (red numbers).
Some edges have specific demand constraints (e.g., at least 2 units, max 8 units), others have none (min 0 units, max 3 units).
Question posed: Is there a feasible flow for this example?

\subsection*{Reduction Overview}
Reducing feasible flow to max-flow problem by using max-flow algorithms as a black box.
Feasible flow seeks any valid flow, whereas max-flow aims for the largest possible one.
To solve feasible flow, transform its input--a directed graph with edge capacities (C of E) and demands (D of E)--into max-flow input using a function, g.
A new graph (G prime) and capacities (C prime) are defined for this purpose, but without demand constraints.
Once the max-flow black box returns a result (F prime), another function, h, is used to convert it back into a feasible flow (F) for the original input graph, respecting the given capacities and demands.
The challenge lies in detailing transformations g and h to manipulate inputs and outputs between the problems.

\subsection*{Reduction  Vertices}
Transforming a feasible flow problem (FFP) into a max flow problem (MFP) by modifying a directed graph G into G prime.
In the example provided, graph G has four vertices with edges labeled with pairs of numbers indicating demand and capacity, respectively.
Graph G prime is created by retaining G and adding vertices, including a new source (s prime) and sink (t prime); original source (s) and sink (t) become internal vertices.
Edge capacities in G prime, denoted C prime, reflect both original capacities and demands from G\@.
The construction of edges and capacities in G prime is not detailed but is crucial for solving the MFP based on the given FFP input.

\subsection*{Original Edges}
Edges from original flow network G are retained in new network G`.
Capacities are modified to reflect both original capacities and demands, ensuring non-negative flow exists if and only if flow \textgreater{}= demand in G\@.
New network flows must meet transformed capacity constraints, analogous to G`s.
A zero flow in G` represents a flow of d (demand) in G, requiring a capacity shift by d units in G`.
Example: an edge originally with capacity 10, after accounting for 3 units of demand, now has a reduced capacity of 7 in G`.
Other edges similarly adjusted to capacities of 1 and 4, reflecting original capacities minus respective demands.

\subsection*{New Edges}
Created network G-prime by adding edges from G; capacities are original capacities shifted by demand.
Need additional edges to ensure valid G-prime flows equate feasible G flows.
Zero G-prime flow equals demand levels in G (e.g., zero to three).
In G, zero inflow at a node doesn`t guarantee valid outflow due to demand.
To rectify, add edges from s-prime to G nodes and G nodes to t-prime\. s-prime to vertex v edges capacity equals v`s demand in G\@.
Edges to t-prime represent demand out of respective vertices.
G-prime now structured to solve max-flow problem; viable solution corresponds to feasible flow in original network G\@.

\subsection*{One More Edge}
In constructing graph G prime, an additional edge from T to S is introduced to adjust for S and T being internal vertices with differing constraints on flow compared to the original graph.
Despite no flow capacity into S and out of T in the current setup, this edge, with infinite capacity, ensures the flow out of S equals the flow into T, balancing the system and completing the construction of G prime.

\subsection*{Saturating Flows}
Max flow size in a graph is limited by the total capacity out of source S-prime or into sink T-prime, which is defined by the total demand D in the original network.
Total demand is calculated by summing the demand on edges or at vertices.
For the constructed graph G-prime, total capacity from S-prime equals total demand into vertices; similarly, capacity into T-prime equals total demand out of vertices, both equating to D\@.
A flow F-prime is saturating if it reaches this maximum value D, fully using edge capacities out of S-prime and into T-prime.
A feasible flow exists in the original network G if and only if G-prime has a saturating flow.
Finding a feasible flow involves transforming the problem into a max-flow problem, running max-flow on G-prime, and checking if the max-flow equals D\@.
If saturating, a feasible flow F for G can be constructed from the saturating flow F-prime in G-prime, thus solving the feasible flow problem by leveraging solutions to the max-flow problem.

\subsection*{Saturating Feasible}
Proving that a saturating flow G` guarantees a feasible flow in G\@.
By taking saturating flow F` in G`, this allows the construction of feasible flow F in G\@.
For each edge, flow in G set to the flow in G` plus its demand.
F must be valid (inflow equals outflow for internal vertices and flow on edge does not exceed its capacity) and feasible (meeting demand constraints).
F is feasible because F` non-negative ensuring F meets minimum demands.
F also respects capacity constraints since flow F on edge E equals C` of E plus demand, where C` is the capacity in G` defined as original capacity C minus demand.
Thus, F meets demand and capacity constraints--only need to verify inflow=outflow for every vertex to confirm F`s validity.

\subsection*{f is Valid}
To validate flow F in a graph, it`s necessary to prove that for each internal vertex (excluding source S and sink T), the inflow equals the outflow.
Given that F` is a valid flow from the max flow problem on constructed graph G`, inflow equals outflow at each vertex in G`.
Mapping this to G, edges from new vertex S` - in G` - to vertex V and original connections U to V remain, where edges out of S` are fully utilized as per F` saturating flow, with edge capacities equal to V`s demand.
By defining F` as F plus demand on each edge, F`s inflow at V equals F`s inflow when adjusting for edge demands, as the demands cancel out.
Similarly, algebra shows F`s outflow at V equals F`s outflow.
Hence, F`s inflow and outflow at V match, verifying F as a valid flow.
This demonstrates that a saturating flow F` can be converted to a feasible flow F by adjusting F` for edge demands.
The reverse proof remains.

\subsection*{Feasible Saturating}
Proving the reverse direction involves transforming a feasible flow f in graph G into a saturating flow f` in a modified graph G`.
This adaptation is achieved by reducing the flow in each edge of G by its demand and defining new flows in G` that match the demand at every node, specifically with edges from the new source s` and to the new sink t`.
Flows from s` to each vertex and from each vertex to t` are set to match their respective total incoming and outgoing demands.
Additionally, the edge from t to s, which has infinite capacity, is assigned a flow equal to the magnitude of flow f.
The final step involves validating that the constructed flow f` satisfies all conditions to qualify as a saturating flow in G`.

\subsection*{f is Valid}
Flow f on edge E meets demand, hence f-prime (f - demand) is non-negative.
Since f doesn`t exceed capacity, f-prime is at most capacity minus demand, aligning with G-prime network`s capacity definition.
New flow f-prime is thus non-negative and adheres to capacity restrictions.
Must verify if f-prime also fulfills the flow conservation law (inflow equals outflow).

\subsection*{f Constraints}
To ensure flow conservation at each vertex (excluding S` and T`) under flow F`, it`s checked against the original flow F, where inflow equals outflow at any vertex V\@.
F` is defined by setting the flow from S` to meet V`s demand and adjusting other inflows (U to V) by subtracting the demand on that edge.
Both adjustments negate the effect of V`s total demand, equating the total inflow under F and F`.
A similar process confirms that outflow under F` equals outflow under F at V\@.
Consequently, as inflow equals outflow for vertex V under F and these conditions hold under F`, flow F` is validated as a feasible flow.

\subsection*{Max Feasible Flow}
Lemma proved: G has a feasible flow iff G` has a saturating flow.
To find a feasible flow in G, construct G`, run max-flow, check if max-flow equals D for saturation.
If saturated, convert the flow f` in G` to a feasible flow f in G\@.
For maximum size feasible flow in G, start with feasible flow f, not zero flow as in Ford-Fulkerson or Edmonds-Karp methods.
In residual graph, augment f if possible.
Continue until no s-t path in residual.
Residual graph capacities: forward edges are capacity minus current flow, reverse edges are flow minus demand (if flow exceeds demand).
To find max feasible flow, start with feasible f, adjust residual capacities for reverse edges to flow exceeding demand, not just flow.

