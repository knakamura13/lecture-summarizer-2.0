\section*{Knapsack}

\subsection*{Lecture Outline}
SAT acknowledged as NP-complete; all NP problems reducible to SAT, thus solvable in polynomial time if SAT is.
3SAT also NP-complete, reduced from SAT\@.
Independent Set problem declared NP-complete through reduction from 3SAT\@.
Reduction notable: logic (3SAT) to graph (Independent Set) problem conversion.
NP-completeness of Clique and Vertex cover problems established via simpler reductions from Independent Set.
Aim to now prove NP-completeness of Knapsack problem, starting with proving Subset-sum problem NP-complete by reduction from 3SAT, marking a complex logic (3SAT) to optimization (Subset-sum) problem transformation.
Following Subset-sum NP-completeness, Knapsack`s NP-completeness proof deemed straightforward, set as homework.
Prior study revealed a non-polynomial time Dynamic Programming Algorithm for Knapsack due to its dependency on N and B factors, the latter exponential relative to 3SAT input size, hence not polynomial.
The clever reduction from 3SAT to Subset-sum (and Knapsack) highlights this complexity.
Next, focus on defining and proving Subset-sum NP-completeness.

\subsection*{Subset Sum}
The subset sum problem involves determining whether a subset of n positive integers sums to a specific target value t.
If such a subset exists, it is outputted; if not, the output is NO\@.
This problem serves as a useful exercise for dynamic programming techniques, which can solve it with a time complexity of O(nt), where n is the number of integers and t is the target sum.
Dynamic programming is recommended for practice with this type of algorithmic challenge.

\subsection*{Subset-Sum in P Question}
The statement posits subset-sum problem solvable in nt time, suggesting it`s in P class.
This is true; if a problem has a polynomial-time solution (here, n * t), it`s in P\@.

\subsection*{Subset-Sum in P Solution}
Running time for the problem must be N times poly(log t) to be polynomial in the input size, similar to the knapsack problem.
Current belief that it is polynomial is incorrect.

\subsection*{Subset-Sum NP-Complete}
Proved subset-sum problem NP-complete.
First, est\. it`s in NP by verifying proposed solution S in poly-time: summing elems takes O(n log t) which is polynomial.
Second, showed it`s as hard as any NP problem by reducing 3SAT problem to subset-sum.
This indicates even O(n*t) time algorithm inefficient for subset-sum, underlying its complexity.

\subsection*{3SAT Subset-Sum  Input}
Subset sum problem setup with 2n + 2m + 1 numbers: n variables (X\_1 to X\_n), 2n literals (V\_1, V\_1`, \ldots, V\_n, V\_n`), 2 numbers per m clauses, and desired sum.
Each number has n + m digits max, working in base 10 to avoid carries in sums.
Numbers are extremely large, e.g., t is \textasciitilde{}10\textasciicircum{}(n+m), hence, an n*t algorithm is inefficient.

\subsection*{3SAT Subset-Sum  Variables}
Specifying input for subset sum prob using number correlation to n literals.
Vi - Xi true, Vi` - Xi false.
Require exactly one of Vi or Vi` in subset S to reflect true/false assignment of Xi.
Ensure no inclusion of both/no Vi or Vi` for clear Xi setting.
Use base 10 to prevent digit carryover for independent digit behavior.
T, as desired sum, must have one in Ith digit, achieved only by including Vi or Vi` (not both/neither) in S\@.
This aligns subset sum solution to a definitive assignment of literals, satisfaction aside.

\subsection*{3SAT Subset-Sum  Example}
Reduction of 3SAT problem to subset-sum problem using a formula with three variables (X1, X2, X3) and four clauses.
Each variable is assigned a pair of numbers (V1/V1`, V2/V2`, V3/V3`) and each clause a pair (S1/S1`, up to S4/S4`), totaling 15 numbers to define subset-sum input.
Numbers are seven digits, representing variables and clauses.
Digits 1--3 correspond to variables, 4--7 to clauses.
To ensure subset-sum solution reflects a variable`s Boolean value (true/false), digit one is `1` for V1/V1` and t, and similarly for others.
This forces the solution to include exactly one from V1/V1`, ensuring a sum of `1` and mapping correctly to variable assignments.
The structure guarantees that any subset-sum solution corresponds to a potential satisfying assignment by reflecting the truth assignment of variables within the subset.

\subsection*{3SAT Subset-Sum  Clauses}
Encoding 3SAT clauses into subset-sum problem using binary numbers where first n digits rep clauses, following m digits for vars.
Literal Xi in Cj leads to 1 in n+j position of number Vi.
Xi bar gives 1 in n+j position of Vi prime.
Numbers Vi, Vi prime track clauses.
In example: X1 bar, X2 bar, X3 in first clause, correlate with 1s in V1 prime, V2 prime, V3 prime.
For clause satisfiability, need at least one of V1 prime, V2 prime, V3 prime present - place a 3 in corresponding position in target sum t.
Buffer numbers S1, S1 prime, etc., used to achieve necessary sum if only one literal in clause is true.
Full satisfiability means at least one literal in each clause must be true; can augment with buffers to match sum t.
Example: V1 is 1000011, V1 prime is 1001100; t is 1113333.
Reduction from 3SAT to subset-sum formulated.

\subsection*{3SAT Subset-Sum  Buffers}
For clause satisfaction encoding, each literal is represented by V\_i or V\_i`.
A literal`s presence in a clause is noted by a 3 in the nth+j digit of a total (t).
S\_j and S\_j` serve as buffers, marked with a 1 in the same digit.
Other numbers get a 0 in this digit, ensuring a sum of 3 if one, two, or all three literals are true.
If no literals are true, the sum cannot reach 3, indicating unsatisfied clause.
This process allows for the verification of clause satisfaction within a larger computational framework.

\subsection*{3SAT Subset-Sum  Correctness}
Constructed subset sum instance solution proves 3SAT problem satisfiability, confirming value reduction.
Solution S sums to target T\@.
N initial digits match N 3SAT variables X1-Xn.
One must choose Vi or Vi` (representing true or false) for each i to achieve T`s required one per digit, as choosing both sums to two, none to zero.
This subset sum solution translates to a true/false variable assignment, which must be further examined to confirm it satisfies the original 3SAT problem.

\subsection*{Proof  Satisfying Assignment}
Solving specific subset sum instances equates to satisfying assignments for Boolean variables \textbackslash{}(X\_1\textbackslash{}) to \textbackslash{}(X\_n\textbackslash{}) in an input formula \textbackslash{}(f\textbackslash{}).
The first n digits in these instances indicate true/false for \textbackslash{}(X\_1\textbackslash{}) to \textbackslash{}(X\_n\textbackslash{}), and subsequent digits are tied to clauses \textbackslash{}(C\_j\textbackslash{}).
A digit 3 in position \textbackslash{}(n+j\textbackslash{}) indicates clause \textbackslash{}(C\_j\textbackslash{}) must be satisfied by including at least one corresponding number from literals in \textbackslash{}(C\_j\textbackslash{}).
Various scenarios: satisfying one literal uses two buffer numbers (\textbackslash{}(S\_j\textbackslash{}) and \textbackslash{}(S\_j`\textbackslash{})) to sum to 3, two literals require one buffer number, and three satisfied literals need no buffer, automatically summing to 3.
No satisfied literals mean no sum of 3, thus no solution.
Hence, a solution to the subset sums assures satisfaction of all clauses, providing a satisfying \textbackslash{}(f\textbackslash{}) assignment.
Moreover, satisfying a 3SAT formula guarantees a corresponding solution to the subset sum, confirming a bidirectional implication between the two problem instances.

\subsection*{Reverse Implication}
Prove reverse imp.: satisfying 3SAT assigns -\textgreater{} solution to subset-sum.
Given satisfying assigns\. for f, construct solution S: if X\_ith true, add V\_ith to S; if X\_ith false, add V\_ith prime to S\@.
Only one of V\_ith or V\_ith prime added to S, ensuring correct ith digit in T\@.
For jth clause (since at least one lit\. satisfied), corresponding nums\. ensure sum of 1, 2, or 3 in digit n+j of T, which is 3.
Can use S\_j and/or S\_j prime to reach sum of 3.
Thus, all digits of T correct, providing solution to subset-sum.
Reduction proven correct.

\subsection*{Knapsack is NP-complete}
Proved subset sum problem (SSP) NP-complete.
Now show knapsack problem (KP) NP-complete.
Correct NP version of KP needed.
For NP-completeness: 1) prove KP in NP, 2) reduce known NP-complete problem to KP\@.
Use SSP as it`s similar to KP for reduction.
It`s about direction: take SSP input, transform to KP input, then transform KP solution back to SSP solution.
Avoid common mistake of wrong direction in reduction.
Essential: properly reduce from known NP-complete problem to target problem.

