\section*{Edmonds Karp Algorithm}

\subsection*{Max-Flow Min-Cut Algorithms}
Ford-Fulkerson and Edmonds-Karp algorithms both address the max-flow problem.
Ford-Fulkerson uses augmenting paths in the residual graph via DFS or BFS, with a runtime of O(M*C), assuming integral capacities.
Edmonds-Karp is a variant of Ford-Fulkerson, employing BFS for finding paths.
Its runtime outperforms Ford-Fulkerson at O(m\textasciicircum{}2*n) and does not insist on integral capacities, only positivity.
This session will recap Ford-Fulkerson and highlight Edmonds-Karp`s distinctions before exploring the latter`s runtime analysis in depth.

\subsection*{Ford Fulkerson Algorithm}
Ford-Fulkerson algorithm operates on a directed graph with integer edge capacities, denoted C\_sub\_e, to find a maximum flow from source, s, to sink, t.
Initially, flows are set to zero.
Algorithm constructs a residual network, G\_superscript\_f, then searches for an augmenting path using BFS or DFS from s to t.
If no path exists, current flow is max size; if path exists, find minimum capacity, c(P), along path and augment flow by c(P).
This process repeats, updating residual network and searching for paths until no augmenting path can be found, indicating termination of the algorithm and an optimal solution.
Focus now shifts to Edmonds-Karp algorithm.

\subsection*{Edmonds-Karp Algorithm}
The Edmonds-Karp algorithm differentiates from the Ford-Fulkerson by exclusively using BFS to search for st-paths in the residual network.
The algorithm`s efficiency doesn`t rely on integer capacities; any positive value is sufficient.
An essential attribute of both Edmonds-Karp and Ford-Fulkerson is that during each iteration, the residual network must alter by at least one edge reaching full capacity and thus being removed in the subsequent step, whether it`s a forward or backward edge.
Additional edges may also be removed or added, and understanding these changes is important.

\subsection*{Proof Outline}
Edmonds-Karp algorithm`s run time proven to be O(m\textasciicircum{}2n).
Number of algorithm`s augmenting path rounds \textless{}= m*n.
Each round involves BFS, which is O(m) time given m \textgreater{}= n.
Total run time O(m\textasciicircum{}2n) due to O(m) per round and O(mn) rounds.
Focus: Prove max rounds = mn.
Each round, residual graph`s edge, min capacity on augmenting path, deleted.
Edge can be re-added later.
Key lemma: An edge e deleted and re-added to residual graph \textless{}= n/2 times.
As m edges exist, each potentially deleted n/2 times, total rounds \textless{}= n/2 * m, confirming the algorithm`s run time.
Proof details follow to support lemma.

\subsection*{BFS Properties}
Key lemma states an edge in the residual graph is deleted and reinserted \textless{}= n/2 times.
To prove it, we use BFS properties, which operates on directed/undirected graphs without considering edge weights.
BFS starts at vertex s and outputs an array of size n, where dist(v) is the min number of edges from s to v, ignoring edge weights.
For weighted edges, Dijkstra`s algorithm is needed.
BFS also determines the shortest path from s to v using the fewest edges.

\subsection*{BFS Example}
BFS (Breadth-First Search) algorithm analysis when applied to flow network at initial flow (zero flow along edges) shows that residual network corresponds to the original network.
Running BFS begins at source vertex s, exploring adjacent vertices a, b, and c in sequence, then their subsequent connections which include d, e, and f.
A path is formed (s-a-d-t) characterized by its edges connecting vertices whose levels increment by 1, signifying BFS finds the shortest path in terms of edge count from source to target (t).
The path`s progression observes vertices at levels that correlate to their minimum distance from source s, incrementally rising by 1 per edge; no level skips or regressions being possible due to BFS`s layer-by-layer exploration, ensuring the shortest path is found.
This level increment consistency is a vital BFS property underlining the reliability of the path established.

\subsection*{BFS Properties - Part 2}
In the Edmonds-Karp algorithm, vertex levels in the residual network are examined, focusing on vertex Z\@.
As the algorithm progresses and the residual network evolves with edge deletions and additions, it is shown that vertex levels can only stay the same or increase, never decrease.
For example, if an edge from A to D is removed, the level of D increases because the shortest path from source S to D gets longer.
Conversely, adding edges, such as from B to D, might intuitively seem to reduce D`s level, but that`s not the case.
The proof lies in understanding the modifications within the residual network, particularly which edges get added or removed, supporting the claim about non-decreasing levels of vertices.

\subsection*{AddDelete Edges}
A residual graph`s edges are adjusted during flow network operations based on whether they are in the network or are back edges.
A forward edge is added when its flow is lessened from full capacity, implying a back edge was used in an augmenting path; it`s removed when its flow meets capacity due to augmentation on that edge.
Conversely, a back edge is added when flow on the forward edge starts from zero, indicative of the forward edge`s use in augmentation; it is removed when the forward edge`s flow is depleted by sending flow back, meaning the back edge was on the augmenting path.
Thus, edge adjustments in residual networks are directly linked to their roles in augmenting paths that alter flow levels.

\subsection*{Conclusion}
Adding edge y to z in a residual network implies reverse edge z to y is on the augment path.
Removing edge y to z indicates edge y to z is on the augment path.
Edge removal equals capacity elimination, necessitating increased flow via this edge; it must be on augment path for flow increase.
Adding an edge equals spare capacity rise; flow reduction on the edge requires augmentation of the reverse direction edge, placing it on the augment path.

\subsection*{Proof of Claim}
Vertex levels in the Edmonds-Karp algorithm remain stable or increase, never decrease.
Levels might seem to drop if a shorter path appears, but this is discounted by edge removals; focus is on edge additions.
Analyzing a Y-to-Z edge addition, the reverse must be part of the augmenting path (by BFS properties, every subsequent vertex level is one higher than the previous).
Given Z`s level is i, Y`s is i+1, making the Y-to-Z addition from higher to lower level, not reducing Z`s level.
Hence, vertex levels remain constant or rise with each algorithm round, refuting the possibility of level decrease.
The proof underscores this stability while hinting at a bound on algorithm rounds derived from occasional strict level increases.

\subsection*{DeleteAdd Effect}
The lemma addresses how the levels of vertices change when an edge is deleted and then added back into a residual network, specifically regarding vertex v and its neighbor w.
If an edge from v to w is deleted, it implies that w`s level is one more than v`s level at the time of deletion.
When the edge is reintroduced, v`s level must then be one more than w`s.
Since v`s level doesn`t decrease and w`s level was at least i+1 earlier, reintroducing the edge ensures v`s level is at least i+2.
In essence, deleting and re-adding an edge to the network causes the associated vertex level to increase by at least two.

\subsection*{Finishing Off}
Edge deletion and reinsertion in a residual network leads to at least a two-level increase for the originating vertex.
Vertex levels range from 0 for source S to n.
Thus, an edge can be deleted and readded to the residual network a maximum of n/2 times.
With m edges in the graph, this establishes a limit of nm rounds for the Edmonds-Karp algorithm, concluding its analysis.

