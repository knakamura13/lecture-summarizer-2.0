\section{RA2: RSA}

\subsection{Fermats Little Theorem}
Before we dive into the RSA cryptosystem, we`re going to dive into the mathematical concept which is the basis for the whole cryptosystem.
The key concept is Fermat`s little theorem.
Take a prime number P, then for every number Z, which is at most, P minus one and least one.
If we look at Z raised to the power P minus one mod p, what is that going to equal? That`s going to be congruent to one.
So for any Z, which is relatively prime to P, so this statement here, can be replaced by any Z, which is relatively prime to P\@.
So the gcd of Z and P is one.
For any such Z, if P is prime, if we take Z, raiseed to the power P minus one is going to be congruent to one mod p.
This easy to state fact is going to be the basis of the RSA cryptosystem.
It`s also going to be the basis of a primality testing algorithm.
So, if we`re given a number x, and we want to test whether x is prime or not, we`re going to devise an algorithm based on Fermat`s little theorem.
Now, we`re going to look at the proof of Fermat`s little theorem.
It`s not too hard, it`s not too easy, it`s reasonable proof, but it`s a very beautiful proof, very elegant.
And there are several important ideas in the proof.
For example, how does it come up that P is prime? How does that come into the proof? Just a little bit of foreshadowing, if P is prime, what does that tell you about Z? Well, these- Z`s are all relatively prime to P\@.
So, what does that imply? What that implies that these Zs all have an inverse mod P\@.
That`s the key thing about a prime number P and that`s going to come out in the proof.
One last historical note.
This is Fermat`s little theorem, it`s not Fermat`s last theorem.
Fermat`s last theorem is the one which he claimed to have a proof and he couldn`t fit it into the margin and which was proofed roughly 400 years later by Andrew Wiles, after 10 years of hard work.
So, this Fermat`s little theorem is quite easy to prove and we`re going to prove it now.
We`ll go through the proof of Fermat`s last theorem in the homework.
Don`t worry, we`ll give you a few hints.

\subsection{Fermats Thm.  Proof}
So, here`s Fermet`s little theorem again.
For prime number p, for every z between one and p minus one, if we look at z raised to the power of p minus one and take in mod p we end up with one.
So well let`s look at the proof of this theorem.
Let me look at the set of possible z`s.
What are these? These are all the numbers between one and p minus one.
One, two, three up to p minus one.
Let`s define the set s as the possible z`s.
So, these are the numbers one, two, three up to p minus one.
Which is defined in this set s.
Now let`s look at another set s prime.
S Prime is going to be defined by taking the elements of s, each of these p minus one elements, multiplying each one by z.
So we got z times s and then for each of those elements we take it mod p.
So the first element is going to be one time z mod p.
The second element is going to be two times z mod p.
The last element is going to be p minus one times z mod p.
Let`s look at a simple example.
Let`s take p as seven and let`s take z as four.
First off, what`s the set S? S there are the numbers one, two, three, four, five, six.
What`s S prime? Well, the first element is one times four which is four mod seven, which is going to be four.
The second element is going to be two times four which is eight mod seven which is one.
The third element is going to be three times four which is 12 mod seven which is five and four times four is 16 mod seven is two.
Five times four is 20 mod seven is six.
Finally six times four is 24 mod seven is three.
Look at this set.
It`s the same as s except in different order.
So the set of elements are the same just it`s a permutation of s.
So the key thing is that s and s prime are the same sets, just in different order.
Let me prove to you this fact, that in general s and s prime are the same sets and then we`re going to use this fact that s and s prime are the same sets to prove the Fermet`s little Theorem.

\subsection{Proof  Key Lemma}
Let me prove to you now that this fact that S and S prime are the same sets, just in different order.
Recall S as a set one, two, three, up to P minus one, where P is a prime number from the hypothesis of the theorem statement.
And the set S prime is the following, first element is Z mod P, the second element is 2Z mod P, and so on.
The last element is P minus one times Z mod P\@.
Now let`s go ahead and prove that S and S prime are the same.
We`re going to show that the elements of S prime are distinct and non-zero.
What are the possible values for the elements of S prime? We take them all mods P, right? So they`re all between zero and P minus one.
Okay.
So those are possible values, zero up to P minus one.
Now if we show they`re all non-zero then the possible values are one, two, up to P minus one.
So, this P minus one possible values.
How many elements are there in S prime? Well there`s P minus one elements in there.
So, since the size of S prime is P minus one, and the possible values are P minus one possible values.
So if they`re all distinct then they must contain exactly each of these elements one time and, therefore, it`s the same.
S prime is the same as S, just in possibly different order.
So, once we show that the elements of S prime are distinct and non-zero, we`re done.
Let`s start off by showing that they`re distinct.
We`ll do it by contradiction.
So let`s suppose they`re not distinct.
So let`s suppose for some I not equal to J\@.
The Ith element and the Jth element of S prime are the same.
The Ith element of S prime is I times Z mod P and the Jth element of S prime is J times Z mod P\@.
So we`re supposing that the Ith element and the Jth element are the same, so that IZ is congruent to JZ mod P\@.
Now let`s use the fact that P is a prime.
If P is a prime, what do we know? We know then that every element in S has an inverse mod P\@.
So, Z is an element in S, right? Z is a number between one and P minus one.
So to recap, P is a prime so that implies that Z has an inverse mod P\@.
So, let`s use that inverse.
What can we do? Look at this equation IZ is congruent to JZ mod P\@.
Let`s multiply both sides by Z inverse.
It exists.
I don`t know what it is, but no matter what it is multiply both sides by Z inverse, okay? So what do we get? So, I multiplied the left hand side by Z inverse and the right hand side by Z inverse.
What do I get? Now I have Z times the inverse.
What is that? That`s one.
Z times the inverse, that`s one.
What am I left with? I is congruent to J mod P\@.
What does that mean? That means they`re the same when I look between one and P minus one.
So, therefore, they`re the same index.
I suppose that they were different indexes, which gave me the same value but in order to come up with the same value, they have to be the same index.
So I get my contradiction.
Therefore, there is no repeat elements in S prime and they`re all distinct.
Now, we`ve shown that the elements of S prime are distinct.
Now let`s show that the elements of S prime are non-zero.
Again, let`s do a proof by contradiction.
Suppose one of the elements is zero.
So take the Ith element of S prime and suppose it`s zero.
So that means I time Z is congruent to zero mod p.
What do we know? We know the inverse of Z exists mod P, right? So let`s multiply both sides by Z inverse again, and what are we left with? We`re left with I is congruent zero mode p, which means that I is not one of the indices here, the indices here are between one and P minus one.
So this is not an element of S prime.
So we get a contradiction, therefore the elements of S prime are non-zero and they`re distinct and therefore, they contain the elements one through P minus one just in possibly different order and therefore, S and S prime are the same.

\subsection{Proof  Finishing Up}
So let`s recap where we are.
Our goal is to show that Z raised to the power P minus one is congruent to 1 mod P\@.
What have we done so far? We`ve looked at these two sets S, which are the possible values for Z\@.
One, two, up to P minus one, and we have this set S Prime, which we obtained by taking the elements of S, multiplying each one by Z, and then taking at mod P\@.
So, we have this set of elements, and we`ve shown, that S and S Prime are the same.
Just in possibly different order.
Now, we`re going to use that fact that S and as Prime are the same, in order to prove this theorem statement.
So, what are we going to do.? We`re going to multiply the elements of S together, and we`re going to multiply the elements of S Prime together.
The elements are the same, just in possibly different order.
So, when we multiply the elements of S together, we should get the same product, as when we multiply the elements of S Prime together.
So, when we multiply the elements of S together, what do we get? We get 1 times 2 times 3 up to P minus 1.
When we multiply the elements of S Prime together, what do we get? We get one time Z, then we get two times Z, then we get three times Z, and so on, the last term is P minus one times Z\@.
And these are the same, modulo P\@.
So, on the left hand side, we`ve got the elements of S multiply together.
On the right hand side, we`ve got the elements of S Prime multiplied together.
So, we`ve got the same numbers, just in different order, and we`re taking it mod P, because these sets are the same with respect to mod P\@.
What do we got on the left hand side? Let`s just simplify it a little bit.
This is P minus 1 factorial right? What do we got in the right hand side? Well, we got Z, Z, Z\@.
How many Z`s do we got? We got P minus one Z`s.
Now that looks good.
Now we`re starting to look like the theorem statement.
And what else do we got? We got one times two times three, up to P minus one.
So, we got another P minus one factorial.
And these are the same mod P\@.
Okay? Can we get rid of this P minus one factorial? In normal real arithmetic, we can just cancel it out.
We cancel out here.
Well, does the inverse exist? Notice P is prime.
So, that means that there is an inverse for one, two, up to P minus one.
Each of these numbers has an inverse mod P\@.
Each of these P minus one numbers has an inverse mod P, because P is a prime.
So, they`re all relatively prime to P\@.
So, we can multiply both sides by these inverses.
And what happens? We get rid of this one times two times three, up to P minus one from both sides.
So that cancels out the P minus 1 factorial.
And then, what are we left with? We`re left with, one is congruent to Z to the P minus one mod P\@.
Which is the theorem statement we`re trying to prove.
So we`re done.

\subsection{Eulers Theorem}
Now I misled you a little bit.
We`re going to use a generalization of Fermat`s little theorem called Euler`s theorem.
So from this little theory as Gann says, if P is a prime then for every Z between one and P minus one, we take that Z raised to the power P minus one is congruent to 1 mod P\@.
Euler`s theorem is going to generalize it to any number N here instead of just a prime number P\@.
So Euler`s theorem replaces P by any N, not necessarily a prime number, and any Z but we need that Z and N are relatively prime to each other.
By relatively prime, we mean that the GCD of Z and N is one.
Now if Z and N are relatively prime to each other then we`re going to perform an operation similar to Fermat`s a little theorem.
So we`re going to raise Z to the power phi of N, I`ll tell you what phi of N is in a second, and that`s going to be congruent to one.
Now what is this crazy phi of N? When N is a prime then phi of N is P minus one or N minus one.
In general, what does it phi of N? It`s a number of integers between one and N which are relatively prime to N\@.
So in other words, it`s the size of the set of those Xs where X is between one and N and X is relatively prime to N, which means that the GCD of X and N is one, and we want to look at the cardinality of this set.
How many Xs between one and N, or N minus one, have GCD of one with N? This function phi of N is called Euler`s totient function.
Try to say that ten times quickly.
So let`s take a look at what Euler`s totient function is for the case when N equals P is a prime number.
For prime number P, what is phi of P? How many numbers between one and P are relatively prime to that prime number P? Well every number between one and P minus one, one, two, three, up to P minus one are all prime, relatively prime to P\@.
So the number of relatively prime to P is P minus one.
So if you look at Euler`s theorem for the case when N is a prime number P then we`re raising Z to the power of P minus one, and that`s congruent to one mod P\@.
So it`s exactly the same as Fermat`s little theorem.
So Euler`s theorem is a generalization of Fermat`s little theorem to arbitrary N\@.

\subsection{Eulers Totient Quiz Question}
Now here`s Euler`s theorem again.
Let`s look at this Euler`s totient function for the case that we`re going to use.
We`re going to have a pair of prime numbers P and Q\@.
So we saw when N equals one prime number P, then Euler`s theorem is the same as Fermat`s little theorem.
We`re going to apply Euler`s theorem for the case when capital N is the product of P and Q, where P and Q are both prime numbers.
What is phi of N in this case? Why not go ahead and try to figure out what is phi of N for the case when N is P times Q for prime numbers P and Q\@.
So we want to figure out how many numbers between 1 and P times Q are relatively prime to PQ\@.

\subsection{Eulers Totient Quiz Solution}
Now, let`s go ahead and look at what phi\_of\_n is for this case when n\_equals\_p\_times\_q.
Let`s write out all the numbers that we`re looking at.
We`re looking at the numbers between one and p\_times\_q.
So, this pq numbers here.
How many of these numbers are multiples of p? Well, p, two\_times\_p, up to q\_times\_p, are all multiples of p.
How many of those are there? There`s q of them.
So, q are multiples of p.
How many are multiples of q? There`s q, two\_times\_q, up to p\_times\_q.
So, p of these numbers are multiples of q.
So, each of these numbers has a common divisor with p\_times\_q, namely p.
And each of these numbers has a common divisor with p\_times\_q, namely q.
So, we have to exclude all these numbers.
So we had p\_times\_q numbers originally.
Q of them are multiples of p, and p of them are multiples of q.
And look, we double counted.
P times q is counted twice, so let`s add one while I can.
This can be rewritten as p\_minus\_one\_times\_q\_minus\_one.
So, what is phi\_of\_n for the case when n is p\_times\_q for prime numbers p and q? Is p\_minus\_one\_times\_q.

\subsection{Eulers Thm\. for N=pq}
What is Euler`s Theorem tell us for the case when N equal P times Q, where P and Q are prime numbers? For any Z which is relatively prime to N, which means that the GCD of Z and N is one, and if we take Z raised to the power phi of N, which is P minus one, Q minus one, we get back one\_mod\_N, which is P\_times\_Q\@.
This fact here, is going to be the basis for the RSA algorithm.
We`re going to generate a pair of prime numbers, P and Q\@.
And we`re going to look at P\_times\_Q, and we`re going to have a message, and we`re going to use this P\_minus\_one, Q\_minus\_one, to generate a encryption key and the decryption key, and this is going to give us a basic fact, which is going to allow the encryption and decryption.
Let`s investigate this a little bit more and then will see the basic idea of the RSA algorithm.

\subsection{RSA Alg.
Idea}
Now we can get to the basic idea behind the RSA algorithm.
Let`s first look at from Fermat`s Theorem.
It talks about a prime number P\@.
Let`s look at a pair of numbers, B and C, which are relatively prime to each other mod p minus one.
Why is P minus one? Because remember Euler`s totient function or Fermat`s Little Theorem, the exponent there is P minus one.
Okay? That`s where that P minus one comes from.
So we`re taking a pair of numbers B and C which are inverses to each other modulo P minus one.
So what does that mean? That means B times C is one mod P minus one, or in other words B times C is one plus some multiple of P minus one.
So K is a sum integer.
So that means when we take B times C and we divide it by P minus one then we get a remainder of 1.
Let`s take a number Z between one and P minus one, and let`s raise it to the power B times C, and let`s take the whole thing mod P\@.
Now we can replace B times C by this formula.
So we get to Z then we get Z to the K times P minus one.
We can do the P minus one.
We can raise the whole thing to the Kth power.
Now what do we get? Well the Z sticks around.
What about the Z to the P minus one? What do we know? Fermat`s Little theorem tells us that Z to the P minus one is congruent to one mod P\@.
So we can replace this by 1 and then one to the K is what? It`s one.
So this whole thing is one.
So it drops out and we`re left with Z\@.
So Z raised to the power B times C is Z mod P\@.
So we get back Z itself.
Okay? We`re starting to get the idea for the encryption and decryption algorithm.
I`m going to take a message Z, and now, if I raise it to the power B to encrypt it and then later I raise it to the power C to decrypt it, that`s the same as taking Z to the power B times C and that`s going to end up with Z itself.
So if I encrypt by raising to power B and then decrypt by raising it to the power C, I end up with the original message itself.
Now, the problem is that I have to tell everybody P, so that they can do these operations.
If I tell them P then they know P minus one, and therefore, given B then they can figure out C, which is the inverse of B with respect to mod P minus 1.
So I want to conceal this fact.
I want to conceal this P minus one term.
How am I going to do that? Well I`m going to use Euler`s theorem instead of Fermat`s little theorem.
For Euler`s theorem, I want to have a pair of primes, P and Q\@.
I`m going to look at their product N\@.
Now, what am I going to do? I`m going to take a pair of numbers, D and E, which are inverses of each other with respect to P minus one times Q minus one.
So, D times E is congruent to one mod P minus one, Q minus one.
Why did they use P minus one and Q minus one instead of P minus one? Well, in Fermat`s little theorem when I raise Z to the P minus one I get one mod P\@.
What happens in Euler`s theorem? Well, the totient function of N we saw was P minus one, Q minus one for this case when it`s P times Q\@.
So if I take a number Z and I raise it to this P minus one Q minus one, I`m going to get one, when I look at it, mod N\@.
So, that`s what I`m going to do.
I`m going to take a number Z\@.
I`m going to raise it to the power D times E\@.
What do I get? D times E is congruent to one mod P minus one Q minus one.
That means D times E is equal to one plus some multiple of P minus one times Q minus one so I get the one and then I get some multiple of P minus one times Q minus one.
Now what is this, Z raised to the power P minus one, Q minus one? I`m looking at this whole thing mod N\@.
Euler`s theorem tells me that if Z is relatively prime to N then this thing is one.
So then one raised to the power of K, just like before, is one.
So this term goes away, I`m left with Z, and there it is.
Notice the terms D and E decryption, encryption.
So, you give me a message Z\@.
I`m going to tell you E and I`m going to tell you N, what are you going to do? If you want to send me the message Z, you`re going to take Z and you`re going to raise it to power E and then get mod N, and then you`re going to send it to me.
That`s the encrypted message.
What do I do? I compute this D, which is the inverse of E mod P minus one, Q minus one.
That`s my decryption key.
So I take that encrypted message with the Z to the E mod N\@.
I raise it to the power D\@.
And what am I left with? I`m left with Z itself, the original message.
Now, the point is you know N, which is P times Q\@.
I know P and Q, but you don`t know them.
You just know the product of them, you know N\@.
So from N, can you figure out P minus one times Q minus one? I don`t know how to do that unless you tell me P and Q\@.
If you just tell me their product P times Q, I don`t know how to get P minus one times Q minus out of P times Q\@.
But I know P and Q, so I can take P and Q and compute P minus one times Q minus one, and then I can run extended Euclid algorithm to compute the inverse of E mod P minus one, Q minus one.
So I can compute D but nobody else knows how to compute D because they don`t know P minus one, Q minus one.
They simply know N, P times Q\@.
So, everybody`s going to know N, and everybody is going to know E, but I alone know D and that`s the decryption key.
So I tell everybody E and N, we take the message, raise it to the encrypted power, take it mod N, that`s the encrypted message.
Send that.
The whole world can see that, but only I can decrypt it.
So now let`s go ahead and detail the RSA algorithm.

\subsection{Crypto Setting}
Here`s the general Crypto setting.
I got Alice over here and we got Bob over here.
Alice has a message M, that she wants to send to Bob.
Now, they have a communication line that they can talk over.
The problem is, that there`s someone eavesdropping on this conversation.
So, whatever transmitted over this communication line can be seen in the clear by Eve, who`s the eavesdropper on the communication.
So, Eve sees everything in the clear that`s communicated on this line.
So, Alice doesn`t want to send the message by itself.
She wants to encrypt a message.
So, Alice is going to take the message M, she is going to feed that into the encryption scheme and out is going to pop E(M), the encrypted form of M, the message.
She`s going to send that along the transmission line.
Bob, meanwhile is going to have his decryption box.
So, he`s going to take this encrypted message, feed that as input into his decryption box, an arrow is going to pop the decrypted message.
Meanwhile, Eve what did she see? She sees this encrypted message, because that`s what sent over this communication line.
And, from the encrypted message, she doesn`t know how to decrypt.
Only, Bob knows how to decrypt.
Now, this is a public key cryptosystem.
By public key, we mean that no communication needs to be happen between Alice and Bob in private.
So, they don`t need to have any coordination beforehand.
What`s going to happen is, Bob is going to do a computation on his own and he`s going to compute a public key and a private key.
His public key is going to be two numbers, N capital N and and E\@.
Capital N it`s going to correspond to P times Q, the product of two primes and E is going to be some number which is relatively prime to P minus one times Q minus 1.
He`s going to announce to the world, N and E\@.
Now, anybody that wants to send a message to Bob is going to take his public key.
They`re going to take their message and using M, they`re going to encrypt their message using N and E, the public key.
But, the thing is only Bob knows how to decrypt messages that are encrypted using his public key.
So, he`s going to have a private key D, what is D? D is going to be the inverse of E mod P minus one times, Q minus one.
Only he knows P and Q, so only he can compute D\@.
He can compute D and then he can take this encrypted message and he can raises the power D and then take it mod N and that will give him the original message.
But, only Bob knows how to compute this private key, so only he can do this decryption.
But, there`s no coordination that needs to happen between Alice and Bob.
Bob does this computation on his own.
He computes N and E and announces it to the world and he computes this private D and keeps it to himself.
Meanwhile, anybody that wants to send a message to Bob looks up his public key, sees his N and E, he uses that to encrypt, the sender encrypted message in the clear.
So, anybody can see the encrypted message and then, only Bob can decrypt it.
So, let`s detail this scheme, this is a RSA cryptosystem.

\subsection{RSA Protocol  Keys}
Let`s detail the RSA protocol.
Let`s start with the receiver, Bob.
So what is Bob going to do? Bob has to compute his public key and his private key and he has to announce to the world his public key.
How does he do it? The first step that Bob is going to do, is he`s going to choose two N-bit numbers P and Q which are primes.
And he`s going to do that at random.
So he`s going to choose two random prime numbers P and Q\@.
How does he do that? We haven`t discussed that at all.
And we`re going to skip it until after we see the whole protocol.
The idea is that, first he`s going to generate two random N-bit numbers P and Q\@.
And then he`s going to check whether those random N-bit numbers are prime or not.
So we`re gonna have a primarily testing algorithm, so we can efficiently test whether a number is prime or not.
So he`s going to generate random numbers, test whether they are prime, if the are prime, then they are random prime, if they`re not prime, he`s going to repeat and generate a new random number and check whether its prime again and keep going until he gets a prime number.
How do you generate a random number? You just try and generate a random string of zeros and ones of length and that gives you an N-bit random number.
How do you check whether it`s prime or not? Well, that`s a little bit more complicated.
And it turns out that a random number has a reasonable chance of being prime.
Primes or dense in some sense.
We`ll see details of all that next, after we go through the whole RSA protocol in detail.
The second step, Bob chooses an E which is relatively prime to P minus 1 times Q minus 1.
How does he do that? He`s going to try E equals 3 and he`s going to check whether the GCD of three and P minus 1, Q minus 1 is 1.
If it is, then they are relatively prime to each other.
How does he check it? He runs Euclid`s GCD algorithm.
And if three is not relatively prime, then what does he do? He tries five and then seven and then 11 and so on.
You know, if you get up to 13 or 17 or 19 and none of those are relatively prime to P minus 1, Q minus 1, what do you do? Usually, you go back to the first step, choose two new primes P and Q and try again.
It`s nice to keep this encryption key small.
It makes it easy for somebody to encrypt their message, to send to you.
Now what do you do? Let`s let N equal P times Q\@.
Now Bob can publish his public key, N and E\@.
N is the product of P times Q\@.
So he`s publishing that product.
He`s not publishing P or Q, he`s just telling what the product is and he`s telling them what E is, where E is relatively prime to P minus 1, Q minus 1.
So the whole world can know N and E\@.
Meanwhile, Bob computes his private key.
What`s his private key? It`s the inverse of E relative to P minus 1, Q minus 1.
So D is the inverse of E mod P minus 1, Q minus 1.
How do we know that D exists? Because E is relatively prime to P minus 1, Q minus 1.
We chose it so that it was relatively prime therefore it has an inverse mod P minus 1, Q minus 1.
And we can find that inverse, how? By using the extended Euclid algorithm, we can find the inverse.
This is going to be Bob`s decryption key.
He keeps it private, he doesn`t tell anybody.
He tells the whole world N and E, but he keeps private his decryption key, D\@.

\subsection{RSA Protocol  Encrypting}
Now, let`s look at things from Alice`s perspective.
Alice has a message M that she wants to send to Bob.
What`s the first step that Alice does? She looks up Bob`s public key which is this pair N, E\@.
Now, she needs to encrypt her message using Bob`s public key.
What does she do? She encrypts using their public key.
She takes the message M raises its power E and takes that mod N\@.
And that`s her encrypted message, Y that she sends to the world.
Now, one key thing is, E might be a little bit large.
So, how does she raise M to the power E Mod N? She uses our fast modular exponentiation algorithm that we just saw earlier in this lecture.
Finally, Alice can send the message Y\@.
Now, let`s look at the final step of the procedure.
What happens for Bob? Bob receives this encrypted message Y that Alice sent.
Now, Bob decrypts this message.
How does he decrypt it? He computes Y, this encrypted message, raises it to the power D which is his private key and he takes that mod.
N\@.
What is that going to equal? That`s going to give him back M\@.
So, he`s going to end up with the original message M\@.
Let me recall why that`s the case.
Remember, how did we choose D? D is the inverse of e mod P minus one Q minus one.
That means D times E is congruent to one mod P minus 1 Q minus one.
That means D times Z is one plus some multiple of P minus one times Q minus one.
Now, what are we doing? We`re starting with the message M\@.
Alice is encrypting it by taking that message M raising it to the power E then taking it mod N\@.
Call N is P times Q where P and Q are prime numbers.
This M to the E is Y\@.
Now, what does Bob do? Bob takes this as message Y and he decrypts it by raising it to the power D\@.
What do we get then? We get M to the power E times D\@.
What do we know about E times D? That`s equal to one plus some multiple of P minus one Q minus one.
We get M for this one and then we get this multiple of P minus one Q minus one.
What is M raised to the power P minus one Q minus one? Well, when M is relatively prime to N, then Euler`s theorem tells us that this is one.
So, this whole term drops out and what are we left with? We`re left with M, the original message.
So, we take this message M raise it to power E and then raise it to the power D, what do we end up with? M the original message.
This is the case when M is relatively prime to N\@.
And also it holds when M and N have a common factor namely, P or Q\@.
In which case you can prove this statement still holds.
But it takes a little bit more work.
You got to use Chinese Remainder Theorem.
But that`s the basic idea.
We use Euler`s theorem.
So, that gives us this P minus one Q minus one term.
That`s the whole RSA algorithm.
It`s fairly simple.
The only thing that`s missing now for us, is how do we generate a random prime number? How do we get this P and this Q? We said, generate a random number and then check whether it`s prime.
So, we`ve got to look at how to check whether a number P is prime or not.
We`ll do that next.
But let`s first look at some simple issues that might arise in the RSA algorithm that you have to be careful about.
One important note before we move on.
How do we compute this Y to the D mod N? Notice we, typically tried to make E small.
Why did we make E small? So, that it`s easy, it`s fast to compute the message raised to the power E mod N\@.
So, it`s easy and it is fast to encrypt the message.
So, we want to make it easy for somebody to send us an encrypted message.
But then we`re going to put in extra work in order to decrypt it.
Why? Because this D, if E is smaller then D is probably going to be a huge number.
So, how do we take this mass encrypted message Y and raise it to the power D? Here is where we really need to use our fast modular exponentiation algorithm.
This was the algorithm based on repeated squaring.
And then using this, we can compute Y to the D efficiently, mod N\@.
But if we don`t use this fast algorithm, we use a naive approach.
Then this is going to be exponential time and there`s no way we`re going to be able compute it efficiently.

\subsection{RSA  Potential Pitfalls}
let`s look at some of the issues that can arise when we`re implementing RSA first off suppose in our message little m is not relatively prime to capital n so the GCD of little m and capital n is greater than 1 now capital n is a product of two primes P and Q so it`s only divisors are P and Q so what can be the common divisor of m and capital n it`s got to be either P or Q so let`s suppose that GCD of little m and capital n is P now what happens in the RSA protocol or we take this message m raise it to the power E and look at that mod capital N and then the receiver takes this encrypted message raises it to the power D and then takes it mod N and what do we get back well this equals the original message M now we didn`t prove this case we proved the case when they`re relatively prime to each other and then if followed by Euler`s theorem and we claim that this case when they`re not relatively prime followed by the Chinese remainder theorem this is in fact true but there`s a potential problem in this case what`s the problem well P divides m and N and if we look at the encrypted message Y which is m to the e mod n or of P divides M and it divides n then it`s also going to divide Y so the GCD of Y and capital n is also going to be P what does this mean well anybody that`s eavesdropping is gonna see this encrypted message and from this encrypted message and the public key capital n by simply running Euclid`s algorithm they can take the GCD of these two numbers and they find that the GCD is little P prime P and once they know prime P then they can factorize n and therefore they can break the RSA cryptosystem they can find the decryption key little T so before using this message m and sending the encrypted version of it we have the first chat that m and n are relatively prime to each other if they share a common factor then anybody will be able to use this encrypted message to break this crypto system what are some other issues that can arise we need that little m is not too large in particular we need that little m is strictly less than capital n now typically our message is text so we first have to convert it into a number how might we do that well we can take the binary version of the text now the binary version of the text is gonna be a huge number so what we do is we break this huge number into n bit segments these are segments of length little N and therefore little m will be at most strictly less than 2 to the little n so if we break the message into little n bit strings then we get this property and if we ensure that P and Q are sufficiently large so we ensure that they`re leading bit is 1 then if we look at capital n we know that capital n will be at least 2 to the N and therefore little m will be strictly smaller than capital n so this property that little m is not too large as easy to insure but we also need that m is not too small why is that the case well suppose that little e equals 3 which is a common practice and suppose that M is very small number so m cubed is strictly smaller than n or when we look at M cubed mod n what do we get well the mod n isn`t doing anything because M cubed is smaller than n so M cubed or m to the e is the same as m to the e without the mod n so this mod n is not doing anything when this number m is too small and that means that our encrypted message the message we`re sending in clear text is simply M cubed the Madonna isn`t doing anything now if we see this message m cubed how do we decrypt it we just take the cube root so anybody seeing this encrypted message can simply take to the to brood and we get the original message back so it`s easy to decrypt it`s easy to break this crypto system in this case cube roots are difficult to do when we`re doing it with respect to modular arithmetic but in real arithmetic cube roots are easy to do so how do we avoid this issue when we have a small message well we can choose a random number R and we can look at M plus R or M exclusive or with R and we can send this new message this padded message and we can also send a second message which is just R itself so we send two messages the padded message are itself and as long as our is not too small then this will be okay and if our is too small just choose a new random string until we get an R which is sufficiently large now there`s one last issue I want to point out if we use the same message multiple times then we have a potential problem suppose we have the same message that we want to send to three different people each of these three people have a different public key but they`re all using a equals two three suppose the first recipient has e equals to 3 and n one second recipient has n 2 and e equals to three third recipient has n 3 and e equal to three and suppose we use the same message m to send to these three people now the encrypted messages are going to be different the first encrypted message is going to be M cubed mod and 1 the second encrypted message is going to be M cubed mod and 2 the third one is going to be M cubed mod and 3 so we got three different encrypted messages y1 y2 and y3 but it turns out that if somebody sees these three encrypted messages which all come from the same message same number M then they can decrypt they can figure out M from y1 y2 and y3 how do they do that they use the Chinese remainder theorem this is a homework problem in the textbook in the desc coupe 2 textbook in Chapter 1 is problem 44 now all that remains for specifying the RSA protocol is to describe how we do primarily testing so let`s dive into that

\subsection{Recap of RSA}
Let`s take a moment now to recap the RSA algorithm and also the mathematics behind the algorithm.
We start with Fermat`s little theorem, that was the basis of the whole algorithm.
Let`s recap that.
So the setting is, we have a prime number p and we have another number z, where z and p are relatively prime to each other, which means that the GCD of z and p is one.
This means that they`re relatively prime if their GCD is one.
Now, the theorem says, if we take z and raise it to the power p\_minus\_one and we look at it mod\_p, then what are get? We get one.
And that`s true for any z, so take any z which is not a multiple of p and we raise it to power p\_minus\_one we get one\_mod\_p.
Actually, we didn`t use Fermat`s little theorem, we used the generalization known as Euler`s theorem.
Now Euler`s theorem is a general theorem that holds for any capital N, but we used it for the particular case where capital N was the product of two primes, p and q.
So let`s recap it for this particular case that we`re interested in.
So, for primes p and q, look at their product, capital N, and take a z, which is relatively prime to capital N\@.
Now in this case, if we raise z to the power p\_minus\_one\_q\_minus\_one, then we get the analog of Fermat`s little theorem.
This is going to be one, when we look at it, mod\_N, where N is p\_times\_q.
Now, where did this exponent p\_minus\_one\_times\_q\_minus\_one come from? Well, that came from, we looked at how many numbers between one and capital N? This is the capital N here.
How many numbers between one and capital N are relatively prime to capital N? For the case where capital N is p, then all numbers from one to p\_minus\_one are relatively prime to this N\@.
So that`s why the exponent is p\_minus\_one.
For the case where N is p\_times\_q, then the number of numbers between one and p\_times\_q, which are relatively prime to it, are p\_minus\_one\_times\_q\_minus\_one.
That`s what we should solve before

\subsection{Recap of RSA \#2}
Now, Euler`s theorem was the basis for RSA algorithm.
Now, let`s go ahead in detail again the RSA algorithm.
The first step is to choose a pair of primes p and q.
These are the ones that we haven`t seen actually how to do.
We`re going to explain how to choose prime numbers, after we review the RSA algorithm.
Now, after we choose the pair of primes p and q, we look at their product.
Hence let N be the product of p and q.
The next step is to find an e which is relatively prime to (p-1)(q-1).
By relatively prime, again we mean that the gcd of e and (p-1) and (q-1) is one.
So they have no common factors.
Now, why did this (p-1)(q-1) come up? Because recall Euler`s theorem, that`s the exponent here.
So, what is the key implication of the fact that e is relatively prime to (p-1)(q-1)? That means that e has an inverse.
So, the third step is to find the inverse of e relative to (p-1)(q-1).
So, let d be the inverse of e mod (p-1)(q-1).
We know it exists, because e is relatively prime to (p-1)(q-1).
How do we find this inverse? We use the extended Euclid algorithm.
Now, we can publish our public key N and e.
We tell the whole world this public key N and e and anybody that wants to send us a message will encrypt that message using this public key, but nobody is going to know our private key little d.
We`re going to keep this private key d secret, because anybody that finds this private key d can decrypt messages.
Now, anybody that wants to send us a message, let`s say, m, they`re going to encrypt the message using our public key.
They take that message m, they raise it to power of e and look at a mod N\@.
And then they send that Y, which is congruent to m to the e mod N\@.
They send that message Y and the whole world can see that message Y, but only we can decrypt it, because only we know little d.
Finally, we receive this message Y, this encrypted message.
How do we decrypt it? We use little d in the following way.
We take this encrypted message Y, we raise it to the power of little d and we look at that mod N\@.
And it turns out that this equals m.
What are the algorithms that we need to use? Well, first off, to find this e, which is relatively prime to (p-1)(q-1), what do we do? We try e equals three, five, seven, 11, 13, at some point we give up.
For each of those e`s, though, how do we check whether it`s relatively prime to (p-1)(q-1)? We check its gcd which we do using Euclid`s algorithm.
Then once we find an e which is relatively prime, we compute its inverse.
How do we do that? We used the extended Euclid algorithm, and we publish our key, usually e is small, so that taking m to the e mod N is relatively easy.
If it`s not, then we can use repeated squaring, our fast modular exponentiation algorithm, and definitely here, when we`re raising this encrypted message Y to the power d, that`s probably going to be to a huge power.
So, how do we do Y to the d mod N? Here we need to use our fast modular exponentiation algorithm.
The algorithm based on repeated squaring, and that`s going to take time which is polynomial in little N, the number of bits in these numbers Y and d and N\@.
Finally, the key thing about why this works is look at what`s happening to the message.
We`re raising the message m to the power e and then to the power d.
And recall that e times d is congruent to one mod (p-1)(q-1).
So, when we apply Euler`s theorem what we`re going to get out is, we`re going to get the message m back out.
So, m raised to the power e times d modulo N is congruent to little m, because of Euler`s theorem.
What remains? We need to look at how we choose these prime numbers p and q.
The other thing is let`s just make an aside about why this algorithm is hard to break.
Notice the whole world knows N, which is p times q, but only we know (p-1)(q-1), and therefore only we can compute the inverse of e mod (p-1)(q-1).
The point is, can you get (p-1)(q-1) from N without knowing p and q, individually? The assumption is that no, we can not do that, that the only way to get (p-1)(q-1) is to get factor N into p and q.
If you don`t know how to factor N into the pair of primes which compose it, then you cannot get (p-1)(q-1).
That`s our assumption.
So, this algorithm is as hard as factoring N\@.
We don`t know any other way to get (p-1)(q-1).
And our assumption is that factoring is difficult, computationally difficult to solve.
Now, to finish off the RSA algorithm, let`s look at how we find prime numbers p and q.

\subsection{Random Primes}
Now, we want to choose a pair of primes P and Q\@.
One way to do that, is to have a table of prime numbers, and then just go through that table of prime numbers.
What`s the problem with that? Well, if somebody has access to our table, then it`ll be easy to crack our cryptographic scheme.
So, we want something more secure.
So what`s the better approach? A better approach is to choose random primes P and Q, and we want these primes P and Q to be chosen, so that every time we run the algorithm they are being chosen independently from previous runs.
So, how do we choose these primes at random? Well, first off how do we choose random numbers? That`s, what we`re going to do first.
Let R be a random N-bit number.
How do we choose R? Let`s say, little N is six, actually in practice though, little N is going to be a huge number like a 1000 or 2000.
How do we generate this random 6 bit number? We have a one dimensional array of size 6 and we generate each of the bits, for each bit we choose a random number 0 or 1.
And, we make sure that every bit is generated independently of every other bit.
And, then every time we run the algorithm, the bits are generated independent of previous runs of the algorithm.
So, this is quite easy to generate random N-bit number, but we want a random N-bit prime number.
So what do we do? We choose a random N-bit number regardless of whether it`s prime or not.
And then we check whether this random number is prime or not.
How do we do that? We`ll see how to do that in a moment.
But, suppose we have a test for primality, so we given a number we can check whether it`s prime or not.
Then, if this random number happens to be prime then, what do we know? Then, we know it`s a random prime number.
So in that case, if R is prime we can output it, because it is a random prime number.
What do we do if it`s not a prime number? Then we repeat this procedure.
We generate a new random number, check whether it`s prime or not.
How long is this algorithm going to take before it finds a prime number? Well the key thing is that primes are dense.
What do we mean more precisely? The probability that this random number R, happens to be a prime number is roughly 1 over little N, which is the number of bits in it.
So, for generating a 1000 bit number or a 2000 bit number the probability that it`s going to be prime is about 1 over a 1000 or 1 over 2000, which means how many times are we going to have to repeat this procedure, before we find a prime number? In expectation, is going to be about little N, it`s going to be about a 1000 or 2000 times, which is not a big deal, to repeat this a 1000 or 2000 trials before we find a prime number.
But the question remains, how do we check whether a given number R is prime or not?

\subsection{Primality  Fermats Test}
Now, we`re going to derive a Primality Testing Algorithm using Fermat`s Little Theorem.
Let`s first recall Fermat`s Little Theorem.
If r is a prime number, then for all z in this set, all z between one and r minus one.
If we look at z raised to the power r minus 1 modulo r, then it`s congruent to one.
Now, we`re going to use this as the basis for figuring out whether a given number r is prime or not.
So, if it`s prime it satisfies this.
What if it`s composite? We`ll, what if we find a z between one and r minus one, where z raised to the power r minus one modulo r is not controlling to one.
We`ll, by Fermat`s Little Theorem if this is the case, then what do we know about r? We`ll, r must be composite if there`s such a z.
So, this z where z raised to r minus one is not congruent to one mod r is a witness.
It proves that r is composite.
And hence, we call this z a Fermat witness because it`s a witness with respect to Fermat`s Little Theorem.
The first question we want to ask is whether every composite number has a Fermat witness.
The answer to this first question is, yes it does.
Every composite number has at least one Fermat witness and we`ll demonstrate this in a moment.
After we demonstrate this, then we have to look at how we find a Fermat witness.
After we show that every composite number has at least one Fermat witness, then we`ll look at how to find Fermat witnesses.
What will show is that every composite number has many Fermat witness.
We`ll, there are some exceptions called Pseudo Primes but if we ignore Pseudo Primes, then every other composite number has many Fermat witness.
So, in fact it will be easy to find Fermat witnesses for composite numbers.
And that will give us our Primality Testing Algorithm.
And then, we`ll see how to deal with these Pseudo Primes.
Let`s first prove this fact that every composite number has at least one Fermat witness.

\subsection{Trivial Witness}
Once again, we have a number R and we`re trying to determine if R is prime or composite.
Now we say number Z is a Fermat witness for R\@.
First off, Z lies between one and R minus.
And crucially Z raised to the power R minus one is not congruent to one mod R\@.
Then by Fermat`s little theorem, we know that R is composite.
Because if R is prime every such Z, Z raised to the power R minus one, is congruent to one mod R\@.
So if we get a Z where is not true then this R must be composite.
So Z is a witness to the fact via Fermat`s little theorem that R is composite.
We want to first prove that every composite number has at least one Fermat witness.
So let`s prove this fact.
Now, R is composite so we know it has at least two divisors.
So take a Z which is a divisor of R\@.
So clearly this Z, which is a divisor of R, lies between one and R minus one.
What else do we know about this Z? Well, look at its GCD\@.
What do we know about the GCD of R and Z? Well, if we take Z to be a divisor of R then we know the greatest common divisor of these two numbers is Z itself.
And since R is composite, we know this divisor is greater than one.
Well, actually, this proof works for any Z where this statement is true.
So any Z which is not relatively prime to R\@.
And if R is composite, we always know there is at least two such Z`s which are not relatively prime to R\@.
Now, if the GCD of R and Z is greater than one, what do we know? Well, if you recall from our previous lecture about modular arithmetic, what about the inverse of Z mod R? We know it does not exist.
There is no inverse of Z mod R\@.
Why? Because the inverse of Z mod R exist, if and only if the GCD of R and Z is one.
They`re relatively prime to each other.
Now, we want a proof for this Z that Z raised to the power R minus one is not congruent to one mod R\@.
What are we going to do? Let`s suppose that it is congruent to one mod R\@.
So suppose that Z raised to the power R minus one is congruent to one mod R\@.
That`s the opposite of what we`re trying to prove.
So what will show is that, if this holds then we get a contradiction.
How will we get a contradiction? Will show that Z has an inverse mod R\@.
And we know that`s not the case therefore we have a contradiction, therefore this assumption cannot hold, and therefore Z raised to the power R minus one must not be congruent to one mod R\@.
Okay? Now, we have the fact that Z raised to the power R minus one is congruent one mod R\@.
Now, Z raised to the power R minus one is the same as the following.
It`s the same as Z raised to the power R minus two times Z\@.
This product is Z raised to the power R minus one.
So if Z raised to the power R minus one is congruent to one mod R, then Z raised to the R minus two times Z is also congruent to one mod R\@.
Now, look at this statement.
Z times Z to the R minus two is congruent to one mod R\@.
That means that Z has an inverse.
What`s its inverse? It`s this number Z to the are minus two.
Because when we look at the product of these two, it`s one mod R\@.
That`s the definition of the inverse.
So Z to the R minus two is the inverse of Z and Z is the inverse of Z to the R minus two.
But we know that Z does not have an inverse mod R, therefore we have a contradiction.
And thus, this assumption cannot hold and therefore we know that Z to the R minus one is not congruent to one mod R\@.
And that proves that every composite R has at least one Fermat witness.

\subsection{Non-Trivial Witness}
Now we call this Fermat witness that we just derived a trivial Fermat witness.
Why is it a trivial Fermat witness? This is a Fermat witness, z, where it also has the property that the greatest common divisor of z and r is greater than 1.
So z and r are not relatively prime.
Notice, such as z already proves that r is composite.
If z and r have a non-trivial divisor, then we know that r has a non-trivial divisor, and therefore, it`s not prime.
So, any z where this is the case, proves that r is composite, there`s no reason to run Fermat`s test.
So that`s why we called such a z, a trivial Fermat witness.
Now there always exists a trivial Fermat witness for composite numbers.
Why? Because every composite number has at least two non-trivial divisors.
The question is whether a composite number r, has a non-trivial Fermat witness.
This is a number z which is relatively prime to r.
Now it turns out that some composite numbers have no non-trivial Fermat witnesses, these are called pseudo primes, but those are relatively rare.
For all the other composite numbers, they have at least one non-trivial Fermat witness, and if they have at least one, then in fact, they have many Fermat witnesses.
And therefore it will be easy to find a Fermat witness and that`s the key point.
Trivial Fermat witnesses always exist.
Every composite number has at least two trivial Fermat witnesses.
But if a composite number has a non-trivial Fermat witness, then there are many Fermat witnesses, they`re dense and therefore they`re easy to find.
And that`s what we`ll utilize for our primality testing algorithm.

\subsection{No Non-Trivial Witnesses}
Recall, for a number r, if we`re checking whether r is prime or not, we say that z is a Fermat witness, if it proves that r is composite, according to Fermat`s little theorem.
What does that mean exactly? That means, that if z to the r minus one is not congruent to one mod r, then that z proves that r is composite.
Now, we say it`s non-trivial, if in addition GCD of z and r is one, so z and r are relatively prime.
Because if GCD of z and r is greater than one, so they`re not relatively prime to each other, then that z gives us a non-trivial divisor of r, and therefore we know already, trivially, that r is composite.
Now, the question is, does every composite number have a non-trivial Fermat witness? We know it has a trivial Fermat witness.
Namely, if it has a one of its divisors is a trivial Fermat witness, doesn`t have a non-trivial Fermat witness.
Well, in fact there are composite numbers which do not have non-trivial Fermat witnesses.
These are called Carmichael Numbers.
These are sort of pseudo primes.
Okay.
They behave like primes with respect to Fermat`s little theorem.
A Carmichael number is a composite number r, which has no non-trivial Fermat witnesses.
Therefore, such a number is going to be inefficient to use Fermat`s test, to prove that r is a composite number.
We`re going to have to find a different way to deal with Carmichael numbers.
But, the key thing is that Carmichael numbers are rare.
The smallest one is 561 and 1105 and so on.
But, since they`re relatively rare, let`s ignore them for now and when we ignore Carmichael numbers, we`re going to have a simple algorithm to check whether number r is prime or not, using Fermat`s test and trying to find a Fermat witness when it`s composite.

\subsection{Primality  Many Witnesses}
So let`s ignore Carmichael numbers for now since they`re relatively rare.
Let`s assume that every composite number r, has at least one non-trivial Fermat witness.
If we assume that it has at least one non-trivial witness, how many witnesses does it have, actually? How prevalent are these non-trivial witnesses? The key fact is that, if r has at least one non-trivial Fermat witness, then there are many non-trivial Fermat witnesses.
In fact, if we look at the set of possible witnesses, one through r minus one, then at least half the numbers in this set are Fermat witnesses.
Now, this is a relatively simple fact to prove and it`s a very nice proof.
The proof is in the book.
But let`s get the proof for now and let`s try to use this, Lemma, to have an algorithm for checking whether a number is prime or not, when we ignore Carmichael numbers.
So we assume that a composite number has at least one non-trivial witness.
Then this Lemma tells us that at least half the numbers in this set are Fermat witnesses, are witnesses to the fact that r is composite.
Can we use this fact now to get a test for whether r is composite or not? The point is that when r is prime, all of these numbers, when we raise them to the power r minus one is congruent one mod r.
And when r is composite, then at least half these numbers, when we raise them to the power r minus one is, are not congruent one mod r.
So how are we gonna check whether a number is prime or not? We`re going to take a z, from this set, and raise it to the power r minus one.
Look at mod r and see whether it`s one or not.

\subsection{Simple Primality Alg.}
We have an n-bit number r, and here`s our simple algorithm for checking whether r is prime or not based on Fermat`s Little Theorem.
And this is ignoring Carmichael numbers, it will be an efficient algorithm for primality testing.
Recall, that if r is composite then, at least half these numbers in this set are witnesses to that fact using Fermat`s Little Theorem.
Whereas, if r is prime, then none of these numbers in this set are witnesses to the fact because r is prime.
So, what are we going to do? We`re going to choose a Z from this set.
How do we choose a Z from this set? Well, we choose a Z randomly, uniformly at random from this set.
So, we choose a random number between one and r\_minus\_one.
Now, we do Fermat`s Test for this Z\@.
So, we compute Z raised to the power r\_minus\_one\_mod\_r and we check whether that`s congruent to one, or not.
If Z raised to the power r\_minus\_one is congruent to one\_mod\_r then what do we know? Actually, we don`t know for sure either way, but we think that r is prime.
What happens if Z raised to r\_minus\_one is not congruent to one\_mod\_r? Then, in this case we`re sure that r is composite because this Z is a witness to the fact that r is composite.

\subsection{Primality Alg.
Analysis}
Let`s take a look at how this simple Primality testing algorithm performs.
If r happens to be prime, what does the algorithm do? It always outputs that r is prime, because for every z in this set, z raised to the power r\_minus\_one is congruent to one\_mod\_r, by Fermat`s little theorem.
So, the probability that the algorithm outputs that r is prime, is one, this is always going to do so.
So, it`s always correct when r is prime.
Now, what happens when r is composite? And let`s assume that r is not a Carmichael number.
Now, sometimes the algorithm is going to be correct, it`s going to output that r is composite.
When is that the case? When it finds a z which is a Fermat witness.
So z raised to the power r\_minus\_one is not congruent to one\_mod\_r.
But sometimes it`s going to get confused, it`s going to make a mistake, and it`s going to find a z, which z raised to the power r\_minus\_one is congruent to one\_mod\_r.
So it`s going to think that r is prime.
What is the probability that the algorithm outputs that r is prime, so, it makes a false positive statement? Well, we know that at least half the zs in this set are Fermat witnesses.
So what`s the chance that it finds a non witness? Well, at most half of them are not witnesses, so therefore, the probability of finding a non-witnesses is, at most, a half.
So, the probability of a false positive is, at most, a half.
Okay, so, we have a reasonable algorithm with probability of most a half that we get a false positive, and when it is prime, it`s always correct.
But can we get this better? Can we improve this error probability of a false positive?

\subsection{Boosting Success}
Here again is our original primality testing algorithm.
The problem was that it had a false positive rate of at most a half.
We want to prove that false positive rate.
We want to get it down smaller.
So, what are we going to do? Recall the algorithm starts by choosing a random number between one and R minus one, and then we run for a max test for that Z that we chose at random.
Now, what we`re going to do is instead of choosing one rule number at random from this set, we`re going to choose K numbers at random from this set, where K is a parameter that we`re going to choose.
Now, for all of these K numbers, we`re going to run for a max test.
So we`re going to take that ZI and we`re going to raise it to the power R minus one and we`re going to check mod R, whether it`s congruent to one or not.
Now what we need is just that one of these ZI`s is a Fermat witness.
If any of these ZI`s is a Fermat witness, so ZI raised to the power R minus one is not congruent to one mod R then we know for sure that R is composite.
Whereas if all of these ZI`s are not witnesses, then we have a very good chance that R is prime.
So our final check is whether any of these ZI`s is a witness or not.
So if for all I`s it passes the test, so ZI raised to the power R minus one is congruent to one mod R, then what do we know? We know there`s a good chance that R is prime.
And if any of these ZI`s is a witness then what do we know? Then we know for sure that R is composite\. more precisely let`s look at our analysis from before.
Suppose that R is a prime number, what does the algorithm do? It always outputs that R is prime because every ZI, when we raise it to the power R, is going to be congruent to one mod R by Fermat`s little theorem.
So the probability the algorithm outputs R is prime is one, when R is in fact prime.
What happens when R is composite but not Carmichael? The previous algorithm had a false positive with probability at most a half, because at most half of the Z`s are non-witnesses.
Now, we`re choosing KZ`s.
We just need that one of them is a witness.
What`s the probability that none of them are witnesses? So think of the following analogy: say it`s a witness if I flip a coin and it`s a heads, and if it`s tails it`s a non-witness.
So the probability of a tails for each of these K flips is at most a half.
So suppose it was a fair coin, what`s the chance that I have K tails? That means that all K of these are non-witnesses.
If I have just one heads, or at least one heads, then I have a witness.
What`s the chance that none of them are witnesses? So the chance that all K of them are tails? That`s at most one half to the K\@.
So the probability of a false positive in this scenario is at most one half to the K\@.
So if I choose K to be, let`s say, 100.
So I choose 100 numbers at random from this set, this is a huge set, this is about the order two to the 1000, or two to the 2000.
So choosing 100 numbers from there, running this procedure 100 times is not much of a cost.
Then the probability of a false positive is at most one half to the 100, which is a tiny, tiny probability.
So there is a very minuscule chance of that.
So I`m willing to take that chance of a false positive, in which case my scheme, my cryptographic scheme might be easy to break, the chance of that is very minuscule.
So that completes our primality testing algorithm in the case when we assume that the composite numbers are not Carmichael numbers.
So we`re ignoring these pseudo prime numbers.
It turns out that, actually, to deal with these pseudo prime numbers, Carmichael numbers, it`s not that much more complicated of an algorithm.
That`s detailed in the textbook and I`ll leave that to there.

\subsection{Addendum  Pseudoprimes}
Here is the general algorithm for primality testing which handles Carmichael numbers.
The algorithm is essentially the same as before with one small observation.
I`ll explain the algorithm using an example.
Let`s consider the example of 1,729, this is a composite number, in fact, it`s a Carmichael number.
So our previous algorithm, is unlikely to detect that is compiler.
Now, let`s first recall our previous approach.
We first choose a random number Z between one and 1,728.
For concreteness and simplicity, let`s suppose that Z is five, I chose a small number to make it simpler.
Now, our previous approach takes these numbers Z, which is five, and we raise it to the power of 1,728, which is X minus one, and we take that mod 1,729, which is X\@.
So, we look at Z raised to the power X minus one mod X\@.
Now, if this is not one, then we have a proof that this number is composite.
Well, since this is a Carmichael number, this is unlikely to work and in fact it doesn`t work.
Five raised to the power of 1,728 is congruent to one, mod 1,729.
So, firmus test fails in this case.
Well, let`s go back and look at how we compute five raised to this power.
Well of course, we do use repeated squaring.
Now, in the spirit of repeated squaring, let`s take this number 1,728 and let`s take out all the factors of two that we can.
Now, notice that this number is even Y, while this number X, we`re checking whether it`s prime or not.
So, it`s odd and therefore X minus one is even or 1,728 is equal to two times 864, we take out another factor of two and repeat.
We can do it six times, so we get two to the six times 27.
We stop when we reach an odd number.
Let`s start a new approach by computing five raised to the power 27 mod 1,729.
Now of course, this exponent might be very large, so of course, we`re going to use the fast modular exponentiation algorithm to compute it.
Suppose we ran it, and we computed this exponent, it turns out it`s congruent to 1,217.
Now, let`s apply repeated squaring six times to get to this result.
So let`s take this result and square it.
So, we`re computing five to the power 54, and we`re doing this mod X\@.
So we take the previous results squared, and that`s congruent to 1,065 mod X, then we take this previous result and we square it, and it turns out that 1,065 squared is congruent to one mod X\@.
Now, we continue, of course once we get one it`s going to continue one, so the next result will be one squared which is one of course, and we square it again and we`re going to get one again.
We do it a few more times, and eventually we get to five raised to the power, two to the six times 27, and that will be one, which we know from before, mod 1,729.
Now, let`s look backwards.
So, we end with one here, we get this string of ones.
Let`s look at the first number which is different from one, what do we know about it? In this case, it`s 1,065 but 1,065 squared is one mod X\@.
Since 1,065 squared is one mod X therefore, 1,065 is a square root of one mod X\@.
In fact, it`s what we call a non-trivial square root of one mod X\@.
Why is it non-trivial? What are trivial squared roots of one mod X? Well, the trivial square roots of one mod X are one and minus one.
Why? Well, we always know that one squared is one and negative one squared is one.
That`s true in real arithmetic and it`s also true in modular arithmetic.
So, every number X has these two trivial square roots of one mod X\@.
It turns out that prime numbers X only have these two square roots of one mod X\@.
So, one and minus one are the only two square roots of one mod X when X is prime.
But if we can show a non-trivial square root of one mod X then therefore, that implies that X is composite because prime numbers only have the trivial ones.
So, if we show a non-trivial one, that proves that X is composite.
In this case, we`ve shown that this number X has a non-trivial square root of one, namely 1,065.
It turns out that for a composite number X, even if it`s Carmichael for at least three quarters of the choices of Z, this algorithm works.
Namely, if we compute Z raised to the power X minus one mod X and if we get one and we go backwards in this approach, so we use this repeated squaring and then we work backwards from the one to the first non-one, this leads to a non-trivial square root of one mod X, and therefore, proves that X is composite and this works for at least three quarters of the choices of Z\@.
Now, the mathematics for proving that at least three quarters of the choices of Z work is quite complicated but the algorithm itself that we`re using here is basically the same as before with the repeated squaring added in.
So, to deal with Carmichael numbers, we use basically the same algorithm as before, except when formats test fails, we go back and we check whether we get a non-trivial square root of one mod X\@.

