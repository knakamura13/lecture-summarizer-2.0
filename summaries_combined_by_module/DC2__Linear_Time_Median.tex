\section*{Linear Time Median}

\subsection*{Median Problem}
Divide and conquer strategy can efficiently solve the median finding prob\. in an unsorted list A\@.
For n even, median is defined as the smallest element at n/2 ceiling, for n odd, it`s the (n/2)+1 smallest.
The goal extends to finding the kth smallest, where 1 \textless{}= k \textless{}= n.
Trivially, if A is sorted, select kth element from A\@.
Normally, sorting A using algorithms like merge sort takes n log n time, hence a kth smallest element finding algo would also take n log n time if sorted array is a prerequisite.
However, the Blum et al\. algorithm from 1973 achieves this in linear n time without pre-sorting, showcasing a significant optimization.
This was innovatively deduced in a single lunchtime session.

\subsection*{QuickSort}
Adopting a divide and conquer strategy similar to quicksort for solving a problem.
Quicksort involves choosing a pivot and partitioning an array into elements less than, equal to, and greater than the pivot.
Quicksort then recursively sorts the lesser and greater elements, ultimately combining them.
Choosing an effective pivot, ideally the median, is critical as a poor choice leads to inefficient O(n\textasciicircum{}2) performance.
The goal is to improve on quicksort`s O(n log n) by only recursively processing one partition.
The challenge lies in selecting an appropriate pivot to achieve O(n) time complexity.

\subsection*{Search Example}
Sorting an array of n nums using a chosen pivot, 11, into \textless{}p, =p, \textgreater{}p buckets.
Scan array, assign nums to corresponding buckets.
Dual occurrences of pivot noted.
K`s value determines the target list for Kth smallest num: if K \textless{}= the smallest list size, search only that list; if K=5 or 6, output pivot; if K \textgreater{} 6, search for (K-6)th smallest in \textgreater{}p list.
Recursive search limited to one list, unlike QuickSort which sorts both \textless{}p and \textgreater{}p lists.
Pseudocode forthcoming to illustrate the general case.

\subsection*{QuickSelect}
Algorithm to find the K smallest elements in an unsorted list A involves selecting a pivot P and partitioning A into three sub-arrays: elements smaller than, equal to, and larger than P\@.
The pivot selection is crucial.
Depending on K`s value and sub-array sizes, the search for the K smallest recurses into the appropriate sub-array.
If K is within the size of the smallest array, the search continues there.
If K equals the size of the smallest array plus one, the pivot P is the K smallest element.
If K extends into the larger sub-array, the search recurses there for K adjusted by the sizes of the other sub-arrays.
The crux is how to choose an efficient pivot that enables the algorithm to run in O(n) time.

\subsection*{D\&C  High-level idea}
Aiming for O(n) run time using divide and conquer, we consider recurrences for an O(n) solution.
The recurrence T(n) = T(n/2) + O(n) solves to O(n), implying an algorithm where the pivot P is the median of the list, giving subproblems each at most n/2 size.
This satisfies the recurrence, but finding the median is the original problem.
A pivot that is an approximate median will still work, within the n/4 to 3n/4 smallest elements in the list.
Worst-case, subproblems reach 3n/4 size, but the recurrence T(n) = T(3n/4) + O(n) still solves to O(n).
Even if the pivot is merely guaranteed between the n/100th to 99n/100th smallest, the subproblem at .99n size still leads to O(n).
The key is using a constant fraction less than one.
Thus, a `good pivot` is within n/4 to 3n/4 smallest elements, creating a recurrence mimicking the latter but allowing some slack; a good pivot must satisfy the constant fraction condition for the algorithm to meet the runtime goal of O(n).

\subsection*{Goal  Good Pivot}
Pivot p is good if it ranks between n/4 and 3n/4 in a list, guaranteeing at most 3n/4 elements are either smaller or larger than p.
Need to find such a pivot in linear time (O(n)) to enable a recurrence where the subproblem`s size is at most 3n/4, also solved in linear time, leading to an overall linear time solution.
Main challenge: devise a method to identify a good pivot within O(n) time.

\subsection*{Random Pivot}
Selecting a good pivot in an array A can be done by choosing a random element, with a 1/2 probability of it being good, based on the median and quartiles.
This random pivot is good if it`s within the middle n/2 elements when sorted.
To check if a pivot is good, partition A into elements smaller, equal, or bigger than the pivot, which takes linear time.
If it`s bad, select another until a good one is found--akin to flipping a coin with a 1/2 chance of success, expecting to find a good pivot after an average of two tries.
Although this method has an expected linear runtime, the aim is an algorithm with a worst-case linear runtime.

\subsection*{D\&C  Recursive Pivot}
To achieve an order n time algorithm for finding a good pivot, the algorithm utilizes a recursive strategy where the pivot is the median of a subset of size n/5 of the original array A\@.
This selection guarantees that the sub-problems are reduced to a maximum size of three-fourths of the original, with the optimal running time for the algorithm satisfying T(n) \textless{}= T(3/4 n) + T(n/5) + O(n).
This fits a recursive pattern that results in overall linear time complexity, O(n).
Key to this success is ensuring the subset S is a representative sample of the entire array A; the method for choosing S is critical and initially approached naively, with improvements derived from the lessons of that naive approach`s failure.

\subsection*{Representative Sample}
Choosing the first n/5 elements of an array A as the subset S to determine a pivot results in a suboptimal pivot when A is sorted.
The median of S in such a case would be the n/10th smallest element, leading to an imbalanced partition where the larger side can have up to 9/10ths of n elements.
This makes the pivot ineffective for balanced partitioning as it skews the algorithm`s running time towards a worst-case scenario (t(n) being reduced only to t(9/10 n) instead of the more efficient t(3/4 n)).
Considering this, a more strategically selected set S, by incorporating some analysis of A, could potentially offer a better pivot for the algorithm.

\subsection*{Recursive Rep.
Sample}
Aiming to select a subset S from set A that approximates A`s median; S`s median mediates the elements well, with each of its members (like X) having at least two elements from A that are both \textless{}= X and \textgreater{}= X\@.
By partitioning A into groups of five (assuming N, the number of elements in A, is a power of five for clean partitioning), we can isolate N/5 representatives by selecting the median from each group.
These medians concurrently represent their groups and the larger set A, ensuring S aptly mirrors A`s makeup.
To find a suitable pivot for array A to utilize in algorithms, perform the following: divide A into groups of five, sort each--sorting takes constant time due to the small group size--select medians as reps.
The aggregate of these N/5 medians forms S\@.
The median of S is then found recursively, serving as the pivot for A\@.
We`ll present the algorithm`s pseudo code and substantiate both the pivot`s effectiveness and the algorithm`s runtime.

\subsection*{Median  Pseudocode}
Developed a linear time median algorithm, taking an unsorted array A of size n and an integer k to output the k-th smallest element.
The process involves dividing A into groups of five elements, G\_i, and sorting them to find median mi for each group.
All group medians, set S, are collected, and its median (pivot P) is determined using a recursive call of the algorithm seeking the n/10-th smallest element in S\@.
Using P to partition A into three subsets (smaller, equal, larger than P), the algorithm then applies a quick select process.
The target k-th smallest element is found by recursively calling the algorithm on the appropriate subset (smaller or larger) or returning P if it lies within the equal subset.
The algorithm`s efficiency is predicated on the assumption of P being a good pivot, which is substantiated later in the analysis.

\subsection*{Median  Running Time}
To demonstrate P as a good pivot, the running time of the algorithm was analyzed step-by-step.
Breaking array A into groups of n/5 takes linear time, O(n).
Sorting each five-element group is constant time, O(1), thus sorting all groups is O(n).
The median of medians algorithm recursively runs on a subset S (N/5 size), taking T(N/5) time.
Partitioning A into three subsets and recursing on one takes O(n) and is bounded by 3/4N\@.
The recurrence relationship considering 3/4N, N/5 for the median of S, and additional O(n) operations shows the sum is less than one, resulting in a linear time complexity, O(n), for the algorithm.
Ensuring P is a good pivot is pivotal for the algorithm`s completion.

\subsection*{Correctness}
Proved P, chosen as pivot in algorithm, effective by median-of-medians approach.
Partitioned array A into n/5 groups of 5, finding each group`s median.
Recursively found median of these medians, determining P\@.
Relabeled groups by median size for proof clarity, identified G1 with smallest and Gn/5 with largest median.
Focused on middle group Gn/10, median becomes pivot P\@.
Identified proved elements \textless{}= P: n/10 medians from subset S plus 2 guaranteed smaller elements per group, yielding \textgreater{}= 3n/10 elements \textless{}= P\@.
By excluding these from A, subset \textgreater{} P is \textless{}= 7n/10, smaller than 3/4n definition of a good pivot.
Similarly, \textgreater{}= 3n/10 elements \textgreater{}= P, leaving \textless{}= 7n/10 elements \textless{} P, confirming P as a good pivot by ensuring both partitions around it contain at most 7n/10 elements, meeting good pivot criteria.
Proof of claim complete.

\subsection*{HW  Groups of 3--7}
To understand the algorithm, break array A into subgroups of different sizes, specifically three and seven, and analyze the recurrence relations for these cases.
Determine if these relations still resolve to order n, as they do with subgroups of five, which would explain why groups of five elements are commonly used.

