\section*{Max Flow Min Cut}

\subsection*{Lecture Outline}
Proving the max-flow min-cut theorem and its application to image segmentation is the initial focus.
Subsequently, establishing the accuracy of the Ford Fulkerson and Edmonds-Karp algorithms, deriving their correctness implicitly from the aforementioned theorem.
Prior discussion of the Ford Fulkerson algorithm provides groundwork for these proofs.

\subsection*{Recap  Ford-Fulkerson}
Ford-Fulkerson algorithm applied to Max flow problem on flow network, directed graph G with source (S) and sink (T) vertices, plus edges with given positive capacities.
For analysis, usually presuppose integer capacities; This condition temporarily disregarded for general problem.
Max flow problem`s goal: determine maximum flow from S to T\@.
Running time analyzed, but not maximum size proof of flow generated by Ford-Fulkerson.
Stopping condition: no augmenting path in residual graph G\textasciicircum{}(f*), from S to T\@.
Lack of augmenting path signifies no further flow increase possible, confirming the flow as Max flow.
Thus, the absence of a S-T path in the residual graph proves that the current flow is the maximum possible.

\subsection*{Quiz  Verifying Max-Flow Question}
To determine if a flow F in a given network is max-flow, time required expressed in Big O notation relative to N vertices and M edges.

\subsection*{Quiz  Verifying Max-Flow Solution}
Checking if a flow f is maximum takes O(n+m) time based on the lemma that if no augmenting path exists in the residual graph from source s to sink t, flow f* is max.
Construct the residual graph and run DFS from s to locate a path to t.
No path means f is max-flow and output `yes.` If a path is present, flow can be augmented, thus f isn`t max-flow as a larger flow is possible.
Further details on lemma proof and max-flow min-cut theorem to be revisited.

\subsection*{Min-Cut Problem}
Cuts in graphs partition vertices into two sets, labeled L (left) and R (right).
St-cuts specifically separate vertices S (in L) and T (in R).
Example: vertices S, A, B, F form L, with no requirement for connectivity within the set, and C, D, E, T comprise R\@.
Capacity of an st-cut concerns directed edges from L to R, calculating the sum of capacities of these edges, ignoring edges from R to L\@.
Edges like C to F, entering L, are excluded from this calculation.

\subsection*{Problem Formulation}
Define the min st-cut problem in directed graphs: identify a cut (L,R) minimizing the sum of edge capacities from L to R\@.
L includes start vertex s, R includes end vertex t.
Example provided where L contains s, a, b, f with a cut capacity of 27.
Presented an optimal st-cut with capacity 12, where R contains only t and L contains all other vertices.
Max-flow-min-cut theorem links max flow size to min st-cut size, validating optimality of the demonstrated st-cut with a corresponding flow of 12.

\subsection*{Max-Flow = Min st-Cut}
The Max-flow-min-cut theorem asserts the equivalence between the maximum possible flow from source vertex s to sink vertex t in a flow network and the smallest capacity among all s-t cuts in the network.
Although the minimum capacity cut issue doesn`t inherently require distinguishing between source s and sink t, in the context of the theorem, distinction becomes necessary, hence the min st-cut terminology.
To prove the theorem, one must establish that the max-flow`s size cannot exceed that of any given st-cut and conversely as well.
The initial part of the proof demonstrates that any flow`s magnitude within the network is necessarily limited by the capacity of an arbitrary st-cut.
This concept is extended to argue that it`s also true when the flow is maximized and the st-cut has the minimum capacity, thereby proving one direction of the theorem.
The proof`s complexity is thus reduced to showing this relationship for any chosen flow and st-cut without the need to confirm their optimality within the network.

\subsection*{Max-Flow  Min st-Cut}
Max-flow size equals the flow from the source (s) to the sink (t), and must be less than or equal to the capacity of the min st-cut in a network.
In an example network, cut L-R has a capacity of 27, representing the edge capacity from L to R, the only exit path for the flow from s to t.
Flow can`t surpass edge capacities, so max-flow`s size is limited to 27.
Proving that the flow size equals the net flow out of set L (total out minus total in) will confirm that max-flow is at most the cut`s capacity.
This proof is straightforward and also contributes to showing the max-flow`s size matches at least the min st-cut capacity, a fact to be demonstrated later.

\subsection*{Proof of Claim}
For any flow f and s-t cut L, R in a graph, the flow size is established to equal the total outflow from set L minus the inflow into L\@.
The outflow from L is the sum of flows on edges from L to outside L, while the inflow to L is the sum of flows on edges from outside L to L\@.
To maintain equality while examining the flows on edges within L, both inflow and outflow are added and subtracted.
When combined, the positive terms (1st and 3rd) account for the total outflow from vertices in L, while the negative terms (2nd and 4th) give total inflow to vertices in L\@.
With the source s in L having zero inflow, the terms for s adjust the total to just the outflow from s.
For vertices in L excluding s, the inflow equals the outflow due to flow conservation, nullifying their contributions to the total and leaving only the outflow from s.
This conclusively shows that the flow size f equals net outflow from set L, meeting the proof`s objective.

\subsection*{Finishing Off}
Proven that flow F`s size equals the outflow from L minus inflow to L, for any flow and ST cut.
Size of F \textless{}= capacity of ST cut (L, R), indicating max flow \textless{}= min ST cut cap.
This follows because the outflow from L must be \textless{}= the sum of edge capacities between L and R\@.
Example: ST cut cap is 27; thus, outflow from L \textless{}= 27, proving max flow size \textless{}= min ST cut capacity.
Next step: show max flow size \textgreater{}= min ST cut capacity to complete the theorem proof.

\subsection*{Reverse Inequality}
Max flow size \textgreater{}= capacity of min ST-cut is demonstrated.
Prev shown max-flow size \textless{}= min ST-cut capacity.
Using Ford-Fulkerson algo, for flow F* w/no augmenting path in resid network, an ST-cut (L,R) is constr w/capacity = size of F*.
Showing size of F* = cap of (L, R) cut implies max flow size \textgreater{}= cap of min ST-cut.
Since F* is a specific flow, its size is \textless{}= max flow, and cap of constructed (L, R) cut is \textless{}= min cut cap.
Therefore, proving the equality for F* confirms the broader inequality: max flow size \textgreater{}= min ST-cut cap.
Just need to verify the proposed equality.

\subsection*{Proof of Claim}
Proved an s-t cut exists corresponding to flow f* with no augmenting path in residual network.
L is the set of vertices reachable from vertex s; since t is unreachable from s, it resides in the complementary set R\@.
Thus, (L, R) forms an s-t cut with capacity equal to flow f* size.
Verified s in L and t in R, confirming separation and cut`s validity.

\subsection*{Properties of Cut}
Examined properties of a constructed flow network example; identified capacities and flows (red/green numbers).
Residual network lacks certain edges, reflecting flow and capacity utilization.
Defined vertex set L as source-reachable in residual network, complement being set R, together creating ST cut LR\@.
Notably, sink vertex T resides in R, confirming ST cut validity.
Key residual network edge properties: 1) Edges leaving L to R (e.g., D -\textgreater{} T, D -\textgreater{} C, S -\textgreater{} C) are absent due to reachability contradiction; thus such edges are at full capacity in the original network.
2) Total flow departing L equals capacity from L to R, since all L to R edges are fully utilized.
3) Edges from R to L in original network (e.g., F -\textgreater{} E) carry zero flow in flow f* because corresponding reverse edges are absent in residual network, precluding flow into L\@.
The observations yield that for ST cut LR, total flow out of set L matches LR`s capacity, and total flow into L is zero.
Concluding the flow f* equals the capacity of cut LR supports the max-flow min-cut theorem`s proof.

\subsection*{Completing the Proof}
Proved the max flow min cut theorem: max flow size equals min s-t cut capacity.
Constructed an s-t cut from a flow without an augmenting path in the residual network--t isn`t reachable from s, creating a valid s-t cut.
Showed flow size f* equals the cut capacity, indicating both are optimal (max flow, min s-t cut).
Ford-Fulkerson and similar algorithms are correct if they stop with no augmenting path.
Demonstrated a method to verify max flow and construct a min s-t cut using max flow f* and vertices reachable from s in the residual network.
This method applies both to prove algorithm correctness (Ford-Fulkerson, Edmonds-Karp) and in image segmentation applications.

