\section{MF4: Edmonds Karp Algorithm}

\subsection{Max-Flow Min-Cut Algorithms}
Now we`ve seen the Ford-Fulkerson algorithm for solving the max-flow problem.
And in this lecture, we`ll look at the Edmonds-Karp algorithm for solving the max-flow problem.
Before we dive into the Edmonds-Karp algorithm, let`s do a quick contrast of the two algorithms.
Now in the Ford-Fulkerson algorithm, the main step is to find augmenting paths which are paths from S to T in the residual graph and to do that, we either use DFS or BFS\@.
Now the running time of Ford-Fulkerson algorithm is order M*C, and recall capital C, is the size of the max-flow.
And recall that this running time analysis assume that the capacities are integer value.
Now in contrast, Edmonds-Karp finds augmenting paths using BFS\@.
So Edmonds-Karp algorithm is an example of a Ford-Fulkerson algorithm.
Does the running time analysis for Ford-Fulkerson algorithms still applies to Edmonds-Karp algorithm when we assume integer capacities? But in fact, we can often make a stronger guarantee on the running time.
We`ll prove that the running time is order m-squared n, and this doesn`t require that the capacities are integer values.
We have no assumptions on the capacities other than, of course, that they`re positive valued.
Now the algorithms are quite similar, so we`ll give a quick recap with a Ford-Fulkerson algorithm and then we`ll point out the differences in the Edmonds-Karp algorithm.
And then afterwards, we`ll dive into the running time analysis for the Edmonds-Karp algorithm.

\subsection{Ford Fulkerson Algorithm}
Let`s start with the Ford-Fulkerson algorithm.
Recall the input to the Ford-Fulkerson algorithm is a flow network which is a directed graph with capacities along the edges specified by C\_sub\_e.
And for the Ford-Fulkerson algorithm, we assume that these capacities are integer values.
Now we start off by setting the flow to zero along every edge of the graph.
Next, we build the residual network for the current flow f.
We denote this residual network by G\_superscript\_f.
Now we look for an augmenting path in the residual network.
More precisely, we check for a path from s to t in the residual network.
And we do this using either BFS or DFS\@.
If such a path exists, that`s denoted by cal P\@.
Now if no such path from s to t in the residual network exists, then we return the current flow.
We`ll prove later that if there is no augmenting path, no path from s to t in the residual network, then the current flow is guaranteed to be a maximum size flow.
Now if there is such a path, then we want to augment along this path as much as possible.
Therefore, we let c(P) be the capacity of this path.
More specifically, it`s the minimum over the edges of this path, of the capacity of these edges in the residual network.
This is the maximum amount that we can augment this path.
Finally, we augment the current flow by c(P) units along this path, and then we repeat the algorithm.
We use the current flow and we build a new residual network, and we check for a path and continue.
And we stop when there`s no st-path in the residual network.
This completes the description of the Ford-Fulkerson algorithm.
Let`s look now at the Edmonds-Karp algorithm.

\subsection{Edmonds-Karp Algorithm}
In the Edmonds-Karp algorithm there`s only one distinction.
When we check for an st-path in the residual network, we have to use BFS\@.
We cannot use DFS\@.
The running time analysis assumes that we`re using BFS to find this path.
That`s the only difference between Edmonds-Karp and Ford-Fulkerson algorithm.
Now, a running time analysis will not assume that the capacities are integer values, so we can remove this assumption.
All that`s required is that the capacities are positive.
Before we dive into the running time analysis, let`s look at one basic property of the Edmonds-Karp algorithm and also the Ford-Fulkerson algorithm.
Notice that the residual network has to change by at least one edge in every round.
In particular, at least one edge reaches full capacity.
We augment along P until we reach the full capacity of at least one edge on that path.
Now, that edge, which reaches its full capacity in the residual graph, will be removed from the residual graph in the next stage.
Now, that edge might be a forward edge or a backward edge, but regardless it will be removed from the residual graph in the next stage.
Now, there may be additional edges which are removed and there might be other edges which are added back into the residual network, and we`re going to need some understanding of which edges are added in or removed from the residual network.
But one key property is that at least one edge is removed in every stage.

\subsection{Proof Outline}
Now we`re going to prove that the running time of Edmonds-Karp algorithm is order m squared n.
Now to prove that, we`re going to prove that the number of rounds of the algorithm, the number of times we augment the path, is at most m times n.
Now in each round, we run BFS\@.
BFS takes linear time.
So assuming the number of edges is at least the number of vertices, that`s order m time, therefore, it`s order m time per round and order mn rounds.
So the total run time will be order m squared n.
So our main task is to prove that the number of rounds is at most mn.
As we just discussed, in every round, the residual graph changes from the previous round.
In particular, at least one edge is deleted from the residual graph in every round.
This edge corresponds to the one with minimum capacity along the augmenting path.
Now, edges make it add it back in in the residual graph.
And in particular, this edge might be deleted in this round and in later rounds might be added back in.
Now what we`re going to prove is that for every edge of the graph, we can bound the number of times that the edge is deleted in the residual graph and then reinsert it back later.
In particular, we`re going to show that the number of times this edge e is deleted from the residual graph, and then later reinserted into the residual graph, is at most n over two times.
Now since there are m edges in the graph, and this holds for every edge of the graph, so we know that every edge is deleted at most n over two times, at least one edge is deleted in every round.
Therefore, there is at most n over two times m total rounds.
This gives our desired bound on the number of rounds of the algorithm which proves the desired running time.
So let`s dive into the proof of this key lemma.

\subsection{BFS Properties}
Now we want to prove the key lemma from the previous slide which states that for every edge of the graph in the residual graph, this edge is deleted and then reinserted later at most n\_over\_two times.
So this deletion, insertion process can occur at most n ove\_ two\_times.
Now of course to prove this lemma, we`re going to have to use properties of the BFS\@.
Let`s recall the setting for BFS, breadth-first search.
The input is a directed or undirected graph G\@.
We`re going to be working with directed graphs.
Our directed graph, the residual network, we`ll have edge weights.
But in fact, BFS doesn`t pay attention to edge weights.
Now BFS also specifies as a start vertex s.
Now what is the output of the breadth-first search algorithm? Well, for every vertex of the graph, so it outputs an array of size n, and dist(v) is defined as the minimum number of edges to go from s to v.
Notice, this just counts the number of edges, it pays no attention to the edge weights.
If we want to pay attention to the edge weights then we have to use an algorithm such as Dijkstra`s algorithm.
Now in addition to computing this distance for every vertex, it also finds such a path from s to v which uses the minimum number of edges.

\subsection{BFS Example}
Let`s take a look at BFS on one of our earlier examples.
Here`s one of our examples of a flow network.
Let`s assume that we`re at the initial flow.
So the flow is zero along every edge of the graph.
Therefore, the capacities along these edges don`t really matter.
Now, since the flow is zero along every edge, there`s no back edges, and the residual network at this stage will be exactly the same flow network.
Now, let`s run BFS on this residual network.
Now, we start off from vertex s, and we explore the edges out of s, and we see a, b and c.
Next, we explore the edges out of a, b and c.
From vertex a, we see d and e.
From vertex b, we see e, which is already explored.
So this is an unexplored edge.
Then from vertex c, we see vertex f.
So the solid red edges are the tree edges, the BFS tree edges, and the dash gray edges are the non-explored edges.
Now we explore the edges out of d, e and f.
From vertex d we see vertex t, our end vertex, and then we have a bunch of unexplored edges.
Now at this stage, BFS is done.
Now, let`s look at some properties of BFS\@.
Notice this is exploring the graph in layers.
Now, in the first layer or level is only vertex s itself.
So this is at level 0.
Now then, the neighbors of s are at level 1.
So a, b and c are level 1.
Then we have that d, e, f are at level 2.
And finally, vertex t is at level 3.
Now the level of vertex is the same as the minimum number of edges to go from s to that vertex.
So formerly, the level of vertex is the same as the distance.
Now, what is the path cal-P that BFS finds from s to t? Well, it`s this path, s to a to d to t.
Notice on this path from s to t, look at the levels.
S is at level zero, a is at level 1, d is at level 2, and t is at level 3.
So the level goes up by plus 1 at every edge.
Now, notice that the level cannot go up by more than one in one edge, because of the definition of the level.
It`s the minimum number of edges from s to that vertex.
Similarly, the level cannot stay the same or go down along an edge.
Otherwise, there`s a shorter path to the vertex.
So the path that BFS finds from s to t, is going to be a path where the level goes up by plus one at every edge.
That`s the key property that we need from BFS, that the path obtained from s to t, the levels of the vertices go up by plus 1 along every edge of their path.

\subsection{BFS Properties - Part 2}
Now, to approve our main demo about the number of rounds of the Edmonds-Karp algorithm, we`re going to analyze the level of a vertex during the course of the Edmonds-Karp algorithm.
Now, in particular we want to look at the vertex Z, and we want to look at how it`s level changes as the residual network changes.
Now, what can happen in the residual network? Well, edges can get deleted.
For instance, suppose we delete this edge from A to D, then what happens to the level of D? Well, now the minimum number of edges to get from S to D is one, two, three.
So, D will be at level three.
So, the level of D will increase when we delete this edge from A to D\@.
Now, we`re also going to add edges into the residual network.
For instance, what if we add an edge from B to D? Or maybe we`ll even add this edge back from A to D, then will the level of D decrease back to level two.
Well, we`re going to prove that that does not occur.
Even though we`re adding edges into the residual network sometimes, the level of vertex is never going to decrease.
We`re going to prove, that for every vertex Z, it`s level over the course of the algorithm does not decrease.
It can stay the same or can increase in some levels.
But if for instance, this vertex D, it`s level increases to three, then it can never go back down to two.
It can only stay the same or increase further.
Now, to prove this claim, we have to understand which edges are added into the residual network, and which edges are removed from the residual network.
So, let`s look at that in more detail.

\subsection{AddDelete Edges}
Let`s take a look in detail how the residual graph changes potentially in a round.
Now, consider an edge of the original flow network.
Now, this edge itself might be in the residual network or its back edge from w to v might be in the residual network.
Let`s look at when the forward or backward edge might be added or removed from the residual network.
When do we add this forward edge to the residual network? Well, in order to not be there before, the flow had to be full.
So if the flow was full, so it means the flow equaled the capacity along that edge, and then the flow was reduced, then this edge would have leftover capacity.
So it would be added into the residual network.
Now, how can the flow be reduced along this edge? That means we send flow along the back edge.
To send flow along the back edge, that means the back edge was in the augmenting path cal-p.
So if this forward edge was added into the residual network, then the back edge must have been on the augmenting path.
Now, let`s look at the other case where we remove this forward edge from the residual network.
Now, we will remove it if there was some leftover capacity, but now there`s no leftover capacity.
So the flow equals the capacity along this edge.
That means we augmented the flow, increased the flow along this edge.
So this forward edge itself is on the augmenting path.
In order to remove this forward edge from v to w from the residual graph, then this forward edge from v to w must be on the augmenting path because we increased the flow along this edge.
Now, the other type of edge that we can add into the residual network is the back edge from w to v.
To add the back edge in, the flow must have been empty and then we increase the flow along the edge from v to w.
To increase the flow from v to w, that means this edge from v to w must be on the augmenting path.
The final case is when we will remove the back edge from the residual network.
Now, in order to remove this back edge from the residual network, there must have been some flow along the forward edge, and then the flow is now empty.
So we must have decreased the flow along this forward edge.
To decrease the flow along the forward edge, that means we send flow along the back edge.
So the back edge from w to v must be on the augmenting path cal-P\@.

\subsection{Conclusion}
Here are the key conclusions that we need from this analysis.
If we add an edge from vertex y to vertex z to the residual network, then what can we conclude? Well that`s this case, and this case.
Notice that in both of these cases, the opposite direction edge is always on the augmenting path.
So if we add this edge from y to z to the residual network, then we know that the edge from z to y is on the augmenting path.
Similarly, if we remove the edge from y to z from the residual graph, then what can we conclude in this case? Well, when we remove, then we notice that the edge itself is on the augmenting path.
So in this case, we know that the edge from y to z is on the augmenting path.
So if we add an edge y to z to the residual graph, then the opposite direction edge is on the augmenting path.
If we remove an edge, then the edge itself is on the augmenting path.
This should make intuitive sense, because to remove an edge from the residual graph, that means we remove the leftover capacity.
In order to remove the leftover capacity, that means we have to increase the flow along this edge.
To increase the flow along the edge, the edge has to be on the augmenting path in order to augment along it.
To add an edge to the residual network, we have to increase the spare capacity.
That means we have to decrease the flow along this edge.
To decrease the flow along an edge, then the reverse direction edge must be augmented.
So it must be on the augmenting path.

\subsection{Proof of Claim}
Now, let`s prove the main claim that we made earlier.
For every vertex Z, the level of that vertex does not decrease during the course of the Edmonds-Karp algorithm.
Now, the residual graph may change.
We may add edges or delete edges from the residual graph, but the level of the vertex never decreases.
How potentially can the level of a vertex decrease? While the level of a vertex might decrease if we add an edge from Y to Z and Y gives a shorter path to Z\@.
Now, suddenly removing edges can decrease the level of a vertex so we can ignore removing edges.
We simply have to look at adding edges and their potential effect.
So suppose the level of Vertex Z is currently I and then suppose that we add this edge from Y to Z into the residual network.
Now, if we add this edge from Y to Z to the residual network, what do we know? Well, as we just saw, then the reverse direction edge must be on the augmenting path.
Now, we have to utilize the properties of the Edmonds-Karp algorithm.
Edmonds-Karp algorithm says this path, this augmenting path is a BFS path.
So we obtained this path by running BFS\@.
Now what do we know about pass from BFS? We know that on this path, the level of the vertices increases by Plus 1 along every edge.
Therefore, the level of Y must be the level of Z plus 1.
We said the level of Z is i so therefore the level of Y is I plus 1.
So we added this edge from Y to Z but this edge goes from a higher level to a lower level.
So it certainly does not decrease the level of vertex Z\@.
It does not give a shorter path to Z\@.
So the level of the vertex Z does not decrease when we add this edge in.
Now, this completes the proof of this claim.
So we prove that the level of a vertex never decreases during the course of the Edmonds-Karp algorithm.
This completes the proof of the claim but in order to bound the number of rounds of the algorithm, we`re going to have to prove something stronger.
We`re going to have to prove that occasionally, the level of a vertex strictly increases and then this will give us a bound on the number of rounds of the algorithm.

\subsection{DeleteAdd Effect}
Now our main lemma bounds the number of times that we can delete an edge and then add it back in.
Let`s look at the effect of deleting an edge and adding it back in on the level of the vertices.
Let`s take a particular vertex v and let`s say the current level of this vertex v is i.
Now suppose at some later round we delete this edge from v to this vertex w from the residual network.
What can we conclude about the level vertex w when we delete this edge from v to w? Now if we were deleting this edge from v to w from the residual network, what do we know about the augmenting path? We know this edge itself from v to w is on the augmenting path.
That means the level(w) is the level(v)+1.
Now at the time that we deleted this edge, the level(v) might have changed from the earlier time, so might not be level i anymore.
But what do we know about the level vertex v? We just prove that the level(v) never decreases.
So this level(v) at this current time must be at least i.
So this is at least i+1.
So the level(w) at this time is at least i+1.
Now suppose later we add this edge from v to w back into the residual network.
Now if we`re adding this edge from v to w into the residual network, then we know that the reverse direction edge, the edge from w to v must be on the augmenting path.
Therefore, at this current time, the level(v) must be equal to the level(w)+1.
By our claim, we know that the level(w) did not decrease from earlier times and we earlier showed that the level(w) was at level at least i+1.
So this is that level at least i+2.
Notice what we just proved.
Suppose the level(v) is i, now we take an edge going out from vertex v, and we delete that edge and later we added it back in.
After we added this edge back in, we know that the level of the vertex is at least two greater than what it previously was.
So the level of this vertex increased by at least two.

\subsection{Finishing Off}
We just showed that if we delete an edge v to w from the residual network and later we add this edge back into the residual network, then the level of this vertex v increase by at least two.
If we look at the level of vertex v before this deletion compared to after the insertion, the level of the vertex increase by at least two.
Now, what do we know about the level of vertices? The minimum level is zero, that`s for vertex S itself.
The maximum level is n.
Therefore, what`s the maximum number of times that we can delete and then add this edge back into the residual network? We can delete it and then reinsert it back in at most n over two times.
Since there m edges in the graph, this proves our main result that there is at most nm rounds in the Edmonds-Karp algorithm.
And that completes the analysis of the Edmonds-Karp algorithm.

