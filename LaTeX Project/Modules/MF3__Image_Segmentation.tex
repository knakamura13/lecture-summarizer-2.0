\section*{Image Segmentation}

\subsection*{Image Segmentation}
Lecture explores max flow use in computer vision for Image Segmentation, aiming to divide an image into objects, specifically separating the foreground from the background, using a given image example to distinguish a triangle from its surroundings.

\subsection*{Formulation}
Examining pixelated images using graphs, with pixels represented as vertices.
Example given with a small magnified image structured by undirected graph G\@.
It has nine vertices, edges connect neighboring pixels.

\subsection*{Parameters}
In the context of an undirected graph G representing pixels, each pixel is associated with two non-negative parameters: fi, representing the likelihood of the pixel being in the foreground, and bi, denoting background likelihood.
Additionally, for each pair of neighboring pixels, a non-negative separation penalty pij is defined, representing the cost of classifying the pixels into different objects.

\subsection*{Example}
In the given image example, parameters were set to define the likelihood of pixels being in the foreground or background, with white pixels more likely foreground and black pixels likely background.
Notably, the matrix defining this is symmetric, allowing for the foreground and background labels to be reversible.
Sample separation penalties are applied when pixels differ, but in this case, there is no penalty for separating pixels into different objects.
The example is simplified, using monochromatic pixels and simple weights for demonstration purposes.

\subsection*{Partition}
Aim to segregate pixel vertices into sets F (foreground) and B (background).
Assign weight to partitions: FB score for foreground pixels, BJ score for background.
Separation penalties subtracted for edges between F and B pixels, with penalty value CIJ\@.
Objective: identify pixel assignment to F and B with the highest cumulative weight.

\subsection*{Min-Cut}
Goal: transform a maximization problem into a min st-cut problem by modifying weights.
Challenge: currently, the problem involves both summation and subtraction of non-negative terms.
Solution: necessary to convert the subtraction of separation penalties into addition to ensure all terms are positive/non-negative, aligning with min st-cut problem where all capacities are positive.

\subsection*{Reformulation}
Sum of foreground (FG) and background (BG) likelihoods across pixels denoted by L\@.
Weight of a partition FB computed using FG weight (fi) per FG pixel and BG weight (bj) per BG pixel.
Weight sum equals L minus FG pixels` BG likelihood sum and BG pixels` FG likelihood sum.
Summing FG and BG pixels` FG likelihood gives total FG likelihood.
Adding BG pixels` BG likelihood yields total BG likelihood.
Upon subtracting edge separation penalty from both sides, equation represents weight W of FB\@.
Define W-prime of FB comprising three terms, showing right-hand side equals L minus W-prime of FB\@.

\subsection*{New Weights}
L represents total pixel likelihoods for foreground (F) and background (B).
Weight of partition F, B is sum of assigned vertices` likelihoods, corrected by a separation penalty for edges.
New weight W` is similar but adds penalty instead of subtracting.
Key relationship: W\_F,B = L - W`\_F,B\@.
Thus, maximization of W is equivalent to minimization of W`, turning our problem into a min instead of a max problem.
All W` terms are positive.
This converts the problem into a mean SD cut, solvable by max flow in flow networks--maximize flow to find mean SD cut solution.

\subsection*{New Problem}
New problem formulation involves an undirected graph representing an image.
Each vertex I (pixel) has associated non-neg\. weights fi and bi.
Edges have non-neg\. weight Pij.
Aim to partition vertices into two sets, F and B, minimizing weight w` of FB, which also maximizes weight w of FB, thus addressing the original image segmentation problem.
Next step: reduce this to a max flow problem.

\subsection*{Flow Network}
Converting image segmentation to max-flow involves redefining the problem`s undirected graph as a directed graph G-prime.
In this conversion, each undirected edge between neighboring pixels is replaced with bi-directional edges representing flows in both directions.
Capacity constraints on these edges are dictated by separation penalties, labeled as Pij, reflecting the cost of segmenting adjacent pixels.
Additionally, the graph incorporates specific vertices representing foreground and background, integrating pixel likelihoods relative to these concepts into the network structure.
This setup prepares the image segmentation issue for resolution through max-flow algorithms.

\subsection*{Foreground}
Adding vertex S to represent foreground pixels in a directed graph.
Shrink vertices for clarity.
Introduce vertices S (source) and t (sink) for max-flow problem: S for foreground, t for background.
Connect S to each pixel vertex with edge capacity fi to encode foreground likelihoods.

\subsection*{Background}
Constructed a flow network from an undirected graph to solve max-flow problem.
Each edge leads from original graph vertices to a new vertex `t`, with `t` as a sink and another vertex `s` as a source.
The capacity of edges to `t` represents the background likelihood, while the capacities from `s` reflect the foreground likelihood; separation penalties are also included.
With capacities defined for every edge, the graph is ready for the max-flow problem input.

\subsection*{Cuts}
Executing max-flow on a flow network yields a flow of max size, equivalent to the min capacity S,T cut.
Understanding S,T cuts involves examining partitions and cut capacities, focusing on edges from set F (foreground) to set B (background), excluding edges from B to F\@.
Assigning pixels to F yields edges to sink T (capacity B,I), assigning pixels to B gives edges from source S (capacity F,J).
For each undirected edge with ends in F and B, its directed edge`s capacity is P,I,J\@.
Summing capacities of edges crossing the cut equals W prime (F,B), the sum of edge capacities from S to B, I to T, and separation penalty P,I,J for F and B assignments.
The goal: minimize W prime.
Running max-flow and finding the min S,T cut achieves this, identifying the partition that minimizes W prime.

\subsection*{Solution}
Solved img segmentation by mapping it to a max flow problem in a network.
Converted img to undirected graph, added source (s) and sink (t) vertices, and turned edges bi-directional for flow network.
Defined capacities, ran max flow to get max-size flow f*.
Max flow size equals min st-cut capacity per max flow min-cut theorem.
Min-capacity cut corresponds to min-weight partition w prime, inverse to original max-weight partition goal w.
Hence, finding min w prime yields max w partition, solving img segmentation via max flow.

