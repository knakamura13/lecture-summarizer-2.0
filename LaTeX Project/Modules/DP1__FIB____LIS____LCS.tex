\section*{FIB    LIS    LCS}

\subsection*{Dynamic Programming Overview}
Dynamic programming (DP) is highly useful and a favorite teaching subject, yet initially challenging for students.
Mastery requires practice, well beyond the provided homework.
It`s crucial to tackle additional problems from various sources.
Starting with calculating Fibonacci numbers as a simple DP example, the course will cover a range of DP problems, including Longest Increasing Subsequence (LIS), Longest Common Subsequence (LCS), the Knapsack Problem, Chain Matrix Multiplication, and shortest path algorithms using DP techniques.

\subsection*{Fibonacci Numbers}
Exploring an algorithm for generating the nth Fibonacci number with a focus on dynamic programming concepts and techniques.
The Fibonacci sequence is characterized by each number being the sum of the two preceding ones, starting with 0 and 1.
Looking to design an efficient algorithm that has a polynomial running time in relation to n, since a basic recursive formula defines Fibonacci numbers.
Will also examine the limitations of a simple recursive approach before delving into more advanced examples of dynamic programming.
The intention is to input a non-negative integer n and output the nth Fibonacci number effectively.

\subsection*{FIB1  Recursive Algorithm}
The natural recursive algorithm for computing the Nth Fibonacci number, Fib1, sums the previous two numbers in the sequence.
It accounts for base cases where the first two numbers are 0 and 1.
The running time T(n) includes constant time for the base cases and the sum of T(n-1) and T(n-2) for recursive calls.
This running time is at least as large as the Nth Fibonacci number, which grows exponentially with the constant phi (\textasciitilde{}1.618), the golden ratio.
Due to the exponential growth of Fibonacci numbers, the running time for computing the Nth Fibonacci is also exponential in n, making Fib1 highly inefficient.
The need exists for a more efficient algorithm due to this poor performance.

\subsection*{FIB1  Exp.
Running Time}
Algorithm computes nth Fibonacci number using inefficient recursion by recalculating subproblems multiple times, particularly smaller Fibonacci numbers at deeper recursion levels exponentially.
Efficiency improved by starting with smaller subproblems using an array F to store Fibonacci numbers; F(i) represents ith Fibonacci number.
Initiates with first two Fibonacci numbers, iteratively computes subsequent numbers using recursive formula.
Method transforms algorithm from top-down recursion to bottom-up dynamic programming, systematically building up the solution and avoiding redundant calculations.
This dynamic approach ensures calculation of nth Fibonacci number is efficient.

\subsection*{FIB2  DP Algorithm}
Developed dynamic programming algo to compute nth Fibonacci number.
Algorithm uses an array F to store Fibonacci sequence, with base cases F\[0\]=0 and F\[1\]=1.
Utilizes for loop from 2 to n, with each Fibonacci number being the sum of previous two stored values.
Essential feature: no recursion used, despite recursive nature of Fibonacci formula.
Running time analysis shows base cases require constant time, and the for loop iterates n times with a constant time operation within, resulting in a linear O(n) total run time for the algorithm.
Algorithm showcases dynamic programming technique.

\subsection*{FIB2  DP Recap}
Dynamic Programming (DP) algorithms are designed based on the recursive nature of problems but are implemented without actual recursion to enhance performance and simplify runtime analysis.
In contrast, memoization employs hash tables to remember solved sub-problems, which we avoid in this course to focus on pure DP techniques and avoid confusion.
While some prefer memoization for its elegance, DP is acclaimed for its efficiency, reduced overhead, and simpler runtime analysis.
Mastery of DP requires ample practice, as initially, it appears complex to students.
Repetitive exercises in and out of class--from textbooks and online resources--are essential for students to internalize the DP approach, eventually finding it intuitive and straightforward to devise DP solutions.
Consistent practice is emphasized as the key to excelling in the subject.

\subsection*{Longest Increasing Subseq.}
Exploring the dynamic programming approach to solve the Longest Increasing Subsequence (LAS) problem, which involves determining the maximum length of an increasing sequence of numbers within a given list of N numbers (A1 to AN).
The focus is on calculating the sequence`s length, not the sequence itself, which simplifies the process of devising an algorithm to identify the sequence later.
Unlike substrings that comprise consecutive elements from a larger string, subsequences are non-consecutive ordered selections from the string, with increasing subsequences defined by each element being strictly larger than its predecessor.
For example, within a 12-number sequence, the longest increasing subsequence identified has a length of 6, composed of the following numbers: -3, 1, 4, 5, 8, 9.
Hence, the developed algorithm should output the length `6` for this instance.
This sets the stage for creating a dynamic programming solution to efficiently compute the length of the longest increasing subsequences in such numerical inputs.

\subsection*{LIS  Subproblem Attempt 1}
Designing a dynamic programming (DP) algorithm involves a standard recipe.
First, define sub-problems; for Fibonacci numbers, this means identifying the i-th number in the sequence (F(i)).
Next, establish a recursive relation connecting the solution of a sub-problem to smaller instances; F(i) is the sum of F(i-1) and F(i-2).
Applying this to the longest increasing subsequence problem, define L(i) as the length of the longest increasing subsequence within the first i elements.
The challenge then lies in expressing L(i) in terms of L(1) to L(i-1).
Understanding how to construct this relationship is key for successfully applying DP to a given problem.

\subsection*{LIS  Recurrence Attempt 1}
To solve for L(I), the length of the longest increasing subsequence (LIS) ending with the I-th element, formulate subproblems based on earlier elements.
For example, an array of n=12 elements, initially, L(1) is an LIS of length 1, since only one element is considered.
As elements are added, track the LIS where L(2) is length 2 for the array \[5, 7\], and maintaining such sequences continues with each subsequent element.
Challenges arise when elements cannot simply be appended, exemplified at L(9), where the current LIS does not accommodate the new element `A`, but an alternative sequence does, resulting in a longer LIS\@.
Instead of tracking only the current LIS, track the smallest ending element of the LIS for more opportunities to append new elements and extend the LIS\@.
Implement this by maintaining the LIS for each potential ending character, always seeking the minimum ending character.
As the ending characters are limited to preceding array elements, there`s a finite set to consider.
A new subproblem formulation includes computing L(I) by maintaining the LIS ending at each element A(I) of the array, providing a basis to compute the LIS when a new element is considered.
To compute L(9), you then evaluate all potential sequences ending with preceding characters to find where `A` can be appended, creating the longest increasing sequence ending at `A`.
This refined approach allows the calculation of L(I) by determining the feasibility of appending new characters onto existing sequences and optimizing the LIS considering all subproblems up to L(I-1).

\subsection*{LIS  Subproblem Attempt 2}
Introduced new subprob formula involving L(i) denoting longest inc subseq length incl a\_i.
Key addition: a\_i must be in subseqs considered.
Can express L(i) recurrence using L\_1 to L\_(i-1).
Example given: L(1) = 1 (single element, 5), L(2) = 2 (5, 7 subseq), diverges at L(3) with length 1 (only 4), L(4) = 1 (only -3), L(5) = 3 (5, 7, 9), L(6) = 2 (-3, 1), L(7) = 4 (5, 7, 9, 10).
For complex L(10), only subseqs that end with elemnts a\_i can append 8; select longest, add 1 for 8.
Hence L(10) = 5, illustrates recurrence relation to solve L(i), relating back to previous L values.

\subsection*{LIS  Recurrence Attempt 2}
Defining longest increasing subsequence(length LIS) L(i) which includes ai, get 1 for ai, and add max length LIS before ai where elements are smaller than ai.
Only consider earlier indices j where aj \textless{} ai for appending ai to form L(i) = L(j) + 1\. i must be after j in the array.
With the recurrence relation for L(i) established, we can describe a dynamic programming (DP) algorithm to solve the problem, which will include pseudocode and runtime analysis.

\subsection*{LIS  DP Algorithm}
Develop pseudocode for the longest increasing subsequence (LIS) dynamic programming algorithm, using previously defined recurrence.
Input sequence is a1 to an; use array L to store solutions.
Fill L bottom-up, i from 1 to n.
Initialize L of i to 1.
Nested for loop where j ranges from 1 to i-1, checks if aj \textless{} ai and if appending ai to L of j yields a longer subsequence than current.
If so, update L of i to this new max.
After constructing L, algorithm`s output isn`t last entry (unlike with Fibonacci sequence), which only gives LIS ending at an.
Instead, scan L to find max value, indicating the longest subsequence length, independent of ending position.
Maintain max index in loop, and return the value at max index.
Algorithm`s efficiency depends on running time analysis.

\subsection*{LIS  DP Running Time Solution}
The algorithm`s runtime is dominated by a set of nested for loops, iterating over a maximum of n elements each, paired with an Order(1) conditional statement, which collectively execute in Order(n\textasciicircum{}2) time.
An additional non-nested for loop searches for a maximum value within the table and operates in Order(n) time.
Consequently, the overall runtime complexity of the algorithm is determined to be Order(n\textasciicircum{}2).

\subsection*{LIS Recap}
Completed the development and time complexity analysis of a dynamic programming (DP) algorithm.
Initially defined the sub-problem to find the longest increasing subsequence (LIS) length of an array prefix.
Unable to find a suitable recurrence relation with the initial sub-problem, reformulated it by adding an extra condition--this mimicked strengthening an inductive hypothesis in a mathematical proof by adding conditions, facilitating the formulation of a recurrence for the sub-problems.
The revised sub-problem included the array element itself.
The algorithm solves this strengthened version, determining the LIS with a specific ending character.
Solving this facilitates addressing the original issue--finding the LIS irrespective of the ending character--by drawing parallels with induction proof techniques.
This approach exemplifies common strategies in DP algorithm design and problem-solving.

\subsection*{Longest Common Subsequence}
LCS (Longest Common Subsequence) dynamic programming example to find longest string length appearing as subsequence in 2 equal length strings X and Y\@.
Focus on subsequence, not substring, with end goal to determine string length, later leading to identifying the actual subsequence.

\subsection*{LCS Example Question}
Examining an example to clarify terminology involving two strings of length seven.
Task entails identifying the longest common subsequence within these strings, marking it, and noting its length.

\subsection*{LCS Example Solution}
Found substring `BCBA` (len 4) illustrating dynamic programming (DP) for Unix diff utility.
DP solves longest common subsequence prob essential for file comparisons in Unix diff, showcasing practical application and relevance in computing tasks.

\subsection*{LCS  Subproblem Attempt 1}
Design dynamic programming algorithms: 1) Define subproblem: Choose a prefix of input, set L(i) as length of the longest common subsequence (LCS) in first i chars of X and Y\@.
2) Define recurrence: Express L(i) using L(1) to L(i-1), determining LCS by breaking problem down into smaller instances.

\subsection*{LCS  Recurrence Attempt 1}
Defining a subproblem in the context of finding the longest common subsequence (LCS) between two prefixes of strings X and Y of equal length i, the author explains a recursive approach for computing the LCS length, not the subsequence itself.
They introduce a function l(i), which represents this length for the prefixes.
To develop a recursive relation for l(i), the author examines the last characters of prefixes X\_i and Y\_i.
Two scenarios are considered: when these characters are the same and when they differ.
For the scenario where X\_i equals Y\_i, it`s established that the LCS must end with this common character and thus, l(i) relates directly to l(i-1) plus one for the added common character.
The second scenario, where X\_i is not equal to Y\_i, is suggested but not elaborated, indicating that the process will differ from the first case.

\subsection*{LCS  Recurrence Problem}
Exploring LCS (longest common subsequence) in sequences X and Y when last chars differ, with important observations: if last char of LCS is in X (Xi  Yi), then Yi isn`t included, vice versa if last char of LCS is in Y\@.
LCS with both chars dropped reverts to subsequence of length (i-1) without increment.
In other cases with just Xi or Yi dropped, no ready solution as prefix lengths differ, thus solution not in table.
Current subproblem definition inadequate because it relies on equal-length prefixes.
Key insight: subproblem must permit prefixes of differing lengths.
Thus, redefining subproblems with two parameters i and j to represent lengths of X and Y prefixes respectively, resulting in a 2D table where L(i,j) gives LCS length of X1:Xi with Y1:Yj, allowing exploration of all i, j possibilities.

\subsection*{LCS  Subproblem Attempt 2}
Revised subproblem for sequences involves using two indices, i and j, to account for different prefix lengths in sequences x and y.
A two-dimensional table will be used, where i and j vary from 0 to n, to represent the length of the longest common subsequence (LCS) between prefixes xi and yj.
L(i, j) defines the LCS length for the first i characters of x and the first j characters of y.
Base cases are set for i=0 or j=0, where LCS length is 0 because one sequence is empty.
These foundations pave the way for defining a recurrence relation for the LCS problem.

\subsection*{LCS  Recurrence Unequal Case}
Developing a recurrence relation for finding the length of the longest common subsequence (LCS) between two sequences, X and Y, with lengths I and J, respectively.
Different lengths of X and Y are considered.
When the last characters, X\_i and Y\_j, differ, LCS could end with either one of these characters or with neither.
The LCS length, denoted as L(i,j), then has two scenarios: 1) If last character of X isn`t included, L(i,j) = L(i-1,j), indicating the LCS without last char of X\@.
2) If last character of Y isn`t included, L(i,j) = L(i,j-1), indicating the LCS without last char of Y\@.
Select the greater length between these two scenarios for the optimal solution.
This provides the recurrence relation for differing characters: L(i,j) = max(L(i-1,j), L(i,j-1)).
This rule applies when the compared characters at the ends of the sequences do not match, providing a method for constructing the length of the LCS iteratively by comparing prefixes of increasing length.

\subsection*{LCS  Recurrence Equal Case}
When developing a recurrence relation for prefixes with different ending characters, we examine a modification when these characters are identical (both end in `A`).
With X\_i = Y\_j, there are three options: omit X\_i (and Y\_j since they match), or conclude the optimal sequence with the shared character.
For the first two options, the subproblem solution lengths are L(i,j) = L(i-1,j) or L(i,j-1) respectively.
The unique third option, possible only when characters match, calculates L(i,j) = 1 + L(i-1,j-1), appending the common character to the optimal solution of the subproblem, determining the length of the longest common subsequence for the new problem.

\subsection*{LCS  Recurrence Equal Recap}
Recurrence for X\_i equals Y\_j involves three options: drop X\_i (L of i-1, j), drop Y\_j (L of i, j-1), or include both (gain +1, then L of i-1, j-1).
However, only need to consider third option of appending common character and taking L of i-1, j-1 due to optimality.
Adding unmatched characters to an optimal solution contradicts its optimality, because it would be shorter.
If X\_i matches an earlier Y, any subsequences in the smaller prefix of Y are also in the larger one, so later matches with X\_i are equivalent or better, simplifying to just the last scenario.
This insight simplifies the recurrence and completes its definition.

\subsection*{LCS  Recurrence Summary}
Recurrence for non-empty strings I, J where I, J \textgreater{}= 1.
Case analysis: X\_I  Y\_J (characters differ), select the larger length from dropping last char from X or Y\@.
If X\_I = Y\_J (chars match), add 1 and take optimal solution for (I-1, J-1).
Base cases, when one string is empty yields length 0.
Dynamic programming builds 2D table L row-wise for optimal solution lengths.
Cells depend on three computed neighbors: diagonal (L\_I-1\_J-1), above (L\_I\_J-1), or left (L\_I-1\_J).
Pseudocode adheres to populating 2D array L using these dependencies to build solutions incrementally.

\subsection*{LCS  DP Algorithm Question}
Developed a dynamic programming algorithm to solve for longest common subsequence (LCS) given two input strings, X and Y\@.
Initialized the top row and the first column of the LCS matrix with zeros, representing base cases.
The matrix is filled row by row, considering two main cases: matching last characters of X and Y leading to a recurrence that adds 1 to the value of the subproblem for strings X and Y minus their last characters, or differing last characters, leading to a recurrence that takes the maximum value between dropping the last character from either X or Y\@.
Final result, the LCS length, is found in the matrix`s bottom right entry, L(n, n), which represents the LCS for the entire strings X and Y\@.

\subsection*{LCS  DP Algorithm Solution}
Algorithm runs in O(n\textasciicircum{}2) time due to nested for loops iterating over an order n seq.
Base cases, with O(1) steps within an O(n) loop, contribute less to total time.
Solving the longest comm subseq problem w/ dynamic programming req a 2D tbl to handle prefixes of x \& y of diff lengths.

\subsection*{DP1  Practice Problems}
Recommended practice with dynamic programming (DP) problems, highlighting text from `Algorithms` by Dasgupta et al.
Problems identified for practice include: finding a max sum contiguous sequence (6.1), planning a trip with minimal penalty hotel stops (6.2), a problem named `yuck Donald`s` (6.3), segmenting a string into words (6.4), and finding the longest common subsequence (6.11), with a suggestion to try its variant involving substrings.
Also outlines steps for creating a DP algorithm: (1) define subproblems, essentially smaller versions of the original problem applied to input prefixes; (2) create a recurrence relation to express a subproblem`s solution in terms of previously solved subproblems, possibly modifying (`strengthening`) subproblems to include an additional element constraint.
Final solution may require searching entire table for optimal result rather than using last table element directly.
Advises solving problems independently before seeking further instruction for maximized learning.

\subsection*{DP1  Practice Problem 6.1}
The problem addresses finding the substring (contiguous subsequence) with the maximum sum within a sequence of numbers a\_1 through a\_n.
To solve this, it is essential to define a subproblem with the prefix of length i and establish a function, s(i), representing the maximum sum obtainable from a substring within this prefix.
A recursive relationship is sought to express s(i) in terms of previous subproblems (s(1) to s(i-1)).
The challenge arises in determining whether to include the number a\_i in the optimal subsequence sum s(i-1), which depends on whether the subsequence includes a\_(i-1) to maintain contiguity.
This difficulty implies that the current subproblem definition is inadequate, as it doesn`t provide information about the endpoint of the subsequence.
Hence, a refinement of the definition is necessary to account for the endpoint, ensuring that s(i) can be expressed in terms of smaller subproblems while preserving the requirement of contiguity.

\subsection*{DP1  Practice Solution}
Solved problem 6.1 by defining s(i) as the maximum sum obtainable from a substring ending with a\_i, from a\_1 to a\_i with the inclusion of a\_i.
Base case s(0)=0 represents an empty substring.
For s(i) where i \textgreater{}= 1, the recurrence is the maximum of either a\_i alone or a\_i plus s(i-1), depending on if s(i-1) is positive.
Final algorithm output is the maximum value across all s(i), not just s(n), since the maximum sum substring may end before a\_n.
The table is filled iteratively from i=0 to n, and the running time of the algorithm is O(n), with each entry requiring constant time and there being n entries.

