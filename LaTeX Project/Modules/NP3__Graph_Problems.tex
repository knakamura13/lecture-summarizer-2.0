\section*{Graph Problems}

\subsection*{Lecture Outline}
Demonstrated 3SAT as NP-complete, explored graph theory`s Independent sets, clique problem, and vertex cover problem, and established their NP-completeness.
Definitions provided for unfamiliar combinatorial issues.

\subsection*{Independent Set}
Independent set defined as subset of vertices in undirected graph G with no edges among them.
The check involves ensuring no vertices in subset are adjacent.
Finding large independent sets is challenging, whereas identifying small or trivial ones, like an empty set or single vertex, is simple.
Example provided with a graph where a set of three vertices represents an independent set, illustrating the concept.

\subsection*{Quiz  Max Independent Set Question}
Max independent set problem confirmed to be in class NP\@.
P=NP scenario could alter this classification.
Standard for true/false questions adjusted to include `known to be` for clarity on such matters.

\subsection*{Quiz  Max Independent Set Solution}
The max independence set problem isn`t in the NP class because, unless P=NP, there`s no polytime method to verify a claimed max size solution`s validity.
Similarly, the max value knapsack problem isn`t in NP, but its search version can be fixed to be in NP\@.

\subsection*{Search Version}
Input: undirected graph and goal g.
Output: independent set S (size \textgreater{}= g) if possible, else NO\@.
The problem is a search problem in NP class.
Independent set problem is NP-complete.

\subsection*{Proof Outline}
The independent set problem is NP-complete, demonstrated through two steps.
First, the problem is in NP because solutions can be verified in polynomial time.
This is done by checking, in O(n\textasciicircum{}2) time, non-adjacency for vertex pairs in a proposed solution set S and confirming, in O(n) time, whether S meets the desired size g.
Second, to establish NP-hardness, a polynomial-time reduction from a known NP-complete problem such as 3SAT to the independent set problem is shown.
3SAT is chosen for its simplicity, with clauses limited to three variables compared to SAT`s potentially longer clauses.
By accomplishing this reduction, the NP-completeness of the independent set problem is proven.

\subsection*{3SAT  IS}
Input from a 3-SAT problem with a formula F containing N variables (X1-Xn) and M clauses (C1-CM), where each clause has up to 3 literals, is translated into an Independent Set problem by constructing a graph G\@.
Set a target `g` equal to M (the number of clauses) for the Independent Set problem.
Construct graph G by making a vertex for each literal in a clause, leading to at most 3M vertices since there are M clauses.
Edges are added within clauses to encode the literal relationships and between clauses to ensure consistent variable assignments across the graph.

\subsection*{Clause Edges}
Two edge types in graphs represent clauses and variables.
Clause edges encode clauses by linking literals in complete subgraphs--a triangle for three literals, an edge for two, and a singleton for one.
An independent set S in such a graph can contain only one vertex per clause, representing the satisfied literal for that clause.
Achieving an independent set of size m aligns with finding a satisfying assignment for a formula, as exactly one literal per clause must be satisfied.
However, a conflict arises when an independent set includes complementary literals, like x\_1 and x\_1 bar, yielding an assignment where x\_1 is both true and false--a contradiction.
Thus, independent sets must be constructed to avoid simultaneously assigning both true and false to the same variable, ensuring they represent valid assignments.

\subsection*{Variable Edges}
Variable edges concept introduced to prevent including contradictory vertices (representing X\_i and X\_i-bar) in an independent set.
This ensures valid assignments in independent sets.
Clause edges guarantee that these assignments satisfy all clauses.
A specific example demonstrates this approach.

\subsection*{Example}
Constructed a graph from a given 3-SAT formula with 4 variables (X, Y, W, Z), creating vertices for each literal in the formula`s clauses and adding edges to form triangles for clauses with three literals and a single edge for clauses with two.
Variable edges connect each variable to its negated counterpart.
Demonstrated that an independent set in the graph (no adjacent vertices) of size four can reflect a satisfying variable assignment for the formula (e.g., X=false, Y=false, W=false), with each chosen vertex in the independent set corresponding to a literal in the formula that reflects a variable`s value.
Posited a general proof that there`s a direct correspondence between independent sets of size m in the graph and satisfying assignments of the formula, meaning each independent set represents a combination of variables that make the original formula true.

\subsection*{Correctness}
Proving a satisfying assignment for three set input F equates to an independent set of size at least g in a constructed graph.
Forward direction proved by starting with a satisfying assignment for F leading to an independent set of size g.
For each clause in F, one satisfied literal chosen, corresponding to a graph vertex, added to set S\@.
Size of S is m (the number of clauses), matching goal size g, thus achieving required size.
S only contains one vertex per clause, does not contain contradictory variables (X\_i and X\_i bar simultaneously) based on the assignment`s truth values.
No clause edges or variable edges in S, meaning it`s edge-free.
S is confirmed as an independent set, fulfilling the forward direction proof by creating an independent set size g from a satisfying F assignment.
The reverse direction remains to be shown.

\subsection*{Reverse Implication}
The text delivers a proof demonstrating the reverse implication in the context of computational complexity theory, specifically establishing the NP-completeness of the Independent Set problem via a reduction from the 3SAT problem.
It is explained that an independent set of size M implies a satisfying assignment for the 3SAT instance since each clause (triangle) of the graph has a vertex (literal) which can be set to true without contradiction--a result of not allowing both a literal and its negation (Xi and Xi bar) within the independent set due to added edges between them, thus avoiding false assignments.
The proof confirms that the satisfaction of all clauses in the 3SAT formula corresponds to a valid assignment derived from the independent set of size at least M, solidifying the reduction`s correctness.
The absence of an independent set of size M, in turn, entails no satisfying assignment for the formula.
This equivalence substantiates that the Independent Set problem is indeed NP-complete.

\subsection*{NP-hard}
The independent set problem is NP-complete, confirmed through a reduction from all problems in NP class to it, indicating its equal or greater difficulty.
Max Independent Set Problem, its optimization version, is not confirmed to be in NP but is reducible from the independent set problem, signifying it`s at least as hard as any NP problem.
If Max Independent Set were in NP, it`d be NP-complete, but its classification is currently NP-hard.
This reflects its comparative difficulty level: NP-hard indicates that solving it would enable solving all NP problems in polynomial time.
NP-complete problems are the most challenging in NP, being both NP-hard and inside the NP class.

\subsection*{Lecture Outline II}
Demonstrated that the independent set problem is NP-complete through reduction from SAT, a logic problem.
Proceeding to establish NP-completeness for Clique and Vertex Cover problems, leveraging the transformation from graph-based independent sets, simplifying the process.

\subsection*{Clique}
Clique defined as fully conn\. subgraph in undir\. graphs.
Any vertex pair (X, Y) in subset S must share edge to form clique.
Given graph has largest clique of size 5, all poss\. vertex pairs connected.
Finding large cliques is hard since small ones (empty set, single vertex, any two connected vertices) are easy to identify.
The real challenge is in identifying the max\. size clique.

\subsection*{Clique  Search Version}
The Clique problem involves finding a subset of vertices S in an undirected graph G that forms a clique of at least size g, aiming for the largest possible clique.
Unlike a trivial task of finding at most g size (where an empty set could suffice), the challenge is to determine if such a clique exists; if it does, output the clique, if not, output NO\@.
The focus is on proving the Clique problem`s NP-completeness.

\subsection*{Clique  Proof Outline}
The Clique problem is established as NP-complete by first demonstrating that it belongs to the NP class through polynomial-time verification of proposed solutions.
Validation involves checking if a subset of vertices, S, in a given graph G forms a clique of size at least g, by assessing whether each vertex pair is connected--an operation that takes O(n\textasciicircum{}3) time with a simple algorithm but can be reduced to O(n\textasciicircum{}2).
The second proof element shows that the Clique problem is at least as hard as any NP problem by reducing an NP-complete problem to it.
The similar graph-based Independent set problem, which is NP-complete, is selected for this reduction, thereby confirming the Clique problem`s NP-completeness.

\subsection*{IS Clique Idea}
Independent sets and cliques in graphs are fundamentally opposite concepts--an independent set lacks edges between its vertices, while a clique is a subset with edges connecting all pairs of vertices.
To utilize a clique-solving algorithm for independent set problems, one can invert a given graph G to its complement G bar, which retains the same vertices but includes an edge between vertex pair X,Y only if such an edge is absent in G\@.
Consequently, a set S is an independent set in G if and only if S is a clique in G bar.
This relationship is intuitive and requires no formal proof; the absence of edges among vertices in S in graph G ensures that in G bar, all vertices in S are interlinked, qualifying S as a clique.
Conversely, if S is a clique in G bar, it must be an independent set in the original graph G\@.

\subsection*{IS Clique}
Demonstrated reduction from independent set problem to clique problem, utilizing graph G and goal g.
Reduction involves using G`s complement G bar and g as inputs for the clique problem.
A solution S to the clique problem in G bar signifies S is an independent set in G\@.
Consequently, verifying that the clique problem remains NP-complete hinges on this reasoning.
If no clique of size g exists in G bar, there`s no independent set of size g in G\@.
The correctness of this reduction is confirmed through these observations, hence establishing the NP-completeness of the clique problem.

\subsection*{Lecture Outline III}
Demonstrated clique as NP-complete; aim to establish vertex cover also as NP-complete.

\subsection*{Vertex Cover}
A vertex cover in an undirected graph G consists of a subset of vertices that covers every edge, meaning for each edge, at least one of its endpoints is included in the set.
This set can be as large as including all vertices, ensuring all edges are covered.
For example, if an edge is denoted by (x, y), then either x, y, or both must be in the vertex cover set S\@.
In a given example graph, the red vertices form such a cover.

\subsection*{VC  Search Version}
Vertex cover problem involves an undirected graph G and aims to find the minimum size vertex cover S within a budget b.
If a vertex cover smaller or equal to b exists, it`s identified, otherwise, the output is NO\@.
This problem is proven to be NP-complete.

\subsection*{VC  Proof Outline}
Vertex cover problem is NP-complete; proved in two steps.
First, vertex cover, S, verification in polynomial time: checking each edge to confirm at least one endpoint in S (done in O(n+m)) and S size validation (done in O(n)).
These show S as a potential solution in polynomial time.
Second, must demonstrate vertex cover problem`s equivalency to other NP-complete problems.
Chose independent set for reduction, being a graph problem similar to clique.
The two-stage proof confirms the NP-completeness of the vertex cover problem.

\subsection*{VC  Reduction Idea}
Red vertices represent a vertex cover in a graph, while white vertices are an independent set.
Evidence suggests that a minimum vertex cover corresponds to an independent set, with a general claim that a vertex set S is a cover if its complement S bar is independent, implying a one-to-one correlation between vertex covers and independent sets.
Proving this claim facilitates a reduction from the problem of finding an independent set to finding a vertex cover.

\subsection*{Forward Implication}
Vertex cover S implies that its complement, S bar, is an independent set.
Demonstrated this by showing that in graph edges, at least one endpoint is always part of the vertex cover S, ensuring that the complement S bar consists of non-adjacent vertices.
Since S bar doesn`t include both vertices of any edge, it qualifies as an independent set, as no two vertices within it share an edge.

\subsection*{Reverse Implication}
Proving reverse implication that if S-bar is an independent set, then S is a vertex cover.
Since no edge lies entirely within independent set S-bar, at least one endpoint of each edge must be outside S-bar, meaning it`s in S\@.
Hence, S covers all edges, proving it`s a vertex cover.
Claim proven, ready for reduction.

\subsection*{IS VC}
Reduced Indep.
Set prob\. to Vertex Cover prob\. by setting budget b=n-g in graph G\@.
Correctness rests on claim: If G has a Vertex Cover (VC) size \textless{}= N-g=b, then comp\. set S is an ind\. set of size \textgreater{}= g.
So, find VC in G with size b ensures Ind.
Set size g.
No VC of size b implies no Ind.
Set of size g.

\subsection*{IS VC Correctness}
Reduced independence set problem (ISP) to vertex cover problem (VCP) using graph G and goals--ISP goal as g to VCP budget b, where b = n - g.
Solved VCP with solution s \textless{}= b.
Complementary set S-bar returned as ISP solution for s as a vertex cover, implying S-bar is an independent set \textgreater{}= g.
No VCP solution means no ISP solution, confirming the reduction`s correctness.
Proved VCP is NP-complete.

\subsection*{Practice Problems}
Chapter Eight`s practice problems are key to understanding NP-completeness, with favorites including problem 8.4, which challenges spotting errors in an NP-completeness proof, and 8.10 requiring proofs of NP-completeness through generalization, comparing with known problems.
Problem 8.14 tackles the Clique plus Independent set issue, while 8.19 deals with the Kite problem.
Proving NP-completeness involves two steps: demonstrating it`s in NP (often straightforward), and reducing a known NP-complete problem to the new problem, where similarity guides selection of the known problem.
Two reduction methods exist: generalization -- setting parameters to transform the new problem into a known one -- and gadget construction, where inputs are modified with additional structures.
Mastery of NP-completeness, like dynamic programming, is achieved through extensive practice.

