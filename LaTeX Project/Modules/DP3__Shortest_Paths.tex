\section*{Shortest Paths}

\subsection*{Shortest Paths via DP}
Exploring shortest path problems in directed graphs using dynamic programming, considering graphs represented as G with weighted edges W(e).
Directed graphs are used to model undirected ones by replacing each edge with anti-parallel edges.
The focus is on calculating the shortest path from a start vertex S to all other vertices, formulated as a distance array, dist(z), for every vertex Z, where dist(z) is the length of the shortest path from S to Z\@.
Dijkstra`s algorithm is traditionally used to solve this problem.
It operates similarly to BFS but requires a min heap or priority queue due to edge weights, making operations cost log(n) time, leading to a total runtime of (n+m) log(n).
However, Dijkstra`s algorithm is limited to graphs with positive edge weights, failing when negative edges are present since it doesn`t reassess distances once computed.
For graphs with negative edge weights, dynamic programming provides a solution to the shortest path problem as Dijkstra`s algorithm falls short in these cases.
The goal is to develop an algorithm that can handle negative edge weights effectively.

\subsection*{Negative Weight Cycles}
Directed graph with edge weights that can be positive or negative.
Need to find shortest path from start vertex S to all other vertices.
Problem`s complexity arises due to negative weight edges.
Example shows changing an edge weight from -2 to -6 can create a negative cycle (B-\textgreater{}A-\textgreater{}E-\textgreater{}B) where each iteration reduces path length by -1, allowing for an infinitely shorter path.
In graphs with negative cycles, shortest path problem becomes undefined since path length can decrease infinitely.
Alternately, if no negative cycle is reachable from S, the problem is well-defined, and shortest-path distances to all vertices can be computed.
Goal: either find a reachable negative cycle or, if none exists, find shortest path lengths from S to all vertices using dynamic programming.

\subsection*{Single Source  Subproblem}
Designing a dynamic programming (DP) algorithm for the single-source shortest path problem in a directed graph with arbitrary edge weights, positive and negative, but no negative weight cycles.
Assume starting at vertex S, needing shortest paths to all other vertices.
Each vertex visited only once due to absence of negative cycles.
Path P from S to a vertex Z has at most N-1 edges.
The DP algorithm conditions on the number of edges: introduce variable I (0 to N-1) indicating the max number of edges allowed in the path.
The subproblem, defined by parameters I and Z (Z being a vertex), is to find the function D of I, Z, representing the shortest path length from S to Z using at most I edges.
Recurrence is built from I = 0, incrementally up to I = N-1 for the final solution.
This implies that the DP approach iteratively expands the number of edges considered to solve the shortest path problem.

\subsection*{Single Source  Recurrence}
Defined a subproblem D(i,z) in dynamic programming to compute the shortest path from a source vertex s to a vertex z using at most i edges in a graph.
Established the base case where D(0,s) = 0, as no edges are used, and any path to another vertex is infinitely long.
Developed a recurrence that calculates D(i,z) by trying all predecessors y of z, and choosing the one that, when added to the edge weight W(y,z), minimizes the path length from s to z using i edges.
Acknowledged that using exactly i edges might not yield the overall shortest path, as fewer may suffice.
Therefore, adjusted the recurrence to minimize over paths using at most i edges, combining the best solution found with exactly i edges and the best found using fewer, i.e., min(D(i-1,y) + W(y,z), D(i-1,z)).
This accounts for the possibility of shorter paths that do not utilize all available edges.

\subsection*{Single Source  Summary}
Determining the shortest path from S to Z with a maximum of I edges involves a dynamic programming approach where the solution is established using the optimal scenarios for fewer edges.
For I=0, we have the base case.
When considering paths with I edges, the path could use either I-1 or exactly I edges.
To resolve this, we compare the shorter of two paths: one found in the table D at I-1 edges (D of I-1 Z) or the shortest path from S to a penultimate vertex Y plus the last edge from Y to Z, taking the minimum for all possible Y\@.
Recurrence relationship for D of I Z is then defined, where computation builds upon the subproblems involving I-1 edges.
This process is repeated, incrementing I from 0 up to N-1, to create a functional dynamic programming solution to find the shortest path.

\subsection*{Single Source  Pseudocode}
Developed in the `50s by Bellman and Ford, the Bellman-Ford algorithm finds the shortest paths from a single source in a weighted directed graph, handling negative edge weights.
The algorithm initializes the distance to the source (S) as zero and iteratively relaxes edges to find the shortest paths, updating distances for a total of n-1 iterations, where n is the number of vertices.
Each iteration inspects all edges and compares the current best-known distance with a new candidate path that includes the edge being relaxed; if the candidate is better, the best-known distance is updated.
The algorithm can also identify negative weight cycles, as such cycles would continue to reduce the path length even after n-1 iterations.
The overall time complexity is O(n*m), with n being the number of vertices and m the number of edges, which is less efficient than Dijkstra`s algorithm but is needed to accommodate graphs with negative weight edges and to detect negative weight cycles.
The algorithm involves reversing the graph for edge relaxation and requires linear time to construct the reversed graph, contributing to its computational cost.

\subsection*{Finding Negative Wt.
Cycle}
Detecting a negative weight cycle in a graph can be achieved using the Bellman-Ford algorithm.
This algorithm utilizes a table with columns representing graph vertices (S, A, B, C, D, E in the example) and rows for path lengths.
Starting with base case I=0, the source vertex S is set to 0 distance, and others are infinite.
The algorithm iteratively updates the table for path lengths I=1 to I=N (number of vertices).
Normally, the algorithm ends at I=N-1 as it considers paths using at most N-1 edges.
If there`s no negative weight cycle, paths will not improve beyond this point.
However, executing the algorithm for an additional iteration (I=N) may reveal shorter paths if a negative weight cycle exists, as this would continually decrease path lengths.
When I=N produces different values than I=N-1, a negative weight cycle is present.
The vertex involved in the cycle will show a decreased path length in row I=N\@.
By backtracking, the cycle`s participating vertices can be identified.
The Bellman-Ford algorithm thus finds the shortest paths and detects negative weight cycles in graphs with both positive and negative weight edges.

\subsection*{All Pairs Shortest Path}
Examining a variant of the shortest path prob., different from previous Bellman-Ford single source to all vertices.
Now, considering all pairs shortest paths in a directed graph with pos\. or neg\. edge weights.
Defined dist y, z as the length of the shortest path from y to z, expanding to include all vertex pairs.
Goal is to find shortest paths for all N2 vertex pairs.
A straightforward method to address this all pairs shortest path challenge is to apply Bellman-Ford algorithm N times, each with a different starting vertex, covering all vertex combinations.

\subsection*{Naive Approach Question}
Bellman-Ford algorithm running time is O(VE); running it N times for all pairs shortest path results in O(NVE) runtime.

\subsection*{Naive Approach Solution}
The Bellman-Ford algorithm is a single-source shortest path solver that employs dynamic programming and runs in O(nm) time, with n being the number of nodes and m the number of edges in a graph.
This runtime is achieved by using an approach that iterates through all possible numbers of edges in a path (up to n-1) and through all the edges in the graph.
Running Bellman-Ford n times results in a runtime of O(n\textasciicircum{}2m).
In contrast, the Floyd-Warshall algorithm, designed to tackle the same problem, operates with a time complexity of O(n\textasciicircum{}3), which can be more efficient since m might be as large as n\textasciicircum{}2.
The algorithms` names are not as crucial as understanding their dynamic programming concepts, which help reconstruct their running times and eventually the algorithms themselves.
Both algorithms utilize distinctive dynamic programming methods distinct from previous examples, with Floyd-Warshall being the more time-efficient choice for certain connected graphs.

\subsection*{All Pairs  Subproblem}
Constructing a dynamic programming (DP) algorithm for shortest paths in a graph requires a different approach than the Bellman-Ford algorithm`s edge count conditioning.
Instead, we consider a prefix of vertices as an input condition, ordering vertices from one to n, and using them as a basis for our DP approach.
This method involves creating a three-dimensional DP table D, indexed by i (the prefix cut-off for vertex set), s (the start vertex), and t (the end vertex).
The value D(i, s, t) represents the shortest path length from s to t using only vertices 1 through i as intermediates.
We iterate over all vertices pairs for s and t and all prefixes i from 0 (empty set) to n (all vertices) to fill in this table.
When i = n, D(n, s, t) indicates the shortest path from s to t over the entire graph.
The next step involves developing a recurrence to efficiently compute this DP table.

\subsection*{All Pairs  Base Case Question}
Defining D(IST) as the shortest path from vertex S to T, constrained by intermediate vertices from a prefix of series up to vertex I\@.
Base case and solution linked to the Bellman-Ford algorithm`s approach to single-source shortest paths.

\subsection*{All Pairs  Base Case Solution}
Base case in algorithm: i = 0, no intermediate vertices, direct path from s to t.

\subsection*{All Pairs  Recurrence}
Two scenarios for a path from S to T: directly connected by an edge or not.
If connected, shortest path D(0, S, T) equals edge length W(ST).
If not, shortest path without intermediates is infinite.
For allowed intermediates 1 through I, we seek D(I, S, T).
We define a recurrence to express D(I, S, T) in terms of smaller I values.
Two cases: path includes I as the last vertex or it doesn`t.
If it doesn`t include I, we examine this simpler scenario first.

\subsection*{Case  i not on path Question}
D(i, s, t) expression needed for a scenario where vertex i isn`t on the shortest path between s and t; this requires breaking it down into smaller subproblems.

\subsection*{Case  i not on path Solution}
When vertex i isn`t on path P, path P uses just vertices 1 to i-1 as intermediates, thus D(i, s, t) = D(i-1, s, t).
This is solved by the corresponding subproblem.
Conversely, if i is on path P, the scenario is different and needs closer examination.

\subsection*{Case  i is on path}
Developing a recurrence for D(i, s, t) considering i is on the shortest path from s to t, using only vertices 1 through i as intermediates.
The path includes four segments: from s to a subset of vertices 1 to i-1, around that subset, through vertex i, and back through a subset of vertices 1 to i-1 to t.
The subsets may be empty or include all vertices in that range.
The goal is to express D(i, s, t) by segmenting the path into these parts, which simplifies creating the recurrence.

\subsection*{Recurrence  i is on path Question}
Determine a recursion for D(i,s,t) based on smaller subproblems.

\subsection*{Recurrence  i is on path Solution}
To create the recurrence for D(i, s, t), we modify the parameter i to i-1 to utilize a previously solved subproblem in a dynamic programming approach.
For a path from s to i using the first i-1 vertices as intermediate points, we use D(i-1, s, i), while for i to t, we have D(i-1, i, t) with the same vertex set as intermediates.
The total path length combines these subproblems.
The recurrence formula for D(i, s, t) is thus the sum of D(i-1, s, i) and D(i-1, i, t).
We then determine the optimal solution by comparing scenarios where vertex i is included or excluded from the path.

\subsection*{Recurrence  Summary}
Dynamic programming approach to finding the shortest path involves recurrence for D(i,s,t), depending on whether a vertex i is on the path.
If i is on the path, compute the shortest path from start (s) to i using the first i-1 vertices and then from i to the target (t).
If i isn`t on the path, D(i,s,t) equals D(i-1, s, t).
Minimize between these two cases to find the shortest path.
The recurrence uses previously solved subproblems, thereby solving D(i,s,t) progressively for i from 0 to n in a bottom-up manner.
Next step is writing the pseudo-code for this dynamic programming algorithm.

\subsection*{All Pairs  Pseudocode}
Developed the Floyd-Warshall algorithm pseudo code for solving all pairs shortest path in a directed graph with potentially positive or negative edge weights.
Start with a base case, initializing distances D\[0\]\[S\]\[T\] for all vertex pairs (S, T) where S and T range from 1 to N\@.
If an edge exists from S to T, the base distance is the edge weight; if not, it`s set to infinity.
Progress to the general case for I \textgreater{}= 1 iterating over all S and T, filling in D\[I\]\[S\]\[T\] based on two scenarios: with or without vertex I on the path.
If I is not on the path, D\[I\]\[S\]\[T\] equals D\[I-1\]\[S\]\[T\].
If I is on the path, the path is split at I, taking the sum of the shortest paths from S to I and I to T\@.
The algorithm concludes by returning a matrix D\[N\] representing the shortest paths for all vertex pairs.
The algorithm`s time complexity analysis is implied but not detailed.

\subsection*{All Pairs  Running Time Question}
Analyze algorithm running time.

\subsection*{All Pairs  Running Time Solution}
Algorithm`s runtime analysis shows nested for loops with constant time operations.
Base cases involve two nested loops, size n, totaling O(n\textasciicircum{}2) time.
General cases have three nested loops, size n, totaling O(n\textasciicircum{}3) time.
Overall runtime is dominated by the O(n\textasciicircum{}3) general case.

\subsection*{Finding Neg.
Wt.
Cycle \#2}
Algorithm assumes no neg weight cycles, when present results incorrect.
Detection of neg cycles essential, example with cycle total -1 shown.
Algorithm revised to identify neg cycles; Focus on a vertex within cycle, outcomes D(n,a,a), D(n,b,b), D(n,c,c) should be -1.
Negative diagonal matrix entries indicate neg cycles.
Floyd-Warshall and Bellman-Ford algorithms detect neg cycles with diffs: former checks diagonals for all pairs, latter single source, both identify neg cycles.

\subsection*{Comparing Algorithms}
Modified example by changing edge direction from d to b.
The focus is on comparing Bellman-Ford and Floyd-Warshall algorithms for detecting negative weight cycles.
Bellman-Ford algorithm is for single-source shortest paths; if initiated from vertex d, it can`t detect a negative cycle like a-b-c since it`s not reachable from d.
Bellman-Ford only finds negative cycles reachable from its start vertex.
Floyd-Warshall, however, computes all-pairs shortest paths and can detect negative cycles anywhere in the graph irrespective of a starting vertex.
Therefore, Floyd-Warshall would find the negative weight cycle a-b-c even when Bellman-Ford could not from the starting vertex d.

\subsection*{DP3  Practice Problems}
Homework problem after a lecture involves detecting arbitrage opportunities using graph algorithms from Chapter Four, by converting currency exchange to the negative weight cycle problem.
Students should use existing algorithms for negative weight cycle detection as a non-modifiable `black box` subroutine to build an algorithm for arbitrage detection.
The task is to create a function that translates currency exchange rates into a graph format, then run the negative weight cycle detection on this graph.
The output may require conversion but should indicate if an arbitrage opportunity exists.
This problem exemplifies the concept of reduction, which will be a recurring theme in the course for designing algorithms and later for demonstrating NP-completeness and computational hardness.
This approach leverages a solution for a known problem to solve a new, related problem.

