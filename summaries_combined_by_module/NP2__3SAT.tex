\section*{3SAT}

\subsection*{NP-Completeness}
Proving 3SAT NP-complete, leveraging SAT`s known NP-completeness from Cook-Levin-Theorem (1971: Cook, Levin).
Importance of NP-completeness underscored by Karp`s 1972 paper, identifying 21 additional NP-complete problems.
Upcoming lectures to explore some of these 21, starting with 3SAT\@.

\subsection*{3SAT}
3SAT problem involves a boolean formula f in conjunctive normal form.
Number of variables n, clauses m.
3SAT`s specific constraint: each clause \textless{}= 3 literals.
Task: find a satisfying assignment of true/false to variables so f evaluates to true.
If no such assignment exists, no output.

\subsection*{Proof Outline}
3SAT problem NP-completeness to be proven.
Firstly, show 3SAT in NP class, considered straightforward.
Main task: reduce SAT to 3SAT as SAT is NP-complete.
Reduction implies any problem in NP reduces to SAT, and then to 3SAT\@.
Hence, a poly-time algorithm for 3SAT would solve all NP problems, due to transitive reductions.
Proof of 3SAT`s place in NP class begins first.

\subsection*{3SAT in NP}
3SAT problem proven to be in class NP by demonstrating efficient verifiability of solutions.
For a given input f and proposed true/false assignment for n variables, process entails checking each clause C--max three literals each.
Verification time per clause is constant, hence for M clauses, verification takes O(M) time total.
Assignment satisfaction of all clauses implies the entire formula f is satisfied, confirming 3SAT`s placement in NP\@.

\subsection*{SAT 3SAT}
Reducing SAT to 3SAT involves transforming a SAT problem`s input formula (f) into a 3SAT-compatible formula (f`).
The challenge is to ensure that large clauses in f, with possibly n literals, are converted into clauses of at most three literals.
The transformation must preserve the essential property that a satisfying assignment (s`) for f` can be converted into a satisfying assignment (s) for f, and vice-versa.
This bi-directional satisfaction ensures that if there are no satisfying assignments for f`, yielding a `No` from an algorithm, then f must also have no satisfying assignments, allowing a consistent `No` response for the original SAT problem.
Therefore, the goal is to construct an algorithm that uses 3SAT solutions to solve SAT problems efficiently and equivalently.

\subsection*{Example}
Converting SAT problem with 4 variables and 3 clauses (C1, C2, C3) into a 3-SAT problem: C1 and C3 remain unchanged as they are already valid for 3-SAT\@.
C2, which has 4 literals, is modified by introducing a new variable y, splitting C2 into two clauses of at most three literals each: (!x2  x3  y) and (!y  !x1  !x4), labeled as C2`.
The transformation maintains the satisfiability equivalence between the original clause C2 and the new pair C2`, ensuring the resulting formula is a valid 3-SAT representation with only clauses of size 3 or less.

\subsection*{Claim  Forward}
Proving bidirectional satisfiability between an original clause of size 4 (C\_2) and a new pair of clauses of size at most 3 (C`\_2).
Forward direction: given a satisfying assignment for C\_2, consisting of x\_2, x\_3, x\_1, x\_4, one of these must be true: x\_2 = false, x\_3 = true, x\_1 = false, or x\_4 = false.
Two cases align with C`\_2 clauses: 1) If x\_2 = false or x\_3 = true, set y = false, satisfying both new clauses.
2) If x\_1 = false or x\_4 = false, set y = true, also satisfying both C`\_2 clauses.
Therefore, a satisfying assignment for C\_2 can be used to construct a satisfying assignment for C`\_2, proving the clauses are satisfiable in the forward direction.

\subsection*{Claim  Reverse}
To show the reverse implication between satisfying assignments for C to the second prime and C to the second, consider the assignment that satisfies C to the second prime.
Ignoring variable y, such an assignment also satisfies C to the second in all cases.
If y is true, either x1 or x4 is false, satisfying C to the second.
If y is false, either x2 must be false or x3 must be true to satisfy C to the second prime, thus also satisfying C to the second.
Therefore, every satisfying assignment for C to the second prime corresponds to a satisfying assignment for C to the second, confirming the reverse implication and establishing the if and only if relationship.

\subsection*{Quiz  5-SAT 3-SAT Question}
Transformed a clause of size 5 into 3 clauses of size 3 by introducing 2 new vars y \& z.
To ensure original clause C`s satisfiability is equivalent to the new clauses C`, C is replaced by C` in the orig\. input F for an equiv\. formula.
New C` is to be defined with three literals each, using vars a-e (representing X2 bar, X3, X1 bar, X4 bar, X5) plus y \& z.
The task is to create C` satisfying the if-and-only-if satisfiability condition with C\@.

\subsection*{Quiz  5-SAT 3-SAT Solution}
Construct a clause in three literals from a clause with K literals using K-3 new vars and K-2 clauses.
Begins by assigning one literal`s complement (X\_1\_bar) to satisfy one clause, then set auxiliary vars Y, Z appropriately to satisfy remaining clauses.
Reverse by considering assignments for Y, Z, and infer the assignment for original clause C using Y`s and Z`s truth values.
This method ensures at least one literal in each clause of C\_prime corresponds to a literal in C; satisfying C\_prime will also satisfy C by appropriate assignments.
Generalize this by creating new vars and clauses based on the clause size K, where size four needs one new var and for five, two new vars.
This construction maintains the satisfaction relationship between the original and constructed sets of clauses.

\subsection*{Big Clauses}
General k-sized clauses labeled a1 to ak generate k-3 new vars Y1 to Yk-3, distinct for each clause, yielding n*m potential new vars across m clauses.
Original clause C replaced by k-2 clauses; pattern starts with a1, a2, Y1, and alternates negations of newly introduced Y vars with subsequent a literals.
Last clause differs, ending with ak-1, ak.
This forms C\_prime formula.
Claim: original clause C`s satisfiability equates to C\_prime`s satisfiability, allowing transformation of any clause \textgreater{}3 literals into 3-literal clauses, preserving satisfiability, critical for reducing to a three SAT problem.
Proof of claim required to verify reduction validity.

\subsection*{General Claim  Forward}
Constructed a formula C with k literals and a related formula C` with k-2 clauses.
Claimed C satisfiable iff C` satisfiable.
Proved forward implication; an assignment satisfying C can be adjusted, using auxiliary variables, to satisfy C`.
Satisfying a literal ai in C ensures the (i-1)th clause in C` is satisfied.
For earlier clauses in C`, set auxiliary variables y1 to yi-2 to true.
For later clauses, set auxiliary variables yi-1 to yk-2 to false.
Thus, if at least one literal in C is true, the corresponding adjustments to auxiliary variables ensure C` is also satisfied.
The reverse implication remains to be demonstrated.

\subsection*{General Claim  Reverse}
Proving the reverse implication shows that a satisfying assignment for C` guarantees a satisfying assignment for C\@.
For C` satisfied by an assignment of K literals plus auxiliary variables, we assume all K literals are false and deduce a contradiction, thereby invalidating the assumption.
Initial clauses necessitate setting auxiliary variables Y1 to Yk-3 to true.
But with these and all K literals false, the last clause fails, contradicting C` satisfaction.
This means at least one K literal must be true, satisfying C, independent of auxiliary variable assignments.

\subsection*{SAT 3SAT}
Formalized reduction from SAT to 3SAT involves handling clauses of an input formula: small clauses with \textless{}=3 literals are kept unchanged; larger clauses are split into (k-2) new clauses using (k-3) new variables, as per a defined construction.
Proving an input formula f is satisfiable iff its transformed version f` in 3SAT is satisfiable is straightforward.
This is due to the fact that each clause C and its transformed version C` maintain the same satisfiability condition.
To show a satisfying assignment for f` can be converted into one for f, auxiliary variables are disregarded, and the original variables` assignment ensures f`s satisfiability.

\subsection*{Correctness}
To demonstrate the equivalence between the original formula f and the new formula f`, it`s established that an assignment satisfying f can also satisfy f` and vice versa.
When a clause C in f has more than three literals, it`s replaced by a sequence called C` with additional variables exclusive to C`.
These extra variables can be set independently without impacting other clauses.
For C`, a specific assignment strategy achieves satisfaction: taking the first satisfied literal in C, setting the preceding variables to true, and the rest to false satisfies C`.
Conversely, a satisfying assignment to f` implies at least one literal in the original clause C must be true, as C` couldn`t be satisfied otherwise.
Therefore, the original variables` assignment satisfies f, disregarding the new variables.
This forms a bidirectional satisfaction, thereby proving the formulas` equivalence.

\subsection*{Satisfying Assignment}
Transformed input SAT formula into 3SAT formula, denoted as f prime.
Proved f is satisfiable iff f prime is satisfiable.
If 3SAT algorithm finds f prime unsatisfiable, then f is unsatisfiable.
If it finds a satisfying assignment for f prime, ignore new variable assignments and retain original ones for a satisfying assignment for f, completing the reduction.
F prime, in the worst case, has nm variables and nm clauses due to a potential n new variables and n new clauses per original clause, yet the size of f prime is still polynomially related to f.
Therefore, the a algorithm`s polynomial running time holds relative to the size of both f and f prime, concluding the NP-completeness proof.

\subsection*{Practice Problems}
First NP completeness proof observed: 3-SAT is NP-complete.
Relevant practice problems:
1) 8.3 involves Stingy SAT--objective to prove NP completeness.
2) 8.8 deals with Exact 4-SAT--each clause contains exactly four literals; aim to prove NP completeness.

