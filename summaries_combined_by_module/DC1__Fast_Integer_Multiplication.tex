\section*{Fast Integer Multiplication}

\subsection*{Divide and Conquer}
Divide and Conquer (D\&C) algorithms, such as binary search or merge sort, demonstrate their power in handling complex large-scale problems which are beyond the capacity of hardware implementations.
Addressing multiplication of two n-bit numbers, often running into thousands of bits as in RSA encryption, D\&C offers a faster solution than standard methods.
This approach also solves finding the median in an unsorted array without sorting it.
Additionally, the Fast Fourier Transform (FFT) algorithm, a cornerstone in fields like signal processing and hailed as a pivotal numerical tool since 1994, hinges on understanding complex numbers.
Despite the initial complexity, mastering FFT reveals its elegance and simplicity, and investing time and effort in learning it proves rewarding, as it is considered a masterpiece in the realm of algorithms.

\subsection*{D\&C  Overview}
Divide and conquer techniques expounded with RSA algorithm examples: fast modular exponentiation and Euclid`s GCD\@.
Exploring n-bit integer multiplication vital for large integers in RSA, where standard hardware operations are insufficient.
Traditional multiplication takes n\textasciicircum{}2 time; aim for a faster algorithm.
Also covering linear-time median calculation and Fast Fourier Transform (FFT).
Assumed prior knowledge includes merge sort and solving recurrences, with textbook resources for refreshment.
Next focus: efficient n-bit integer multiplication.

\subsection*{Multiplying Complex \#s}
Exploring Gauss`s method in processing n-bit integer multiplication: Op cost--a key constraint.
Multiplication proves costly, effort aimed at minimizing mult.
Count while add/sub considered cheap, abundant use tolerated.
Insight applied to complex number multiplication (a+bi)*(c+di), expanding requires four real mults\. traditionally: ac, bd, bc, ad.
Goal: reduce mult\. without sacrificing outcome.
Gauss`s strategy: without calculating bc and ad separately, compute their sum directly, reducing required mults\. from four to three without compromising the product calculation of the two complex numbers.
This technique demonstrates optimization potential in computation, vital in cases where certain ops, such as mult., are significantly more resource-intensive than others like add/sub.

\subsection*{Improved Approach}
Complex number multiplication traditionally requires four real multiplications: ac, bd, bc, and ad.
Goal: reduce to three real multiplications to calculate the product of two complex numbers, a+bi and c+di.
Achieved by computing bc + ad, not individually but as part of an expression.
The method: use the fact that bc + ad are the cross terms in the expansion of (a + b)(c + d).
With expressions rearranged, bc + ad equals (a + b)(c + d) - ac - bd.
The three necessary real multiplications are then: ac, bd, and (a + b)(c + d).
These calculations allow us to determine ac, bd, and additively bc + ad to get the complex product.
This reduction in multiplications applies not only to complex numbers but can accelerate multiplying large n-bit integers beyond the standard quadratic time.
The essence of this approach is to consider the sum of the real and imaginary parts (a + b) and (c + d) of each complex number and multiply them to obtain a key term used to simplify the overall calculation, revealing the clever and efficient nature of the algorithm.
An example with (5 + 3i)(7 - 6i) hints at using these sums, 8 and 1, to efficiently find the product.

\subsection*{D\&C  Naive Approach}
Exploring n-bit integer multiplication using divide and conquer mimics merge sort: split inputs (n is a power of 2 for simplicity) in halves, resolve recursively, then combine\. x and y split into x\_l/x\_r and y\_l/y\_r (left/right n/2 bits).
Demonstrated with binary 182 (1011 0110), split into x\_l (1011 or 11) and x\_r (0110 or 6), with 182 equaling 11 times 2\textasciicircum{}4 plus 6.
This shows x (or y) as x\_l shifted by n/2 (multiplied by 2\textasciicircum{}(n/2)) plus x\_r.
The method aims to simplify computations by handling smaller parts sequentially, leaning on the relations between original and half-sized numbers stemming from binary representation and positional value.

\subsection*{Naive  Recursive idea}
Recursive approach to multiply n-bit numbers x and y by splitting each into two n/2-bit numbers: xl (first n/2 bits of x), xr (last n/2 bits of x), yl (first n/2 bits of y), and yr (last n/2 bits of y)\. x equals xl * 2\textasciicircum{}(n/2) + xr and y equals yl * 2\textasciicircum{}(n/2) + yr.
Multiplying x and y yields 2\textasciicircum{}n * xl * yl + 2\textasciicircum{}(n/2) * (xl * yr + yl * xr) + xr * yr.
To compute x * y, recursively compute products of n/2-bit numbers: xl * yl, xl * yr, xr * yl, and xr * yr, leading to a recursive algorithm for the larger product.
This method needs refinement for efficiency.

\subsection*{Naive  Pseudocode Question}
Divide and conquer algorithm for n-bit int multiplication, assuming n=2\textasciicircum{}k.
Split x into XL (first n/2 bits) and XR (last n/2 bits), y into YL and YR similarly.
Compute products: XL*YL=A, XR*YR=B, XL*YR=C, XR*YL=D\@.
Combine as Z=x*y=2\textasciicircum{}n*A + 2\textasciicircum{}(n/2)*(C+D) + B\@.
Running time not detailed, but algorithm recursively computes four n/2-bit multiplications and combines them to produce result Z\@.

\subsection*{Naive  Running Time}
Algorithm`s run time analysis reveals partitioning x and y into halves takes O(n) time.
Recursively compute A, B, C, D, each n/2 bit pairs, total 4*T(n/2).
Computing final product z requires O(n) time for three additions of n bit numbers and shifting for multiplication by powers of 2, hence also O(n).
Combination of recursive subsolutions and additional operations results in T(n)= 4*T(n/2) + O(n) running time, which solves to O(n\textasciicircum{}2), the same as straightforward multiplication.
Improvement hinges on reducing four subproblems to three, leveraging Gauss`s idea.

\subsection*{D\&C  Improved Approach}
The text outlines a divide and conquer algorithm for multiplying two numbers, X and Y\@.
Initially, X times Y is expressed as a sum involving four sub-products based on left (l) and right (r) partitions of X and Y\@.
The traditional method would compute these four sub-products (x\_l*y\_l, x\_l*y\_r, x\_r*y\_l, x\_r*y\_r) recursively, leading to an O(n\textasciicircum{}2) time algorithm.
The goal is to reduce the subproblems from four to three, inspired by Gauss`s method for complex numbers.
By defining cross terms (x\_l*y\_r + x\_r*y\_l), they can be computed without calculating each individually.
This is achieved by multiplying the sums (x\_l+x\_r) and (y\_l+y\_r) and then subtracting the previously calculated recursive products (A = x\_l*y\_l and B = x\_r*y\_r) from the result (referred to as C).
With these three recursive computations (A, B, and C), they can be combined using the expression 2\textasciicircum{}n*A + 2\textasciicircum{}(n/2)*(C - A - B) + B to find X times Y\@.
This improved algorithm has only three subproblems to recursively solve, making it a more efficient approach.

\subsection*{Improved  Pseudocode Question}
Developed an enhanced multiplication algorithm named Fast Multiply, improving the previous order n square time methodology by modifying the end section.
Maintains original setup: inputs two n-bit integers (x and y), outputs their product.
Inputs split into two halves, XL, XR for x and YL, YR for y.
Fast Multiply recursively computes three products--XL times YL (A), XR times YR (B), and (XL + XR) times (YL + YR) (C)--instead of four.
Applies Gauss`s technique for combining results: multiplies A by 2\textasciicircum{}n, adds product of (C - A - B) by 2\textasciicircum{}(n/2), and adds B\@.
Outputs Z, the final product, significantly reducing subproblems to three.

\subsection*{Improved  Running Time}
New algorithm divides a problem into three sub-problems, each involving the multiplication of n/2-bit numbers, recursively computed.
The combination step of these sub-problems takes O(n) time.
By solving the recurrence relationship with the substitution method, we observe it creates a geometric series with log2n terms, which leads to an increasing series where the last term is dominant.
The complexity is therefore determined to be O(n * (3/2)\textasciicircum{}log2n).
Upon simplifying, the complexity translates to O(n\textasciicircum{}(log2\textasciicircum{}3)), which evaluates to O(n\textasciicircum{}1.59) through the conversion of the base of the exponent.
Hence, the algorithm improves from O(n\textasciicircum{}2) to O(n\textasciicircum{}1.59).
Further reduction of the exponent is possible by segmenting the input into more parts; however, this increases the constant hidden in the Big O notation, making the combination of solutions more computational intensive.

\subsection*{Improved  Summary}
Algorithm demonstrates a clever multiplication technique using binary representation.
For illustrative purposes, consider X=182 and Y=154, in binary: 10110110 and 10011010, respectively.
The approach divides X into XL (1011, or 11 in decimal) and XR (0110, or 6), and Y into YL (1001, or 9) and YR (1010, or 10).
It computes XL*YL (11*9=99), XR*YR (6*10=60), and (XL+XR)*(YL+YR) = 17*19 = 323.
The algorithm then follows a specific formula: Multiply XL*YL by 2\textasciicircum{}n (where n=8), subtract XL*YL and XR*YR from the (XL+XR)*(YL+YR) product, multiply the result by 2\textasciicircum{}(n/2), and add XR*YR to get the final product of X*Y, equaling 28,028.
The process works with seemingly unrelated numbers, like 17 and 19 from our example, a non-intuitive aspect of the algorithm.
Although further details on similar matrix multiplication (Strassen`s algorithm) and the upcoming linear-time median algorithm are available, they`re not covered here due to complexity.
This multiplication method represents a fast way to multiply n-bit integers.

