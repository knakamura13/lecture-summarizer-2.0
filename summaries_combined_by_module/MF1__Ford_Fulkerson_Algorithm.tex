\section*{Ford Fulkerson Algorithm}

\subsection*{Max-Flow}
Exploring the Max-flow graph problem; focusing on the Ford Fulkerson Algorithm and proving its correctness.
Will also cover the Max-flow min-cut theorem.
Application to computer vision via Image Segmentation will be discussed.
To address more complex scenarios, will delve into Max-flow with demand constraints and methods to simplify back to the original problem.
Concluding with speed improvements brought by the Edmonds-Karp Algorithm.

\subsection*{Lecture Outline}
Focused on solving the max flow prob.
Started with Ford-Fulkerson (FF) algo, progressed to Edmonds-Karp (EK) algo; EK runtime analysis more complex.
Introduced the max flow = min-cut theorem, foundational for algo correctness and applications, including image segmentation prob where an image is divided into foreground and background using max flow and the aforementioned theorem.

\subsection*{Problem Setup}
Maximizing supply transfer from start vertex s to end vertex t in a directed graph, ensuring not to surpass edge capacities, whether dealing with data like internet traffic where capacity equates to bandwidth, or physical goods like oil where capacity reflects pipeline limits.

\subsection*{Problem Formulation}
Max flow problem defined as optimizing flow within a flow network, comprised of a directed graph G with source (s) and sink (t) vertices.
Each edge has a set capacity, a positive value.
Objective: maximize flow originating at s and terminating at t by determining optimal flow (f\_sub\_e) along each edge.

\subsection*{Max-Flow Problem}
The max-flow problem involves determining the highest possible flow from a source vertex (s) to a sink vertex (t) in a directed graph that constitutes a flow network.
Edges in this graph have assigned capacities (c\_sub\_e), and the flow (f\_sub\_e) on each edge must not surpass these capacities and must be non-negative.
To satisfy conservation of flow, the input flow to any vertex (except s or t) must equal the output flow, ensuring no flow is lost or created within the network.
Flow is computed as the sum of incoming flows (f\_sub\_wv) to vertex v and the sum of outgoing flows (f\_sub\_vz) from vertex v.
Valid flow must adhere to both capacity and conservation constraints across the network.

\subsection*{Max-Flow Goal}
Max flow problem aims to identify a valid flow that adheres to capacity and conservation constraints, and maximizes the total flow sent.
`Maximum size` refers to this total, measured by the flow out of the source (s) or into the sink (t).

\subsection*{Quiz  Max-Flow Example Question}
Understand max-flow problem formulation by reviewing a flow network example with defined start/end vertices and edge capacities.
Determine a max-sized, valid flow conforming to two constraints: edge capacity limits and conservation of flow at vertices, barring the start (S) and end (T).

\subsection*{Quiz  Max-Flow Example Solution}
Max flow specified by green numbers on edges with red capacity numbers.
Flows in and out of vertices equal (c=7, e=2, d=7).
Flows don`t exceed edge capacities.
Total flow size 12 units from source s to sink t, equals total incoming edge capacity to t--hence, max flow achieved.

\subsection*{Cycles are OK}
Cycles in flow networks, such as C-F-E-D-C, don`t impede max flow calculations as they might in shortest path problems with negative weight cycles.
Utilizing cycles in a flow network isn`t beneficial since flow could bypass the cycle altogether, showing a direct path is more efficient.
While parts of a cycle might be used, it`s not clear without analysis which portions might be beneficial.
The presence of cycles in flow networks does not pose issues; they`re permissible and do not complicate the max flow problem-the problem remains well-defined with or without cycles.

\subsection*{Anti-parallel Edges}
To solve max-flow problems, simplifying the flow network by removing anti-parallel edges is proposed.
Simplification helps streamline upcoming algorithm development.
Example given: a network with A-B and B-A edges is reworked by splitting one of these edges, introducing an auxiliary vertex, and preserving the original capacity, thus eliminating anti-parallel edges.
The capacity distribution ensures the equivalency of max-flow between the two networks, enabling interchangeable solutions for the max-flow problem.
This advancement aids in clean input preparation for efficient max-flow algorithm application.

\subsection*{Toy Example}
Created a toy network with four vertices to test max flow algorithms, identified a max flow of size 17 which matches the network`s capacities from source (s) to sink (t), and will use this simple example throughout the development of algorithms.

\subsection*{Simple Algorithm}
Outline a basic algorithm using a flow network where we initialize flow to zero.
Capacities equal input capacities initially because flow is zero.
Find any s-t path (from source `s` to sink `t`) by running BFS or DFS to identify paths with available capacity.
The path doesn`t need to be optimal; simply must have available capacity on all edges.
Once a path is found, calculate minimum available capacity on this path, which is the max flow that can be sent through it.
Augment this flow by the calculated amount, adjust the flow on the path accordingly, update available capacities, and repeat the process.
Continue until no more s-t paths can be found with available capacity.
The algorithm stops when no further augmentation is possible, which may yield a sub-optimal flow, such as the described example ending with a flow of 10 whereas the network has the capacity for a flow of 17.

\subsection*{Backward Edges}
Identifying a gap in available capacities graph concerning the optimal flow solution.
Necessary to consider the reverse path b-to-a for flow augmentation, as it`s not represented in the initial graph.
Conceptualize flow as a faucet; increasing flow a-to-b opens it, while b-to-a closes it, reducing flow a-to-b.
Adding this reverse path with capacity equal to the existing flow (10 units) from a-to-b allows for such adjustment.
This introduces the concept of the residual network, which accurately reflects all possible flow adjustments, unlike the available capacities graph.
By utilizing the residual network, a path from s-to-t can be found with a capacity for 7-unit augmentation, thus altering flows from s-to-b to 7 units, reducing a-to-b from 10 to 3 units, and increasing a-to-t by 7 units to reach max flow in the example.
Essential to focus on the residual network to effectively identify and implement optimal flow adjustments.

\subsection*{Residual Network}
Residual networks are tied to the flow network in max-flow problems, altering based on the current flow, denoted as fe.
The residual network, expressed as G\textasciicircum{}f, maintains the flow network`s vertices but has modified edges dependent on fe, with differing weights/capacities.
These edges include forward edges, mirroring the flow network but with capacities reduced by fe, and backward edges, existing only if there`s positive flow in the original network, their capacities equating to fe, effectively allowing flow negation.
Residual networks also necessitate removing antiparallel edges from the original flow network to accommodate both forward and backward edges without issue.

\subsection*{Ford-Fulkerson Algorithm}
The Ford-Fulkerson algorithm approaches maximum network flow by utilizing residual networks.
Start with a zero initial flow and construct the residual network which mirrors the flow network initially.
Employ DFS or BFS to find a path from source (s) to sink (t); absence of a path halts the algorithm and outputs the current flow as max flow.
If a path is found, identify its minimum capacity, denoted c(P), and augment the flow along this path by c(P), increasing flow on forward edges and decreasing on backward edges.
Repeat constructing the residual network and searching for an s-t path.
Continue until no s-t path exists, then output the current flow as the max flow solution.

\subsection*{Running Time}
Correctness of the discussed algorithm linked to the Max-flow=min-cut theorem, proof deferred.
Focus on algorithm`s run time analysis under the condition that capacities must be integers.
Relevance of this assumption tied to maintaining integer capacity in the residual network and ensuring each augmentation increases flow by at least one unit.
Flow augmented by a minimum of one unit each iteration, implying a max of C rounds for the algorithm, where C represents the integer value of Max-flow sought.
Upcoming Edmonds-Karp algorithm to address limitations of integer capacity assumption.

\subsection*{Time per Round}
Ford-Fulkerson algorithm efficiency analyzed.
Time for one round (steps 2--5): update residual network, check for st-path, augment flow.
Residual network update takes order N time due to path length \textless{}= N-1 edges\. st-path checked using DFS/BFS, linear time order N+M; bound by order M if edges \textgreater{}= N-1.
Augmenting flow also order N time.
One round time is determined by st-path check, order N\@.
Total running time is order N per round, multipled by capital C (max flow increase).
Thus, Ford-Fulkerson algorithm requires time order N*C\@.

\subsection*{Discussion}
Ford-Fulkerson algorithm for max-flow problem has a pseudo-polynomial running time of O(MC), depending on integer capacities (C).
This is problematic as ideal algorithms should not be influenced by input magnitudes.
Edmonds-Karp modifies Ford-Fulkerson by using the shortest path (minimum edges) in the residual network, leading to O(M2N) run time, which doesn`t depend on max-flow size or integer capacities.
Orlin`s 2013 algorithm further improves the runtime to O(MN), the best known for max-flow on general graphs.

